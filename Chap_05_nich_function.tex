%\chapter{णिच् function}

\section{\texthindi{णिच्} function}
णिच् pratyaya is used in the sense of ‘inspiration’ for 1st to 10th gaṇa. It is easier to understand the syntax of a causative sentence in relation to a non-causative or a pre-causative sentence. Generally, in a causative construction, there are two agents, a) the agent of original action, and b) the agent of causation or instigation. If the original verb is transitive, then there may be an object of the original verb.\\
The णिच् pratyaya is enlisted among the sanādi pratyayas. The causative of a tenth gaṇa verb may be identical with the original in form, though different in meaning. One can have, in theory, numerous degrees of causativeness, and yet the outward form reains the same:\\
\texthindi{गच्छति }    “X goes.”\\
\texthindi{गमयति }  “Y makes X go.”\\ 
\texthindi{गमयति} “Z makes Y make X go.”\\ 
\texthindi{गमयति} “A makes Z make Y make X go.”\\ 
------
------
The difference between different degrees of causativeness is apparent from the form itself, but must be understood from the syntax of the rest of the sentence (Madhav M. Deshpande, 2007, pp. 341-342). Now this \texthindi{णिच्} pratyaya can be added to a derived dhātu \texthindi{णिच्}(x) again.\\

In this Chapter, the results obtained from all the algorithms are presented in separate sub-sections along with the detailed investigations using topographic and observatory measurements. Multi-temporal analyses of the results are also presented. 

\subsection{Definition of \texthindi{णिच्} function}
\texthindi{णिच्}: set of dhātus → set of derived dhātus/ prātipadikas of \texthindi{णिच्}

\subsection{Cases for \texthindi{णिच्} function}

\textbf{Case I:}\\
$x(1)=c$, $x(2)=c$/\texthindi{आ/ई/ऊ/ए}\\
\begin{equation}
	{\texthindi{णिच्}}(x) = x + \text{\texthindi{ इ}} 
\end{equation}


\begin{table}[h!]
	\begin{center}
		\begin{tabular}{ |c|c|c|c| } 
			\hline
			$x$ & $x(1)$ & $x(2)$ & \texthindi{णिच्($x$)}\\
			\hline
			\texthindi{स्पर्ध्}&$c$&$c$&\texthindi{स्पर्धि}\\
			\texthindi{नाथ्}&$c$&\texthindi{आ}&\texthindi{नाथि}\\
			\texthindi{ शीक् }&$c$&\texthindi{ ई }&\texthindi{ शीकि }\\
			\texthindi{ तूष् }&$c$&\texthindi{ ऊ}&\texthindi{ तूषि }\\
			\texthindi{ एध् }&$c$&\texthindi{ ए }&\texthindi{ एधि }\\
			\hline
		\end{tabular}
		\caption{Examples of \texthindi{णिच्} function Case I}
		\label{table:6.1}
		
	\end{center}
\end{table}

\textbf{Case II:}\\
$x(1)=c$, $x(2)=\text{\texthindi{अ}}$
\begin{equation}
	\text{\texthindi{णिच्}}(x) = x[\text{\texthindi{अ }}\xrightarrow{2} \text{\texthindi{आ}}]+ \text{\texthindi{इ}}  
\end{equation}


\begin{table}[h!]
	\begin{center}
		\begin{tabular}{ |c|c|c|c| } 
			\hline
			$x$ & $x(1)$ & $x(2)$ & \texthindi{णिच्($x$)}\\
			\hline
			\texthindi{ दद् }&$c$&$c$&\texthindi{ दादि }\\
			\texthindi{दध्}&$c$&\texthindi{अ}&\texthindi{दाधि}\\
			\hline
		\end{tabular}
		\caption{Examples of \texthindi{णिच्} function Case II}
		\label{table:6.2}
	\end{center}
\end{table}


\textbf{Case III:}\\
$x(1)=c$, $x(2)=$\texthindi{इ/उ/ऋ }
\begin{equation}
	\text{\texthindi{णिच्}}(x) = x[\text{\texthindi{ इ उ ऋ }}\xrightarrow{2} \text{\texthindi{ ए ओ अर् }}]+ \text{\texthindi{इ}}  
\end{equation}

\begin{table}[h!]
	\begin{center}
		
		\begin{tabular}{ |c|c|c|c| } 
			\hline
			$x$ & $x(1)$ & $x(2)$ & \texthindi{णिच्($x$)}\\
			\hline
			\texthindi{ पिट् }&$c$&\texthindi{ इ }&\texthindi{ पेटि }\\
			\texthindi{ मुद् }&$c$&\texthindi{ उ }&\texthindi{ मोदि }\\
			\texthindi{ भृज् }&$c$&\texthindi{ ऋ }&\texthindi{ भर्जि }\\
			\hline
		\end{tabular}
		\caption{Examples of \texthindi{णिच्} function Case III}
		\label{table:6.3}
	\end{center}
	
\end{table}

\textbf{Case IV:}\\
$x(1) =$ \texthindi{इ/ई उ/ऊ ऋ}
\begin{equation}
	\text{\texthindi{णिच्}}(x) = x[\text{\texthindi{ इ/ई उ/ऊ ऋ }}\xrightarrow{1} \text{\texthindi{ आय् आव् आर् }}]+ \text{\texthindi{इ}}
\end{equation}


\begin{table}[h!]
	\begin{center}
		\begin{tabular}{ |c|c|c| } 
			\hline
			$x$ & $x(1)$ & \texthindi{णिच्($x$)}\\
			\hline
			\texthindi{ क्षि }&\texthindi{ इ }&\texthindi{ क्षायि }\\
			\texthindi{ नी }&\texthindi{ ई }&\texthindi{ नायि }\\
			\texthindi{ घु }&\texthindi{ उ }&\texthindi{ घावि }\\
			\texthindi{ सू }&\texthindi{ ऊ }&\texthindi{ सावि }\\
			\texthindi{ वृ }&\texthindi{ ऋ }&\texthindi{ सावि }\\
			\hline
		\end{tabular}
		\caption{Examples of \texthindi{णिच्} function Case IV}
		\label{table:6.4}
	\end{center}
\end{table}


\textbf{Case V:}\\
$x(1) =$ \texthindi{आ}
\begin{equation}
	\text{\texthindi{णिच्}}(x) = x + \text{\texthindi{ प् }} +  \text{\texthindi{इ}}
\end{equation}

\begin{table}[h!]
	\begin{center}
		
		\begin{tabular}{ |c|c|c| } 
			\hline
			$x$ & $x(1)$ & \texthindi{णिच्($x$)}\\
			\hline
			\texthindi{ का }&\texthindi{ आ }&\texthindi{ कापि }\\
			\hline
		\end{tabular}
		\caption{Examples of \texthindi{णिच्} function Case V}
		\label{table:6.5}
	\end{center}
	
\end{table}

\textbf{Case VI:}\\ 
This case only appears in \texthindi{चुरादिर्गण} that is the 10th gaṇa\\
$x(1)=$\texthindi{अ}\\
\begin{equation}
	\text{\texthindi{णिच्}}(x) = x[\text{\texthindi{अ}}\xrightarrow{1}\phi]+ \text{\texthindi{इ}}
\end{equation}

\begin{table}[h!]
	\begin{center}
		
		\begin{tabular}{ |c|c|c| } 
			\hline
			$x$ & $x(1)$ & \texthindi{णिच्($x$)}\\
			\hline
			\texthindi{ रच }&\texthindi{ अ }&\texthindi{ रचि }\\
			\hline
		\end{tabular}
		\caption{Examples of \texthindi{णिच्} function Case VI}
		\label{table:6.6}
	\end{center}
	
\end{table}

\textbf{Case VII:}\\
If $x є$ [\texthindi{क्रम् तम् रम् शम् श्रम् जन् वध्}]
\begin{equation}
	\text{\texthindi{णिच्}}(x) = x + \text{\texthindi{इ}}  
\end{equation}

\begin{table}[h!]
	\begin{center}
		
		\begin{tabular}{ |c|c| } 
			\hline
			$x$ & \texthindi{णिच्($x$)}\\
			\hline
			\texthindi{ क्रम् }&\texthindi{ क्रमि }\\
			\texthindi{ तम् }&\texthindi{ तमि }\\
			\texthindi{ रम् }&\texthindi{ रमि }\\
			\texthindi{ शम् }&\texthindi{ शमि }\\
			\texthindi{ श्रम् }&\texthindi{ श्रमि }\\
			\texthindi{ जन् }&\texthindi{ जनि }\\
			\texthindi{ वध् }&\texthindi{ वधि }\\
			\hline
		\end{tabular}
		\caption{Examples of \texthindi{णिच्} function Case VII}
		\label{table:6.7}
	\end{center}
	
\end{table}

\textbf{Case VIII:}\\
If $x$ $\epsilon$ [\texthindi{ऋ ह्री व्ली री क्नूयी क्ष्मायी}]\\
\begin{equation}
	\text{\texthindi{णिच्}}(x) =c(1^{\prime}) + c(2^{\prime})* +v[\text{\texthindi{ ऋ ई उ/ऊ }}\xrightarrow{1^{\prime}} अ\text{\texthindi{ र् ए ओ}}] + \text{\texthindi{ प् }}+ \text{\texthindi{इ}}
\end{equation}
\\
$c(2^{\prime})$ will be added only when $x(2^{\prime})$ is not a vowel.
\begin{table}[h!]
	\begin{center}
		\begin{tabular}{ |c|c|c| } 
			\hline
			$x$ & $c(1^{\prime}) + c(2^{\prime})* + v[\text{\texthindi{ऋ ई ऊ}} \xrightarrow{1^{\prime}} \text{\texthindi{अर् ए ओ}}]$ & \texthindi{णिच्($x$)}\\
			\hline
			\texthindi{ ऋ }&\texthindi{ अर् }&\texthindi{ अर्पि }\\
			\texthindi{ ह्री }&\texthindi{ ह्रे }&\texthindi{ ह्रेपि }\\
			\texthindi{ व्ली }&\texthindi{ व्ले }&\texthindi{ व्लेपि }\\
			\texthindi{ री }&\texthindi{ रे }&\texthindi{ रेपि }\\
			\texthindi{ क्नूयी }&\texthindi{ क्नो }&\texthindi{ क्नोपि }\\
			\texthindi{ क्ष्मायी }&\texthindi{ क्ष्मा }&\texthindi{ क्ष्मापि }\\
			\texthindi{ रुह्}&\texthindi{ रो }&\texthindi{ रोपि }\\
			\hline
		\end{tabular}
		\caption{Examples of \texthindi{णिच्} function Case VI}
		\label{table:6.8}
	\end{center}
	
\end{table}

\textbf{Case IX:}
\\If $x$ $\epsilon $ [\text{\texthindi{शो छो षो ह्वेञ् व्येञ् वेञ् पा}}]\\
\begin{equation}
	\text{\texthindi{णिच्}}(x) =c(1^{\prime}) + c(2’)* +v[\text{\texthindi{ ओ ए }}\xrightarrow{1’} अ\text{\texthindi{ आ }}] + \text{\texthindi{ य् }}+ \text{\texthindi{इ}}
\end{equation}
$c(2’)$ will be added only when $x(2’)$ is not a vowel.\\


\begin{table}[h!]
	\begin{center}
		\begin{tabular}{ |c|c|c| } 
			\hline
			$x$ & $c(1^{\prime}) + c(2’)* + v[\text{\texthindi{ओ ए}} \xrightarrow{1’} \text{\texthindi{आ }}]$ & \texthindi{णिच्($x$)}\\
			\hline
			\texthindi{ शो}&	\texthindi{ शा}&	\texthindi{ शायि}\\
			\texthindi{ छो}&	\texthindi{ छा}&	\texthindi{ छायि}\\
			\texthindi{ षो}&	\texthindi{ सा}&	\texthindi{ सायि}\\
			\texthindi{ ह्वेञ्}&	\texthindi{ ह्वा}&	\texthindi{ ह्वायि}\\
			\texthindi{ व्येञ्}&	\texthindi{ व्या}&	\texthindi{ व्यायि}\\
			\texthindi{ वेञ्}&	\texthindi{ वा}&	\texthindi{ वायि}\\
			\texthindi{ पा}&	\texthindi{ पा}&	\texthindi{ पायि}\\
			\hline
		\end{tabular}
		\caption{Examples of \texthindi{णिच्} function Case IX}
		\label{table:6.9}
	\end{center}
\end{table}




\textbf{Case X:}\\
If $x$ $\epsilon$ [\texthindi{स्फायी}]\\
\begin{equation}
	\text{\texthindi{णिच्}}(x) = c(1^{\prime}) + c(2^{\prime}) +v(1^{\prime})+ \text{\texthindi{ व् }}+ \text{\texthindi{इ}}
\end{equation}
\begin{table}[h!]
	\begin{center}
		\begin{tabular}{ |c|c|c| } 
			\hline
			$x$&	$c(1^{\prime}) + c(2^{\prime}) + v(1^{\prime})$&	\texthindi{णिच्($x$)}\\
			\hline
			\texthindi{ स्फायी}&	\texthindi{ स्फा}&	\texthindi{ स्फावि}\\
			\hline
		\end{tabular}
		\caption{Examples of \texthindi{णिच्} function Case X}
		\label{table:6.10}
	\end{center}
	
\end{table}

\textbf{Case XI:}\\
If $x$ $\epsilon$ [\texthindi{शदॢ}]\\
\begin{equation}
	\text{\texthindi{णिच्}}(x) = c(1^{\prime}) +v(1^{\prime})+ \text{\texthindi{ व् }}+ \text{\texthindi{इ}}
\end{equation}
\begin{table}[h!]
	\begin{center}
		\begin{tabular}{ |c|c|c| } 
			\hline
			$x$&	$c(1^{\prime}) +  v(1^{\prime})$&	\texthindi{णिच्($x$)}\\
			\hline
			\texthindi{ शदॢी}&	\texthindi{ शाा}&	\texthindi{ शाति}\\
			\hline
		\end{tabular}
		\caption{Examples of \texthindi{णिच्} function Case XI}
		\label{table:6.11}
	\end{center}
	
\end{table}


Thus, we looked at the cases of the णिच् function which gives the output in the form of a णिजन्त word. Now let us look into a kṛt pratyaya, i.e. a Tumun Pratyaya.