\chapter{Review of Literature}
\label{sec:Review of Literature}



\lettrine[findent=2pt]{\textbf{T}}{}he rules in one chapter may control rules in another. In this way, Pāṇini created a brief and immensely dense work (learnsanskrit.org, 2019). The rules have been broadly classified into six broad categories: saṃjñā, vidhi, niyama, atideśa, adhikāra and paribhāṣā rules.\\ 
The first category of specialized rules is saṃjñā, where essential words can have technical meanings which exist only within the scope of Aṣṭādhyāyī. One such example is the word 'vṛddhi’ which originally means "growth" or "gain", but within Aṣṭādhyāyī, the letters ā, e and au are called 'vṛddhi’. The second category is vidhi, which describes things such as word formation, and the application of sandhi. The third category of rules is the niyama, which contradicts the earlier vidhi rules. Essentially, it contains an exception to a previous rule. The fourth ones are the atideśa rules, which specify that some feature has the properties of another. The fifth type of rule is the adhikāra rules. This sort of rule establishes an idea that extends to the rules that follow it. Such a rule sometimes specifies how far it extends, but usually, its extension is clear from the context. The sixth type is paribhāṣā rules. \\

Aṣṭādhyāyī has been a subject of curiosity for linguistic researchers worldwide due to its highly compact but complete nature, and has become a model for the later specialist technical texts and sūtras (Jonardon Ganeri, 1909). It is quite evident that each syllable has been meticulously chosen so as to ensure that minimalistic writing style is followed throughout the text. One such example is the word ‘hal’, which is used to represent the term ‘all consonants’. 
Aṣṭādhyāyī takes input data from lexical lists, such as the Dhātupātha and the Ganapātha and describes rules that can be applied to them for the generation of words. The sequence in which these sūtras has been written is extremely important as sūtras oftentimes are based on the prior knowledge of the sūtras that precede it. This technique is referred to as ‘Anuvṛtti’ and it eliminates the need to write certain words again and again. In the example given below, the highlighted part in the second sūtra is Anuvṛtti, and it does not explicitly appear in the sūtra however its presence is implicit there.\\
\texthindi{वृद्धि: आत् ऐच्\\
	अत् एङ् गुणः वृद्धि:}
\\The effects produced by sūtras become part of an ever-evolving environment which may trigger other (Sohoni \& Kulkarni, 2018). Hence, for the most part Pāṇini’s Aṣṭādhyāyī is highly systematized with every single syllable carefully accounted for. It can be fairly said that it is set of non-randomly written sūtras and carefully chosen group of syllables within those sūtras. It is to be noted that this is an attempt to systemize the production of random data i.e. the words produced in a natural language (Sanskrit in this case). 
Expressing the natural language in the form of functions is a step towards formalism. It can act as a bridge between natural and formal languages. When we think about the word ‘function’ in respect of Mathematics, a function is a process that associates to each element of a set $X$ a unique element of a set $Y$. So basically, function is a process to which we feed various inputs, and to each input it gives out a single unique output. In the definition of function, $X$ and $Y$ are respectively called the domain and the codomain of the function $f$. And the notation that we use is $y = f (x)$. By using functions (mathematical) we can express various grammatical operations that are easy to comprehend for a person with basic knowledge of mathematics. We can understand this simply by taking the example of pratyaya. 
Let us assume that we have a pratyaya A. Let f be a function where A is added to each dhātu that is fed into f. The output y would be a prātipadika. Writing the grammatical operations of a natural language in the form of mathematical functions also helps in the development and implementation of a controlled natural language, which is easier for the non-native speakers to understand and it also eases the process of computer processing of a natural language.
So, there are two ways to start with the formalization of a natural language. Either start with the grammar or with the data set. The Sanskrit Grammar, especially the one written by Pāṇini is already quite structured. So, it is easy in case of the Sanskrit language to start with the grammar as we do not have to start from scratch. These functions then at a later stage can be combined to form a larger superfunction comprising of all the individual sub-functions. 
However, this method of writing functions for natural languages has its own limitations. Not all the words can be fitted in this mechanism. There are some exceptions, which have to be separately mentioned. Moreover, there might be some operations which we may not be able to effectively express as functions. An alternate mechanism has to be developed in order to capture such exceptions and ambiguities. 
We started by constructing functions for pratyayas. Similarly, one can proceed with sandhi, samās etc and progressively cover the sections which can be covered using the functions. This exercise can be done for other Indian as well as for foreign languages too.  
In Pāṇinian grammar, a sentence is derived from words which are derived in turn from the respective roots and suffixes. Pāṇini's grammar can thus be said to be based on the concept of compositionality of the sentence form is believed to correspond with the compositionality of the sentence meaning as well as the compositionality of the essential accent.
Let us take the example of a sentence S, which is made up of units called words denoted by w1, w2, w3 and so on. Such words or units are of two types:
Root, called prakṛti in Sanskrit (indicated by R) and Termination, called Pratyaya in Sanskrit (indicated by T). Root lies to the left and the Termination lies to the right.

Figure~\ref{fig:s} shows the sentence formation using the Roots and Terminations.
\begin{figure}[!h]
	\centering
	\includegraphics[width=1.0\textwidth]{Figures/SentenceFormation.png}
	\hspace{1mm}
	\caption{sentence formation} 
	\label{fig:s}
\end{figure}

In other words, a vākya i.e. a sentence is made up of many padas. The padas are made up of prakṛti and pratyaya. Once we add a pratyaya to a dhātu, it becomes prātipadika. 
Pāṇinian grammar’s exhaustive yet precise nature (Kulkarni A., Brevity in Pāṇini’s Aśṭādhyāyī, 2016, p. 1) makes it condusive to the formalization of Sanskrit language. In fact, Briggs (Briggs, 1985, pp. 32-39) even demonstrated in his article the salient features of Sanskrit language that can make it serve as an artificial language. Although, the attempts in formalization of natural laguages is not limited to the Sanskrit language. Various efforts in mathematical modelling of natural languages including that of the Indian languages have been made before. Joseph Kallrath in his book ‘Modeling Languages in Mathematical Optimization’ says that ‘a modeling language serves the need to pass data and a mathematical model description to a solver in the same way that people especially mathematicians describe those problems to each other’ (Kallrath, 2013, p. x). Mathematical modelling of languages also impacts our understanding of the language and its grammar. As scholars are delving into the question of formalizing various natural languages, it is also having an impact on how we understand the language itself. Recent work in theoretical and computational linguistics has influenced the interpretation of grammar (Scharf, 2008, p. 97). Statistical analysis of a language is a vital part of natural language processing (Goyal, 2011, p. 1). Though well-defined rules for Sanskrit morphology exist in Aṣṭādhyāyī, for a typical computational linguist without any knowledge of Sanskrit, it is difficult to build a system incorporating these rules (Kulkarni \& Shukl, 2009, p. 1). For the grammar to fit mathematical functions, we ‘need a strong and unambiguous grammar which is provided by Maharishi Pāṇini in the form of Aṣṭādhyāyī’ (Agrawal, 2013, p. 1135). We have followed a similar approach, wherein we have modelled the Pratyayas in Sanskrit in the form of functions. Similar to mathematical functions which can be expressed as y = f (x) or f (x) = y where the function of accepts x as an input giving out y as an output. The effects produced by sūtras become part of an ever-evolving environment which may trigger other’ (Sohoni \& Kulkarni, 2018, p. 18), such that theorectiacally they form a continuous loop where outputs can again be treated as inut for the next functions. 
According to how components of the target linguistic phenomenon are realized mathematically, available models of language evolution can be classified as rule-based and equation-based models. Equation-based models tend to transform linguistic and relevant behaviors into mathematical equations (Tao Gong, 2013, p. 2), which is what we have attempted in this paper. However, ambiguity is inherent in the Natural Language sentences (Tapaswi \& Jain, 2012, p. 1) and mathematical modelling of such natural languages helps to remove this ambiguity. Traditionally too, there have been attempts by various scholars like Kātyāyana, Patanjali and Bhartṛhari to provide extensive commentaries. Several attempts have been made to address the ambiguities and give clarifications. They do not question Pāṇini’s basic model, but rather explain it, refine it and complete it (Huet, 2003, pp. 307-325). Explanations and clarifications in the form of various vārtikas also come handy while dealing with such ambiguities. Another mathematical approach towards understanding Pāṇini is to verify the conciseness of his work, similar to what has been done by Peterson in her paper titled ‘A Mathematical Analysis of Pāṇini's Śivasūtras’ (Petersen, 2004, pp. 471-489), where she has proved the minimal approach of Pāṇini while writing the śivasūtras.
Our aim is to construct mathematical functions for various operations in Pāṇinian grammer to make it comprehensible for a person who does not belong to the field of Sanskrit and Linguistics and has a basic understanding of mathematical functions.
