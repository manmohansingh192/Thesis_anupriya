\chapter{Examples of some cases by Pāṇini’s sūtras}

\section{Examples of the texthindi{णिच्} function cases by Pāṇini’s sūtras}

%\begin{table}[h!]
%	\begin{center}
		\begin{longtable}{ |c| } 
			\hline
			\rowcolor{red!10}
			Final form of \texthindi{धातु} \\
			\hline
			\rowcolor{green!10}
			Additional rules \\
			\hline
			\rowcolor{blue!10}
			Application of pratyaya\\
			\hline
			\rowcolor{yellow!10}
			The final output\\
			\hline
		\caption{Scheme of sūtras}
		\label{table:a1}
		\end{longtable}


%	\end{center}
%\end{table}

%\begin{table}[h!]
%	\begin{center}
		\begin{longtable}{ |p{1.6cm}|p{14.4cm}| } 
			\hline
			\rowcolor{red!10}
			&\texthindi{हेतुमति च (3.1.26)}\\
			\rowcolor{red!10} \multirow{-1.5}{*}{ \texthindi{स्पर्ध्}}
			&The affix \texthindi{णिच्} is used after a root, when the operation of a causer is to be expressed.\\
			\hline
			\rowcolor{blue!10}
			\texthindi{स्पर्ध्+णिच् }&
			\texthindi{हलन्त्यम् }(1.3.3)\\
			\rowcolor{blue!10}
			&The ending \texthindi{हल्} letter of an \texthindi{उपदेश} is called '\texthindi{इत्}'.
			\\
			\hline
			\rowcolor{blue!10}
			\texthindi{स्पर्ध्+इ }&
			\texthindi{तस्य लोपः }(1.3.9)\\
			\rowcolor{blue!10}
			&Removal of ‘\texthindi{इत्}’
			\\
			\hline
			\rowcolor{yellow!10}
			\texthindi{स्पर्धि}&
			\texthindi{सनाद्यन्ता धातवः }(3.1.32)\\
			\rowcolor{yellow!10}
			&All the roots ending with the affixes \texthindi{सन्} and others are called \texthindi{धातु} ।
			\\
			\hline
		\caption{Sūtras for \texthindi{स्पर्ध्$+$णिच्} function}
		\label{table:a2}
		\end{longtable}

%	\end{center}
%\end{table}

%\begin{table}[h!]
%	\begin{center}
		\begin{longtable}{ |p{1.6cm}|p{14.4cm}| } 
			\hline
			\rowcolor{red!10}
			\texthindi{दध्}
			&	
			\texthindi{हेतुमति च} (3.1.26) \\
			\rowcolor{red!10}
			&The affix \texthindi{णिच्} is used after a root, when the operation of a causer is to be expressed.
			\\\hline
			\rowcolor{green!10}
			\texthindi{दध्}+\texthindi{णिच्}
			&\texthindi{अत उपधायाः} (7.2.116) \\
			\rowcolor{green!10}
			&In a \texthindi{अङ्ग} (stem) ending in a consonant with an अ immediately preceding it, the \texthindi{वृद्धि} is substituted for such \texthindi{अ} when followed by an indicatory \texthindi{ञ्} or \texthindi{ण्}।
			\\\hline
			\rowcolor{blue!10}
			\texthindi{दाध्}+\texthindi{णिच्}
			&
			\texthindi{हलन्त्यम्} (1.3.3)\\
			\rowcolor{blue!10}
			&The ending \texthindi{हल्} letter of an \texthindi{उपदेश} is called '\texthindi{इत्}'.
			\\\hline
			\rowcolor{blue!10}
			\texthindi{दाध्}+\texthindi{इ}
			&
			\texthindi{तस्य लोपः} (1.3.9)\\
			\rowcolor{blue!10}
			&Removal of ‘\texthindi{इत्}’
			\\\hline
			\rowcolor{yellow!10}
			\texthindi{दाधि}
			&\texthindi{सनाद्यन्ता धातवः} (3.1.32)\\
			\rowcolor{yellow!10}
			&All the roots ending with the affixes \texthindi{सन्} and others are called \texthindi{धातु} ।
			\\\hline

		\caption{Sūtras for \texthindi{दध्$+$णिच्} function}
		\label{table:a3}
		\end{longtable}
%	\end{center}
%\end{table}


%\begin{table}[h!]
%	\begin{center}
		\begin{longtable}{ |p{1.6cm}|p{14.4cm}| } 
			\hline
			
			\rowcolor{red!10}
			\texthindi{पिट्}
			&	
			\texthindi{हेतुमति च} (3.1.26) \\
			\rowcolor{red!10}
			&The affix \texthindi{णिच्} is used after a root, when the operation of a causer is to be expressed.
			\\\hline
			\rowcolor{blue!10}
			\texthindi{पिट्}+\texthindi{णिच्}
			&\texthindi{हलन्त्यम्} (1.3.3)\\
			\rowcolor{blue!10}
			&The ending \texthindi{हल्} letter of an \texthindi{उपदेश} is called '\texthindi{इत्}'.
			\\
			\hline
			\rowcolor{blue!10}
			\texthindi{पिट्}+\texthindi{णिच्}
			&\texthindi{तस्य लोपः} (1.3.9)\\
			\rowcolor{blue!10}
			&Removal of ‘\texthindi{इत्}’
			\\
			\hline
			\rowcolor{green!10}
			\texthindi{पिट्}+\texthindi{इ}
			&\texthindi{पुगन्तलघूपधस्य च} (7.3.86) \\
			\rowcolor{green!10}
			&When followed by a \texthindi{सार्वधातुक} or an \texthindi{आर्धधातुक प्रत्यय}, a \texthindi{पुगन्त अङ्ग} and a \texthindi{लघूपध अङ्ग} get a \texthindi{गुणादेश}.
			\\\hline
			\rowcolor{yellow!10}
			\texthindi{पेटि}
			&\texthindi{सनाद्यन्ता धातवः} (3.1.32) \\
			\rowcolor{yellow!10}
			&All the roots ending with the affixes \texthindi{सन्} and others are called \texthindi{धातु} । 
			\\\hline
		
		\caption{Sūtras for \texthindi{पिट् +णिच्} function}
		\label{table:a4}
		\end{longtable}
%	\end{center}
%\end{table}


%\begin{table}[h!]
%	\begin{center}
		\begin{longtable}{ |p{1.6cm}|p{14.4cm}| } 
			\hline
			\rowcolor{red!10}
			\texthindi{नी}
			&\texthindi{हेतुमति च} (3.1.26) \\
			\rowcolor{red!10}
			&The affix \texthindi{णिच्} is used after a root, when the operation of a causer is to be expressed.
			\\\hline
			\rowcolor{green!10}
			\texthindi{नी}+\texthindi{णिच्}
			&\texthindi{अचो ञ्णिति} (7.2.115)\\
			\rowcolor{green!10}
			&Before the affixes having an indicatory \texthindi{ञ्} or \texthindi{ण् , वृद्धि} is substituted for the end-vowel of a \texthindi{अङ्ग} (stem).
			\\\hline
			\rowcolor{blue!10}
			\texthindi{नै}+\texthindi{णिच्}
			&\texthindi{हलन्त्यम्} (1.3.3)\\
			\rowcolor{blue!10}
			&The ending \texthindi{हल्} letter of an \texthindi{उपदेश} is called '\texthindi{इत्}'.
			\\\hline
			\rowcolor{blue!10}
			\texthindi{नै}+\texthindi{णिच्}
			&\texthindi{तस्य लोपः} (1.3.9)\\
			\rowcolor{blue!10}
			&Removal of ‘\texthindi{इत्}’
			\\\hline
			\rowcolor{green!10}
			\texthindi{नायि}
			&\texthindi{एचोऽयवायावः} (6.1.78)\\
			\rowcolor{green!10}
			&The vowels \texthindi{ए ऐ ओ} and \texthindi{औ} are respectively substituted \texthindi{अय् आय् अव्} and \texthindi{आव्} when a vowel follows.
			\\\hline
			\rowcolor{yellow!10}
			
			\texthindi{नायि}
			&\texthindi{सनाद्यन्ता धातवः} (3.1.32) \\
			\rowcolor{yellow!10}
			&All the roots ending with the affixes \texthindi{सन्} and others are called \texthindi{धातु} । 
			\\
		\hline
		\caption{Sūtras for \texthindi{नी$+$णिच्} function}
		\label{table:a5}
		\end{longtable}
%	\end{center}
%\end{table}


%\begin{table}[h!]
%	\begin{center}
		\begin{longtable}{ |p{1.6cm}|p{14.4cm}| } 
			\hline
			\rowcolor{red!10}
			
			\texthindi{का}+\texthindi{णिच्}
			&\texthindi{हेतुमति च} (3.1.26) \\
			\rowcolor{red!10}
			&The affix \texthindi{णिच्} is used after a root, when the operation of a causer is to be expressed.
			\\\hline
			\rowcolor{green!10}
			\texthindi{काप्}+\texthindi{णिच्}
			&\texthindi{अर्त्तिह्रीब्लीरीक्नूयीक्ष्माय्यातां पुङ्णौ} (7.3.36)\\
			\rowcolor{green!10}
			&The augment \texthindi{पुक् (प्)} is added to the roots 1. \texthindi{ऋ} 2. \texthindi{ह्री} 3. \texthindi{व्ली} 4. \texthindi{री} 5. \texthindi{क्नुय्} 6. \texthindi{क्ष्माय} and to a root ending in long \texthindi{आ} when the affix \texthindi{णि} (the Causative) follows.
			\\\hline
			\rowcolor{blue!10}
			\texthindi{काप्}+\texthindi{णिच्}
			&\texthindi{हलन्त्यम्} (1.3.3)\\
			\rowcolor{blue!10}
			&The ending \texthindi{हल्} letter of an \texthindi{उपदेश} is called '\texthindi{इत्}'.
			\\\hline
			\rowcolor{blue!10}
			\texthindi{काप्}+\texthindi{इ}
			&\texthindi{तस्य लोपः} (1.3.9)\\
			\rowcolor{blue!10}
			&Removal of ‘\texthindi{इत्}’
			\\\hline
			\rowcolor{yellow!10}
			\texthindi{कापि}
			&\texthindi{सनाद्यन्ता धातवः} (3.1.32) \\
			\rowcolor{yellow!10}
			&All the roots ending with the affixes \texthindi{सन्} and others are called \texthindi{धातु} । 
			\\\hline
		
		\caption{Sūtras for \texthindi{का$+$णिच्} function}
		\label{table:a6}
		\end{longtable}
%	\end{center}
%\end{table}

%\begin{table}[h!]
%	\begin{center}
		\begin{longtable}{ | p{1.6cm}|p{14.4cm}| } 
			\hline
			\rowcolor{red!10}
			\texthindi{रच}+\texthindi{णिच्}
			&\texthindi{हेतुमति च} (3.1.26) \\
			\rowcolor{red!10}
			&The affix \texthindi{णिच्} is used after a root, when the operation of a causer is to be expressed.
			\\\hline
			\rowcolor{green!10}
			\texthindi{रच्}+\texthindi{णिच्}
			&\texthindi{अतो लोपः} (6.4.48)\\
			\rowcolor{green!10}
			&The \texthindi{अ} standing at the end of a \texthindi{अङ्ग} (stem) is removed if it comes before an \texthindi{आर्धधातुक} affix.
			\\\hline
			\rowcolor{blue!10}
			\texthindi{रच्}+\texthindi{णिच्}
			&\texthindi{हलन्त्यम्} (1.3.3)\\
			\rowcolor{blue!10}
			&The ending \texthindi{हल्} letter of an \texthindi{उपदेश} is called '\texthindi{इत्}'.
			\\\hline
			\rowcolor{blue!10}
			\texthindi{रच्}+\texthindi{इ}
			&\texthindi{तस्य लोपः} (1.3.9)\\
			\rowcolor{blue!10}
			&Removal of ‘\texthindi{इत्}’
			\\\hline
			\rowcolor{yellow!10}
			\texthindi{रचि}
			&\texthindi{सनाद्यन्ता धातवः} (3.1.32) \\
			\rowcolor{yellow!10}
			&All the roots ending with the affixes \texthindi{सन्} and others are called \texthindi{धातु} । 
			\\\hline
		
		\caption{Sūtras for \texthindi{रच्$+$णिच्} function}
		\label{table:a7}
		\end{longtable}
%	\end{center}
%\end{table}


\section{Examples of the \texthindi{तुमुन्} function cases by Pāṇini’s sūtras}

%\begin{table}%[h!]
%	\begin{center}
		\begin{longtable}{ |p{16cm}| } 
			\hline
			\rowcolor{red!10}
			Final form of \texthindi{धातु}\\\hline
			\rowcolor{blue!10}
			Application of pratyaya\\\hline
			\rowcolor{green!10}
			Addition or no addition of \texthindi{इ} based on the categorisation of \texthindi{धातु} on the basis of seṭ/aniṭ\\\hline
			\rowcolor{yellow!10}
			Final form of the  prātipadika \\\hline
		
		\caption{Structure of Sūtras for \texthindi{तुमुन्} function}
		\label{table:a8}
		\end{longtable}
%	\end{center}
%\end{table}


%\begin{table}%[h!]
%	\begin{center}
		\begin{longtable}{ |p{2cm}|p{14cm}| } 
			\hline
			\rowcolor{red!10}
			 &\texthindi{भूवादयो धातवः} (1.3.1)\\
			\rowcolor{red!10} \multirow{1.5}{*}{\texthindi{कथ्}}
			&The words beginning with \texthindi{भू} 'to become' and denoting action, are called \texthindi{धातु} or verbal roots
			\\\hline
			\rowcolor{red!10}
			&\texthindi{सत्यापपाशरूपवीणातूलश्लोकसेनालोमत्वचवर्मवर्ण चूर्णचुरादिभ्यो} \texthindi{णिच्} (3.1.25)\\
			\rowcolor{red!10}
			\multirow{-1.5}{*}{\texthindi{कथ् णिच्}} &
			The affix \texthindi{णिच्} is used after these words - \texthindi{सत्याप, पाश, रूप, वीणा, तूल, श्लोक, सेना, लोम, त्वच, वर्म, वर्ण, चूर्ण} and \texthindi{चुरादि} dhātus. 
			\\\hline
			\rowcolor{red!10}
			&\texthindi{अनुबन्ध-लोपः}
			\\
			\rowcolor{red!10}
			&\texthindi{सनाद्यन्ता धातवः} (3.1.32) \\
			\rowcolor{red!10}
			\multirow{-1.5}{*}{\texthindi{कथि}}
			&All the roots ending with the affixes \texthindi{सन्} and others are called \texthindi{धातु} ।
			\\
			\hline
			\rowcolor{blue!10}
			&\texthindi{समानकर्तृकेषु तुमुन्} (3.3.158)\\ 
			\rowcolor{blue!10}
			\multirow{-1.5}{*}{\texthindi{कथि तुमुन्}}
			&The affix \texthindi{तुमुन्} comes after a verb when another verb meaning 'to wish' is in construction, provided that the agent of both the verbs is the same.
			\\\hline
			\rowcolor{blue!10}
			&\texthindi{हलन्त्यम्} (1.3.3)\\
			\rowcolor{blue!10}
			&In \texthindi{उपदेश}, the final consonant of roots etc. is \texthindi{इत्}.
			\\\hline
			\rowcolor{blue!10}
			&\texthindi{उपदेशे अच् अनुनासिक} (1.3.2)\\
			\rowcolor{blue!10}
			&An \texthindi{अनुनासिक स्वर} present in an \texthindi{उपदेश} is called '\texthindi{इत्}'. 
			\\\hline
			\rowcolor{blue!10}
			&\texthindi{तस्य लोपः} (1.3.9)\\
			\rowcolor{blue!10}
			\multirow{-3}{*}{\texthindi{कथि तुम्}}
			&The \texthindi{इत्} letter is elided during the \texthindi{प्रक्रिया}.
			\\\hline
			\rowcolor{green!10}
			&\texthindi{आर्धधातुकस्य इट् वलादेः} (7.2.35)\\
			\rowcolor{green!10}
			&An \texthindi{आर्धधातुक-प्रत्यय} that starts with a \texthindi{वल्} letter gets an \texthindi{इट् आगम}. \\
			\rowcolor{green!10}
			\multirow{-1}{*}{\texthindi{कथि इट् तुम्}}
			& \texthindi{वल्} - all consonants except \texthindi{य्}
			\\\hline
			\rowcolor{green!10}
			\texthindi{कथि} \texthindi{इ} \texthindi{तुम्}
			&\texthindi{अनुबन्ध-लोपः}
			\\\hline
			\rowcolor{yellow!10}
			\texthindi{कथे} \texthindi{इ} \texthindi{तुम्}&
			\texthindi{सार्वधातुकार्धधातुकयोः} (7.3.84)\\
			\rowcolor{yellow!10}
			&When followed by an \texthindi{सार्वधातुक} or an \texthindi{आर्धधातुक प्रत्यय}, an \texthindi{इगन्त अङ्ग} gets a \texthindi{गुणादेश}.
			\texthindi{अदेङ् गुणः} (1.1.2) \texthindi{अत्} = \texthindi{अ}, \texthindi{एङ्} = \texthindi{ए}, \texthindi{ओ} have \texthindi{गुण संज्ञा})
			\\\hline
			\rowcolor{yellow!10}
			&\texthindi{स्थाने अन्तरतमः} (1.1.50)\\
			\rowcolor{yellow!10}
			&When a common term is obtained as a substitute, the likest of its significates to that in the place of which it comes, is the actual substitute.
			\\\hline
			\rowcolor{yellow!10}
			\texthindi{कथय्} \texthindi{इ} \texthindi{तुम्}&
			\texthindi{एचोऽयवायावः} (6.1.75)\\
			
			\rowcolor{yellow!10}
			&For the vowels \texthindi{ए ऐ ओ} and \texthindi{औ} are respectively substituted \texthindi{अय् आय् अव्} and \texthindi{आव्} when a vowel follows.
			\\\hline
			\rowcolor{yellow!10}
			\texthindi{कथयितुम्}&	 \\\hline
		
		\caption{Sūtras for \texthindi{कथ्$+$तुमुन्} function}
		\label{table:a9}
		\end{longtable}
%	\end{center}
%\end{table}

%\begin{table}[h!]
%	\begin{center}
		\begin{longtable}{ |p{1.6cm}|p{14.4cm}| } 
			\hline
			\rowcolor{red!10}
			\texthindi{श्रि}&
			\texthindi{भूवादयो धातवः} (1.3.1)\\ 
			\rowcolor{red!10}&The words beginning with \texthindi{भू} ‘to become’ and denoting action, are called \texthindi{धातु} or verbal roots
			\\\hline
			
			\rowcolor{blue!10}
			\texthindi{श्रि तुमुन्}&
			\texthindi{समानकर्तृकेषु तुमुन्} (3.3.158)\\
			\rowcolor{blue!10}
			&The affix \texthindi{तुमुन्} comes after a verb when another verb meaning 'to wish' is in construction, provided that the agent of both the verbs is the same.
			\\\hline
			
			\rowcolor{blue!10}
			\texthindi{श्रि तुम्}&
			\texthindi{हलन्त्यम्} (1.3.3)\\
			\rowcolor{blue!10}
			&In \texthindi{उपदेश}, the final consonant of roots etc. is \texthindi{इत्}.\\
			\hline
			
			\rowcolor{blue!10}
			&\texthindi{उपदेशे अच् अनुनासिक} (1.3.2)\\
			\rowcolor{blue!10}
			&An \texthindi{अनुनासिक स्वर} present in an \texthindi{उपदेश} is called '\texthindi{इत्}'.\\
			\hline
			\rowcolor{blue!10}
			&\texthindi{तस्य लोपः} (1.3.9)\\
			\rowcolor{blue!10}
			&The \texthindi{इत्} letter is elided during the \texthindi{प्रक्रिया}.\\
			\hline
			
			\rowcolor{green!10}
			\texthindi{श्रि इट् तुम्}&
			\texthindi{आर्धधातुकस्य इट् वलादेः} (7.2.35)\\
			\rowcolor{green!10}
			&An \texthindi{आर्धधातुक-प्रत्यय} that starts with a \texthindi{वल्} letter gets an \texthindi{इट् आगम}. 
			\texthindi{वल्} - all consonants except \texthindi{य्}
			\\\hline
			
			\rowcolor{green!10}
			\texthindi{श्रि इतुम्}&
			\texthindi{अनुबन्ध-लोपः}
			\\\hline
			
			\rowcolor{yellow!10}
			\texthindi{श्रे इतुम्}&
			\texthindi{सार्वधातुकार्धधातुकयोः} (7.3.84)\\
			\rowcolor{yellow!10}
			&When followed by an \texthindi{सार्वधातुक} or an \texthindi{आर्धधातुक प्रत्यय}, an \texthindi{इगन्त अङ्ग} gets a \texthindi{गुणादेश}.\\
			\rowcolor{yellow!10}
			&\texthindi{अदेङ् गुणः} (1.1.2)\\
			\rowcolor{yellow!10}
			& \texthindi{अत्} = \texthindi{अ}, \texthindi{एङ्} = \texthindi{ए, ओ} have \texthindi{गुण संज्ञा})
			\\\hline
			
			\rowcolor{yellow!10}
			\texthindi{श्रय् इतुम्}&
			\texthindi{एचोऽयवायावः} (6.1.75)\\
			\rowcolor{yellow!10}
			&For the vowels \texthindi{ए ऐ ओ} and \texthindi{औ} are respectively substituted \texthindi{अय् आय् अव्} and \texthindi{आव्} when a vowel follows.
			\\\hline
			
			\rowcolor{yellow!10}
			\texthindi{श्रयितुम्}&	\\\hline 
		
		\caption{Sūtras for \texthindi{श्रि$+$तुमुन्} function}
		\label{table:a10}
		\end{longtable}
%	\end{center}
%\end{table}



\section{Examples of the \texorpdfstring{\texthindi{यङ्}} function cases by Pāṇini’s sūtras}





%\begin{table}[h!]
%	\begin{center}
		\begin{longtable}{ |c| } 
			\hline
			\rowcolor{red!10}
			Application of pratyaya
			\\\hline
			\rowcolor{blue!10}
			Intermediary form
			\\\hline
			\rowcolor{green!10}
			Reduplication and additional rules
			\\\hline
			\rowcolor{yellow!10}
			Final form of the  prātipadika 
			\\\hline
		
		\caption{Scheme of Sūtras for \texthindi{यङ्} function}
		\label{table:a11}
		\end{longtable}
%	\end{center}
%\end{table}


%\begin{table}[ht!]
%	\begin{center}
		\begin{longtable}{ |p{1.5cm}|p{14.5cm}| } 
			\hline
			\rowcolor{red!10} 
			\texthindi{भू}+\texthindi{यङ्}
			&\texthindi{धातोरेकाचो हलादेः क्रियासमभिहारे यङ्} (3.1.22)
			
			The affix \texthindi{वङ्} , in the sense of repitition of the act, comes after a root, having a single vowel and beginning with a consonant.
			\\\hline
			\rowcolor{blue!10}
			\texthindi{भू}+\texthindi{यङ्}
			&\texthindi{आर्द्धधातुकं शेषः}  (3.4.114)
			
			Except for \texthindi{तिङ्} and \texthindi{शित् प्रत्यय}s, other \texthindi{प्रत्यय}s that are attached to verbs are called '\texthindi{आर्धधातुक}'.
			\\\hline
			\rowcolor{blue!10}
			\texthindi{भू}+\texthindi{यङ्}
			&\texthindi{हलन्त्यम्} (1.3.3)
			
			In \texthindi{उपदेश}, the final consonant of roots etc. is \texthindi{इत्}.
			\\\hline
			\rowcolor{blue!10}
			\texthindi{भू}+\texthindi{य}
			&\texthindi{तस्य लोपः} (1.3.9)
			
			The \texthindi{इत्} letter is elided during the \texthindi{प्रक्रिया}.
			\\\hline
			\rowcolor{green!10} 
			\texthindi{भू}+\texthindi{य}
			&\texthindi{अकृत्सार्वधातुकयोर्दीर्घः} (7.4.25)
			
			A long is substituted for the final vowel of the \texthindi{अङ्ग} (stem) before an affix beginning with \texthindi{य्} having an indicatory \texthindi{क्} or \texthindi{ङ्} , when it is not a \texthindi{कृत्} or a \texthindi{सार्वधातुक} affix.
			\\\hline
			\rowcolor{green!10} 
			\texthindi{भू}+\texthindi{भू}+\texthindi{य}
			&\texthindi{सन्यङोः} (6.1.9)
			
			When followed by the \texthindi{सन्-प्रत्यय} or the \texthindi{यङ्-प्रत्यय}, a verb root that have not yet undergone \texthindi{द्वित्वम्} undergoes \texthindi{द्वित्वम्}.
			\\\hline
			
			\rowcolor{green!10}
			\texthindi{भू}+\texthindi{भू}+\texthindi{य}
			&\texthindi{हलादिः शेषः} (7.4.60)
			
			Only the first \texthindi{हल्} letter from \texthindi{अभ्यास} is retained, rest all \texthindi{हल्} letters are removed.
			\\\hline
			\rowcolor{green!10} 
			\texthindi{भु}+\texthindi{भू}+\texthindi{य}
			&\texthindi{ह्रस्वः} (7.4.59) 
			
			\texthindi{अभ्यास} is converted to \texthindi{ह्रस्व}.
			\\\hline
			\rowcolor{green!10} 
			\texthindi{बु}+\texthindi{भू}+\texthindi{य}
			&\texthindi{अभ्यासे चर्च्च} (8.4.54)  \\
			\rowcolor{green!10}
			&The \texthindi{चर्} is also substitute of \texthindi{झल्} letters occuring in a reduplicate syllable, as well as \texthindi{जश्}।
			
			\texthindi{पूर्वोSभ्यास} (6.1.4)\\
			\rowcolor{green!10}
			&When a repetion is done, the first entity out of the two is called \texthindi{अभ्यास}.
			\\\hline
			\rowcolor{green!10} 
			\texthindi{बो}+\texthindi{भू}+\texthindi{य}
			&\texthindi{गुणो यङ्लुकोः} (7.4.82)\\
			\rowcolor{green!10}
			&\texthindi{गुण} is substituted for the final of \texthindi{इ} and \texthindi{उ} (with their long) of a reduplicate when the Intensive character \texthindi{यङ्} follows and also when it is elided.
			\\\hline
			\rowcolor{yellow!10} 
			\texthindi{बोभूय}
			&\texthindi{सनाद्यन्ता धातवः} (3.1.32)
			
			All the roots ending with the affixes \texthindi{सन्} and others are called \texthindi{धातु} ।
			\\\hline
		
		\caption{Sūtras for \texthindi{भू$+$यङ्} function}
		\label{table:a12}
		\end{longtable}
%	\end{center}
%\end{table}

%\begin{table}[h!]
%	\begin{center}
		\begin{longtable}{ |p{1.8cm}|p{14.2cm}| } 
			\hline
			\rowcolor{red!10}
			\texthindi{चर्}+\texthindi{यङ्}
			&\texthindi{लुपसदचरजपजभदहदशगॄभ्यो भावगर्हायाम्} (3.1.24)
			
			The affix \texthindi{वङ्} when it is intended to convey the sense of contempt (\texthindi{गर्हा}) in respect of the sense of the root (\texthindi{भाव}) comes always after the following roots - \texthindi{लुप्} 'to cut off', \texthindi{सद्} 'to sit', \texthindi{चर्} 'to walk', \texthindi{जप्} 'to mutter silently any sacred formula', \texthindi{जभ} 'to gape or yawn', \texthindi{दह्} 'to burn', \texthindi{दश्} 'to bite' and \texthindi{गृ} 'to swallow'. 
			\\\hline
			\rowcolor{blue!10}
			\texthindi{चर्}+\texthindi{यङ्}
			&\texthindi{आर्द्धधातुकं शेषः}  (3.4.114)
			
			Except for \texthindi{तिङ्} and \texthindi{शित् प्रत्यय}s, other \texthindi{प्रत्यय}s that are attached to verbs are called '\texthindi{आर्धधातुक}'. 
			\\\hline
			\rowcolor{blue!10}
			\texthindi{चर्}+\texthindi{यङ्}
			&\texthindi{हलन्त्यम्} (1.3.3)
			
			In \texthindi{उपदेश}, the final consonant of roots etc. is \texthindi{इत्}.
			\\\hline
			\rowcolor{blue!10}
			\texthindi{चर्}+\texthindi{य}
			&\texthindi{तस्य लोपः} (1.3.9)
			
			The \texthindi{इत्} letter is elided during the \texthindi{प्रक्रिया}.
			\\\hline
			\rowcolor{green!10}
			\texthindi{चर्}+\texthindi{चर्}+\texthindi{य}
			&\texthindi{सन्यङोः} (6.1.9)
			
			When followed by the \texthindi{सन्-प्रत्यय} or the \texthindi{यङ्-प्रत्यय}, a verb root that have not yet undergone \texthindi{द्वित्वम्} undergoes \texthindi{द्वित्वम्}.
			\\\hline
			\rowcolor{green!10}
			\texthindi{च}+\texthindi{चर्}+\texthindi{य}
			&\texthindi{हलादिः शेषः} (7.4.60)
			
			Only the first \texthindi{हल्} letter from अभ्यास is retained, rest all \texthindi{हल्} letters are removed. 
			\\\hline
			\rowcolor{green!10}
			\texthindi{चन्चर्}+\texthindi{य}
			&\texthindi{चरफलोश्च} (7.4.87)
			
			The augment \texthindi{नुक् (न्)} comes after the reduplicates of \texthindi{चर्} and \texthindi{फल्} in the Intensive (with expressed or elided \texthindi{यङ्}).
			\\\hline
			\rowcolor{green!10}
			\texthindi{चन्चुर्}+\texthindi{य}
			&\texthindi{उत्‌ परस्यातः} (7.4.88) 
			
			For the subsequent \texthindi{अ} (i.e. for the \texthindi{अ} of the root and not of the reduplicate), there is substituted \texthindi{उ} in the Intensive (with expressed or elided \texthindi{यङ्}).
			\\\hline
			\rowcolor{yellow!10}
			\texthindi{चन्चुर्य}
			&\texthindi{सनाद्यन्ता धातवः} (3.1.32)
			
			All the roots ending with the affixes \texthindi{सन्} and others are called \texthindi{धातु} ।
			\\\hline
		
		\caption{Sūtras for \texthindi{चर्$+$यङ्} function}
		\label{table:a13}
		\end{longtable}
%	\end{center}
%\end{table}

%\begin{table}[h!]
%	\begin{center}
		\begin{longtable}{ |p{1.4cm}|p{14.6cm}| } 
			\hline
			\rowcolor{red!10}
			\texthindi{अट्}+\texthindi{यङ्}
			&\texthindi{आर्द्धधातुकं शेषः}  (3.4.114)
			
			Except for \texthindi{तिङ्} and \texthindi{शित् प्रत्यय}s, other \texthindi{प्रत्यय}s that are attached to verbs are called '\texthindi{आर्धधातुक}'. 
			\\\hline
			\rowcolor{blue!10}
			\texthindi{अट्}+\texthindi{यङ्}
			&\texthindi{हलन्त्यम्} (1.3.3)
			
			In \texthindi{उपदेश}, the final consonant of roots etc. is \texthindi{इत्}.
			\\\hline
			\rowcolor{blue!10}
			\texthindi{अट्}+\texthindi{य}
			&\texthindi{तस्य लोपः} (1.3.9)
			
			The \texthindi{इत्} letter is elided during the \texthindi{प्रक्रिया}.
			\\\hline
			\rowcolor{green!10}
			\texthindi{अट्य}+\texthindi{ट्य}
			&\texthindi{सन्यङोः} (6.1.9)
			
			When followed by the \texthindi{सन्-प्रत्यय} or the \texthindi{यङ्-प्रत्यय}, a verb root that have not yet undergone \texthindi{द्वित्वम्} undergoes \texthindi{द्वित्वम्}.
			\\\hline
			\rowcolor{green!10}
			\texthindi{अट}+\texthindi{ट्य}
			&\texthindi{हलादिः शेषः} (7.4.60)
			
			Only the first \texthindi{हल्} letter from \texthindi{अभ्यास} is retained, rest all \texthindi{हल्} letters are removed. 
			\\\hline
			\rowcolor{green!10}
			\texthindi{अटा}+\texthindi{ट्य}
			&\texthindi{दीर्घोऽकितः} (7.4.83)
			
			A long vowel is substituted for the \texthindi{अ} of the reduplicate in the Intensive (with expressed or elided \texthindi{यङ्}) when the reduplicate receives no augment having an indicatory \texthindi{क्} ।
			\\\hline
			\rowcolor{yellow!10}
			\texthindi{अटाट्य}
			&\texthindi{सनाद्यन्ता धातवः} (3.1.32)
			
			All the roots ending with the affixes \texthindi{सन्} and others are called \texthindi{धातु} ।
			\\\hline
		
		\caption{Sūtras for \texthindi{अट्$+$यङ्} function}
		\label{table:a14}
		\end{longtable}
%	\end{center}
%\end{table}

\section{Examples of the \texthindi{सन्} function cases by Pāṇini’s sūtras}

%\begin{table}[h!]
%	\begin{center}
		\begin{longtable}{ |c| } 
			\hline
			\rowcolor{blue!10}
			Application of pratyaya
			\\\hline
			\rowcolor{red!10}
			Intermediary form
			\\\hline
			\rowcolor{yellow!10}
			Reduplication and additional rules
			\\\hline
			\rowcolor{green!10}
			Final form of the  prātipadika 
			\\\hline
		
		\caption{Scheme of sūtras}
		\label{table:a15}
		\end{longtable}
%	\end{center}
%\end{table}

%\begin{table}[h!]
%	\begin{center}
		\begin{longtable}{ |p{1.4cm}|p{14.6cm}| } 
			\hline
			\rowcolor{blue!10}
			\texthindi{अस् सन्}
			&\texthindi{धातोः कर्मणः समानकर्तृकादिच्छायां वा} (3.1.7)\\
			\rowcolor{blue!10}
			&The affix is optionally attached, in the sense of 'wishing' after a root expressing the object wished and having the same agent of the action as the wisher thereof. 
			\\\hline
			
			\rowcolor{blue!10}
			\texthindi{अस् सन्}
			&\texthindi{आर्द्धधातुकं शेषः}  (3.4.114)\\
			\rowcolor{blue!10}
			&Except for \texthindi{तिङ्} and \texthindi{शित् प्रत्यय}s, other \texthindi{प्रत्यय}s that are attached to verbs are called '\texthindi{आर्धधातुक}'. 
			\\\hline
			\rowcolor{red!10}
			\texthindi{अस् सन्}
			&\texthindi{हलन्त्यम्} (1.3.3)\\
			\rowcolor{red!10}
			&In \texthindi{उपदेश}, the final consonant of roots etc. is \texthindi{इत्}.
			\\\hline
			\rowcolor{red!10}
			
			\texthindi{अस् स}
			&\texthindi{तस्य लोपः} (1.3.9)\\
			\rowcolor{red!10}
			&The \texthindi{इत्} letter is elided during the \texthindi{प्रक्रिया}.
			\\\hline
			\rowcolor{yellow!10}
			\texthindi{अस् इस}
			&\\
			\rowcolor{yellow!10}\texthindi{असिस}
			&\texthindi{आर्धधातुकस्य इट् वलादेः} (7.2.35)\\
			\rowcolor{yellow!10}
			&An \texthindi{आर्धधातुक-प्रत्यय} that starts with a \texthindi{वल्} letter gets an \texthindi{इट् आगम}.\\ 
			\rowcolor{yellow!10}
			&\texthindi{वल्} - all consonants except \texthindi{य्}
			\\\hline
			\rowcolor{green!10}
			\texthindi{असिसिस}
			&\texthindi{सन्यङोः} (6.1.9)\\
			\rowcolor{green!10}
			&When followed by the \texthindi{सन्-प्रत्यय} or the \texthindi{यङ्-प्रत्यय}, a verb root that have not yet undergone \texthindi{द्वित्वम्} undergoes \texthindi{द्वित्वम्}.
			\\\hline
			\rowcolor{green!10}
			&\texthindi{अजादेर्द्वितीयस्य} (6.1.2)\\
			\rowcolor{green!10}
			&Of that whose first syllable begins with a vowel, there are two in the room of the second portion containing a single vowel. 
			\\\hline
			\rowcolor{green!10}
			\texthindi{असि षिष}
			&\texthindi{आदेशप्रत्यययोः} (8.3.59)\\
			\rowcolor{green!10}
			&An \texthindi{अपदान्त सकार} that was obtained either as an \texthindi{आदेश} or through a \texthindi{प्रत्यय} is converted to \texthindi{षकार} when it follows a letter from \texthindi{इण्-प्रत्याहार} or a \texthindi{कवर्ग}.
			\\\hline
			\rowcolor{green!10}
			\texthindi{असिषिष}
			&\texthindi{सनाद्यन्ता धातवः} (3.1.32)\\
			\rowcolor{green!10}
			&All the roots ending with the affixes \texthindi{सन्} and others are called \texthindi{धातु} ।
			\\\hline
		
		\caption{Sūtras for \texthindi{अस्$+$सन्} function}
		\label{table:a16}
		\end{longtable}
%	\end{center}
%\end{table}


%\begin{table}[h!]
%	\begin{center}
		\begin{longtable}{ |p{1.4cm}|p{14.6cm}| } 
			\hline
			\rowcolor{blue!10}
			\texthindi{स्पर्ध् सन्}
			&\texthindi{धातोः कर्मणः समानकर्तृकादिच्छायां वा} (3.1.7)\\
			\rowcolor{blue!10}
			&The affix is optionally attached, in the sense of 'wishing' after a root expressing the object wished and having the same agent of the action as the wisher thereof. 
			\\\hline
			\rowcolor{blue!10}
			\texthindi{स्पर्ध् सन्}
			&\texthindi{आर्द्धधातुकं शेषः}  (3.4.114)\\
			\rowcolor{blue!10}
			&Except for \texthindi{तिङ्} and \texthindi{शित् प्रत्यय}s, other \texthindi{प्रत्यय}s that are attached to verbs are called '\texthindi{आर्धधातुक}'. 
			\\\hline
			\rowcolor{red!10}
			\texthindi{स्पर्ध् सन्}
			&\texthindi{हलन्त्यम्} (1.3.3) \\
			\rowcolor{red!10}
			&The ending \texthindi{हल्} letter of an \texthindi{उपदेश} is called '\texthindi{इत्}'. 
			\\\hline
			\rowcolor{red!10}
			\texthindi{स्पर्ध् स}
			&\texthindi{तस्य लोपः} (1.3.9)\\
			\rowcolor{red!10}
			&Removal of ‘\texthindi{इत्}’ 
			\\\hline
			\rowcolor{yellow!10}
			\texthindi{स्पर्ध् इस}&\\
			\rowcolor{yellow!10}
			\texthindi{स्पर्धिस}
			&\texthindi{आर्धधातुकस्य इट् वलादेः} (7.2.35)\\
			\rowcolor{yellow!10}
			&An \texthindi{आर्धधातुक-प्रत्यय} that starts with a \texthindi{वल्} letter gets an \texthindi{इट् आगम}.\\ 
			\rowcolor{yellow!10}
			&\texthindi{वल्} - all consonants except \texthindi{य्}
			\\\hline
			\rowcolor{green!10}
			\texthindi{स्पस्पर्धिस}
			&\texthindi{सन्यङोः} (6.1.9)\\
			\rowcolor{green!10}
			&When followed by the \texthindi{सन्-प्रत्यय} or the \texthindi{यङ्-प्रत्यय}, a verb root that have not yet undergone \texthindi{द्वित्वम्} undergoes \texthindi{द्वित्वम्}.
			\\\hline
			\rowcolor{green!10}
			\texthindi{पस्पर्धिस}
			&\texthindi{शर्पूर्वाः खयः} (7.4.61)\\
			\rowcolor{green!10}
			&For an \texthindi{अभ्यास}, all \texthindi{खय्} letters that follows \texthindi{शर्} letters are retained, all other \texthindi{हल्} letters are removed. 
			\\\hline
			\rowcolor{green!10}
			\texthindi{पिस्पर्धिस}
			&\texthindi{सन्यतः} (7.4.79)\\
			\rowcolor{green!10}
			&\texthindi{इ} is substituted for the final short \texthindi{अ} of the reduplicate. 
			\\\hline
			\rowcolor{green!10}
			\texthindi{पिस्पर्धिष}
			&\texthindi{आदेशप्रत्यययोः} (8.3.59)\\
			\rowcolor{green!10}
			&An \texthindi{अपदान्त सकार} that was obtained either as an \texthindi{आदेश} or through a \texthindi{प्रत्यय} is converted to \texthindi{षकार} when it follows a letter from \texthindi{इण्-प्रत्याहार} or a \texthindi{कवर्ग}.
			\\\hline
			\rowcolor{green!10}
			\texthindi{पिस्पर्धिष}
			&\texthindi{सनाद्यन्ता धातवः} (3.1.32)\\
			\rowcolor{green!10}
			&All the roots ending with the affixes \texthindi{सन्} and others are called \texthindi{धातु} ।
			\\\hline
		
		\caption{Sūtras for \texthindi{स्पर्ध्$+$सन्} function}
		\label{table:a17}
		\end{longtable}
%	\end{center}
%\end{table} 


