This function is used for the meaning of ‘again and again’. The \texthindi{ङ्} in the \texthindi{‘यङ्}’ goes away and only \texthindi{य} is left. It is used only for \texthindi{आत्मनेपद} dhātus. \\
A frequentative verb may be derived from any root of the first nine conjugations which are monosyllabic. This kind of secondary verb indicates a repeated or a frequently performed action. A frequentative verb root undergoes reduplication. With a reduplicated root, there are two ways to formulate a frequentative base.
An affix \texthindi{य} is added to this reduplicated root, and then the derived base is conjugated only in the middle (\texthindi{आत्मनेपद}), e.g.\texthindi{ कृ} > \texthindi{चेक्रीयते}.
No \texthindi{य} is added to the reduplicated root, and the base is conjugated only in the active (\texthindi{परस्मैपद}), e.g. \texthindi{कृ} >\texthindi{ चर्कर्ति.}\\
Frequntatives are rare in literature, and only the few verbs which are met in literature are listed below with sample active and middle 3rd person singular forms.
The reduplication of the root involved in frequentative verbs is riddles with options, and the Sanskrit grammarians give an enormous number of alternative forms. For instance, for the form \texthindi{चरीकर्ति} above, we have the following alternatives: \texthindi{चर्कर्ति}, \texthindi{चरिकर्ति}, \texthindi{चर्करीति,}\texthindi{ चरिकरीति} and \texthindi{चरीकरीति}.\\
Theoretically, a frequentative verb can have all possible tenses and moods, though these forms are very rare at best.


\subsection{Definition of \texthindi{यङ्} function}
\texthindi{यङ्}: set of dhātus → set of derived dhātus /  prātipadikas of \texthindi{यङ्}

\subsection{Cases for \texthindi{यङ्} function}
\textbf{Case I:}\\
$v^{\prime}(1)$ $=$ \texthindi{इ/ई उ/ऊ}\\
Then\\
\begin{equation}
	\text{\texthindi{ यङ्}}(x) = c^{\prime}(1)[\text{\texthindi{क्/ख् ग् भ् ष्}}\rightarrow\text{\texthindi{च् ज् ब् स्}}]+v^{\prime}(1)[\text{\texthindi{इ/ई उ/ऊ}}\rightarrow\text{\texthindi{ए ओ}]} + x + \text{\texthindi{य}}
\end{equation}

\begin{table}[h!]
	\begin{center}
		\begin{tabular}{ |c|c|c| } 
			\hline
			$x$&
			$c^{\prime}(1)$ [\text{\texthindi{क्/ख् ग् भ् ष्}}$\rightarrow$ \text{\texthindi{च् ज् ब् स्}}]+ $v^{\prime}(1)$ [\text{\texthindi{इ/ई उ/ऊ}}$\rightarrow$ \text{\texthindi{ए ओ}]} &\texthindi{यङ्}$(x)$\\
			\hline 
			\texthindi{भू}&
			\texthindi{बो}&
			\texthindi{बोभूय}\\
			\texthindi{मुद्}&
			\texthindi{मो}&
			\texthindi{मोमुद्य}\\
			\texthindi{कुर्द्}&
			\texthindi{चो}&
			\texthindi{चोकूर्द्य}\\
			\texthindi{खुर्द्}&
			\texthindi{चो}&
			\texthindi{चोखूर्द्य}\\
			\texthindi{गुर्द्}&
			\texthindi{जो}&
			\texthindi{जोगूर्द्य}\\
			\texthindi{गुद्}&
			\texthindi{जो}&
			\texthindi{जोगुद्य}\\
			\texthindi{मिद्}&
			\texthindi{मे}&
			\texthindi{मेमिद्य}\\
			\texthindi{ष्विद्}&
			\texthindi{से}&
			\texthindi{सेष्विद्य}\\
			\texthindi{विज्}&
			\texthindi{वे}&
			\texthindi{वेविज्य}\\
			\hline
		\end{tabular}
		\caption{Examples of Case I of \texthindi{ यङ्} function }
		\label{table:7.1}
	\end{center}
\end{table}

\textbf{Case II:}\\
If $x$ $\notin$ [\texthindi{वञ्च् कस् पत् पद् स्कन्द् फल् चर् हन् ग्रह् ज्या व्यध् वश् व्यच् व्रश्च् प्रच्छ् भ्रस्ज्}]\\
and $x^{\prime}(2)$ $=$ \texthindi{अ/आ/ऋ}\\
or $x(1)$ $=$ \texthindi{ऐ}\\
then\\ 
\begin{equation}
	\text{\texthindi{ यङ्}}(x) = c^{\prime}(1)[\text{\texthindi{क्/ख् ग्/ह् फ् ध् भ्}}\rightarrow\text{\texthindi{च् ज् प् द् ब्}}]+\text{\texthindi{आ}} + x + \text{\texthindi{य}}
\end{equation}

\begin{table}[h!]
	\begin{center}
		\begin{tabular}{|c|c|c|} 
			\hline
			$x$&
			$c^{\prime}(1)$[\text{\texthindi{क्/ख् ग्/ह् फ् ध् भ्}}$\rightarrow$ \text{\texthindi{च् ज् प् द् ब्}}]+\text{\texthindi{आ}}
			&\texthindi{यङ्}$(x)$\\
			\hline 
			\texthindi{गाध्}&
			\texthindi{जा}&
			\texthindi{जागाध्य}\\
			\texthindi{नाथ्}&
			\texthindi{ना}&
			\texthindi{नानाथ्य}\\
			\texthindi{दध्}&
			\texthindi{दा}&
			\texthindi{दादध्य}\\
			\texthindi{दद्}&
			\texthindi{दा}&
			\texthindi{दादद्य}\\
			\texthindi{पर्द्}&
			\texthindi{पा}&
			\texthindi{पापर्द्य}\\
			\texthindi{ध्यै}&
			\texthindi{दा}&
			\texthindi{दाध्याय}\\
			\texthindi{रै}&
			\texthindi{रा}&
			\texthindi{राराय}\\
			\hline
		\end{tabular}
		\caption{Examples of Case II of \texthindi{ यङ्} function }
		\label{table:7.2}
	\end{center}
\end{table}

\textbf{Case III:}\\
If $x$ $\notin$ \texthindi{वेञ्}\\
and $v^{\prime}(1)$ $=$ \texthindi{ए/ओ}\\
then\\
\begin{equation}
	\text{\texthindi{यङ्}}(x) = c^{\prime}(1)[\text{\texthindi{क्/ख् ग्/ह् फ् ध् भ्}} \rightarrow\text{\texthindi{च् ज् प् द् ब्}}]+v^{\prime}(1)+x+\text{\texthindi{य}}
\end{equation}

\begin{table}[h!]
	\begin{center}
		\begin{tabular}{ |c|c|c| } 
			\hline
			$x$&
			$c^{\prime}(1)$[\text{\texthindi{क्/ख् ग्/ह् फ् ध् भ्}}$\rightarrow$ \text{\texthindi{च् ज् प् द् ब्}}]+ $v^{\prime}(1)$
			&\texthindi{यङ्}$(x)$\\
			\hline 
			\texthindi{वेथ्}&
			\texthindi{वे}&
			\texthindi{वेवेथ्य}\\
			\texthindi{द्रेक्}&
			\texthindi{दे}&
			\texthindi{देद्रेक्य}\\
			\texthindi{लोच्}&
			\texthindi{लो}&
			\texthindi{लोलोच्य}\\
			\hline
		\end{tabular}
		\caption{Examples of Case III of \texthindi{ यङ्} function }
		\label{table:7.3}
	\end{center}
\end{table}

\textbf{Case IV:}\\
If $v^{\prime}(1)$ $=$ \texthindi{औ ऐ} \\
then\\
\begin{equation}
	\text{\texthindi{यङ्}}(x) = c^{\prime}(1)[\text{\texthindi{क्/ख् ग्/ह् फ् ध् भ्}} \rightarrow\text{\texthindi{च् ज् प् द् ब्}}]+\text{\texthindi{औ ऐ}}\xrightarrow{v^{\prime}(1)}\text{\texthindi{ओ ए}}+x+\text{\texthindi{य}}
\end{equation}

\begin{table}[h!]
	\begin{center}
		\begin{tabular}{ |c|c|c| } 
			\hline
			$x$&
			$c^{\prime}(1)$[\text{\texthindi{क्/ख् ग्/ह् फ् ध् भ्}}$\rightarrow$\text{\texthindi{च् ज् प् द् ब्}}]+\text{\texthindi{औ ऐ}}&
			\texthindi{यङ्}$(x)$\\\hline 
			\texthindi{रौड्}&
			\texthindi{रो}&
			\texthindi{रोरौड्य}\\
			\texthindi{पैण्}&
			\texthindi{पे}&
			\texthindi{पेपैण्य}\\\hline
		\end{tabular}
		\caption{Examples of Case IV of \texthindi{ यङ्} function }
		\label{table:7.4}
	\end{center}
\end{table}

\textbf{Case V:}\\
If $x^{\prime}(1)$ $=$ \texthindi{अ} \\
then\\
\begin{equation}
	\text{\texthindi{यङ्}}(x) = x + \text{\texthindi{आ}}+ c^{\prime}(1) + \text{\texthindi{य}}
\end{equation}

\begin{table}[h!]
	\begin{center}
		\begin{tabular}{|c|c|c|c|} 
			\hline
			$x$&
			x + \text{\texthindi{आ}}&$c^{\prime}(1)$&
			\texthindi{यङ्}$(x)$\\\hline 
			\texthindi{अट्}&
			\texthindi{अटा}&
			\texthindi{ट्}&
			\texthindi{अटाट्य}\\
			\texthindi{अश्}&
			\texthindi{अशा}&
			\texthindi{श्}&
			\texthindi{अशाश्य}\\\hline
		\end{tabular}
		\caption{Examples of Case V of \texthindi{ यङ्} function}
		\label{table:7.5}
	\end{center}
\end{table}

\textbf{Case VI:}\\
If $x$ $\epsilon$ [\texthindi{वञ्च् स्रंस् ध्वंस् भ्रंस् कस् पत् पद् स्कन्द्}]\\
and $x^{\prime}(1)$ $=$ \texthindi{अ} \\
then\\
\begin{equation}
	\text{\texthindi{यङ्}}(x) = c^{\prime}(1)[\text{\texthindi{क्/ख् ग्/ह् फ् ध् भ्}}\rightarrow\text{\texthindi{च् ज् प् द् ब्}}]+\text{\texthindi{नी}}+x[-\text{\texthindi{ञ् म् न्}}]+\text{\texthindi{य}}
\end{equation}


\begin{table}[h!]
	\begin{center}
		\begin{tabular}{|c|c|c|c|} 
			\hline
			$x$&
			$c^{\prime}(1)$[\text{\texthindi{क्/ख् ग्/ह् फ् ध् भ्}}$\rightarrow$\text{\texthindi{च् ज् प् द् ब्}}]+\text{\texthindi{नी}}&x[-\text{\texthindi{ञ् म् न्}}]+\text{\texthindi{य}}&
			\texthindi{यङ्}$(x)$\\
			\hline 
			\texthindi{वञ्च्}&
			\texthindi{वनी}&
			\texthindi{वच्य}&
			\texthindi{वनीवच्य}\\
			\texthindi{ध्वंस्}&
			\texthindi{दनी}&
			\texthindi{ध्वस्य}&
			\texthindi{दनीध्वस्य}\\
			\texthindi{कसि (कस्)}&
			\texthindi{चनी}&
			\texthindi{कस्य}&
			\texthindi{चनीकस्य}\\\hline
		\end{tabular}
		\caption{Examples of Case VI of \texthindi{यङ्} function}
		\label{table:7.6}
	\end{center}
\end{table}

\textbf{Case VII:}\\
If $x$ $\epsilon$ [\texthindi{स्फूर्ज् स्फुट्}]\\
then\\
\begin{equation}
	\text{\texthindi{यङ्}}(x) = \text{\texthindi{पो}}+ x + \text{\texthindi{य}}
\end{equation}

\begin{table}[h!]
	\begin{center}
		\begin{tabular}{ |c|c|c| } 
			\hline
			$x$&
			\texthindi{पो} + $x$&
			\texthindi{यङ्}$(x)$\\\hline 
			\texthindi{स्फूर्ज्}&
			\texthindi{पोष्फूर्ज्}&
			\texthindi{पोष्फूर्ज्य}\\
			\texthindi{स्फुट्}&
			\texthindi{पोष्फुट्}&
			\texthindi{पोष्फुट्य}\\\hline
		\end{tabular}
		\caption{Examples of Case VII of \texthindi{यङ्} function}
		\label{table:7.7}
	\end{center}
\end{table}

\textbf{Case VIII:}\\
If $x$ $\epsilon$ [\texthindi{फल् चर्}]\\
then\\
\begin{equation}
	\text{\texthindi{यङ्}}(x) = \text{\texthindi{प/च}}+ \text{\texthindi{न्}}+ x[\text{\texthindi{अ}}\xrightarrow{v^{\prime}(1)}\text{\texthindi{उ}}] + \text{\texthindi{य}}
\end{equation}

\begin{table}[h!]
	\begin{center}
		\begin{tabular}{ |c|c|c| } 
			\hline
			$x$&
			\text{\texthindi{प/च}}+ \text{\texthindi{न्}}
			&\texthindi{यङ्}$(x)$\\\hline 
			\texthindi{फल्}&
			\texthindi{पन्}&
			\texthindi{पन्फुल्य}\\
			\texthindi{चर्}&
			\texthindi{चन्}&
			\texthindi{चन्चुर्य}\\\hline
		\end{tabular}
		\caption{Examples of Case VIII of \texthindi{यङ्} function}
		\label{table:7.8}
	\end{center}
\end{table}

\textbf{Case IX:}\\
If $x(1)$ $=$ \text{\texthindi{ऋ}},
\begin{itemize}
	\item 
	If $x(2)$ $\neq$ \texthindi{प् फ् ब् भ् म्}\\
	\begin{equation}
		\text{\texthindi{यङ्}}(x) = x(2) + \text{\texthindi{ए}}+x(2)+ \text{\texthindi{झर्}}+ \text{\texthindi{य}}
	\end{equation}
	
	\item 
	If $x(2)$ $=$ \texthindi{प् फ् ब् भ् म्}\\
	\begin{equation}
		\text{\texthindi{यङ्}}(x) = x(2) + \text{\texthindi{ओ}}+x(2)+ \text{\texthindi{ऊर्}}+ \text{\texthindi{य}}
	\end{equation}
\end{itemize}

\begin{table}[h!]
	\begin{center}
		\begin{tabular}{|c|c|} 
			\hline
			$x$&
			\texthindi{यङ्}$(x)$\\\hline 
			\texthindi{नॄ}&
			\texthindi{नेनीर्य}\\
			\texthindi{पॄ}&
			\texthindi{पोपूर्य}\\
			\texthindi{भॄ}&
			\texthindi{बोभूर्य}\\
			\texthindi{दृ}&
			\texthindi{देद्रीय}\\\hline
		\end{tabular}
		\caption{Examples of Case IX of \texthindi{ यङ्} function}
		\label{table:7.9}
	\end{center}
\end{table}

\textbf{Case X:}\\
If $x$ $\epsilon$ [\texthindi{दा घा(घु) मा स्था गा पा हा सा(सो)}]\\
then\\
\begin{equation}
	\text{\texthindi{यङ्}}(x) = c^{\prime}(1)[\text{\texthindi{क्/ख् ग्/ह् फ् ध् भ्}} \rightarrow \text{\texthindi{च् ज् प् द् ब्}}] +\text{\texthindi{ए}}+x[\text{\texthindi{अ}}\xrightarrow{v^{\prime}(1)}\text{\texthindi{ई}}]+\text{\texthindi{य}}
\end{equation}

\begin{table}[h!]
	\begin{center}
		\begin{tabular}{|c|c|c|} 
			\hline
			$x$&
			$c^{\prime}(1)$[\text{\texthindi{क्/ख् ग्/ह् फ् ध् भ्}} $\rightarrow$ \text{\texthindi{च् ज् प् द् ब्}}]+\text{\texthindi{ए}}&
			\texthindi{यङ्}$(x)$\\\hline 
			\texthindi{पा}&
			\texthindi{पे}&
			\texthindi{पेपीय}\\
			\texthindi{गा}&
			\texthindi{जे}&
			\texthindi{जेगीय}\\
			\texthindi{हा}&
			\texthindi{जे}&
			\texthindi{जेहीय}\\
			\texthindi{ष्ठा (स्था)}&
			\texthindi{ते}&
			\texthindi{तेष्थीय}\\
			\hline
		\end{tabular}
		\caption{Examples of Case X of \texthindi{ यङ्} function}
		\label{table:7.10}
	\end{center}
\end{table}


\textbf{Case XI}\\
If $x(1)$ $=$ \texthindi{ङ् ञ् ण्} \\
and $x(2)$ $=$ \texthindi{ऋ}\\
then\\
\begin{equation}
	\text{\texthindi{यङ्}}(x) = c^{\prime}(1)[\text{\texthindi{क्/ख् ग्/ह् फ् ध् भ्}} \rightarrow\text{\texthindi{च् ज् प् द् ब्}}]+\text{\texthindi{ए}}+x^{\prime}(1)+\text{\texthindi{र्}}+\text{\texthindi{ई}}]+\text{\texthindi{य}}
\end{equation}

\begin{table}[h!]
	\begin{center}
		\begin{tabular}{ |c|c|c| } 
			\hline
			$x$&
			$c^{\prime}(1)$[\text{\texthindi{क्/ख् ग्/ह् फ् ध् भ्}} $\rightarrow$ \text{\texthindi{च् ज् प् द् ब्}}]+\text{\texthindi{ए}}
			&\texthindi{यङ्}$(x)$\\\hline 
			\texthindi{भृञ्}&
			\texthindi{बे}&
			\texthindi{बेभ्रीय}\\
			\texthindi{कृञ्}&
			\texthindi{चे}&
			\texthindi{चेक्रीय}\\
			\texthindi{वृञ्}&
			\texthindi{वे}&
			\texthindi{वेव्रीय}\\
			\hline
		\end{tabular}
		\caption{Examples of Case XI of \texthindi{ यङ्} function}
		\label{table:7.11}
	\end{center}
\end{table}


\textbf{Case XII:}\\
If $x$ $\epsilon$ [\texthindi{जन् सन् खन्}]\\
then\\
\begin{equation}
	\text{\texthindi{यङ्}}(x) = \begin{cases}
		c^{\prime}(1)[\text{\texthindi{क्/ख् ग्/ह् फ् ध् भ्}} \rightarrow \text{\texthindi{च् ज् प् द् ब्}}]+\text{\texthindi{आ}} +x+\text{\texthindi{य}}\\
		c^{\prime}(1)[\text{\texthindi{क्/ख् ग्/ह् फ् ध् भ्}} \rightarrow \text{\texthindi{च् ज् प् द् ब्}}]+\text{\texthindi{आ}}+x^{\prime}(1)+\text{\texthindi{आ}}+\text{\texthindi{य}}
\end{cases}
\end{equation}

\begin{table}[h!]
\begin{center}
	\begin{tabular}{ |c|c| } 
		\hline
		$x$&
		\texthindi{यङ्}$(x)$\\
		\hline 
		\multirow{2}{*}{\texthindi{खन्}}
		&\texthindi{चाखन्य}\\
		&\texthindi{चाखाय}\\

		\multirow{2}{*}{\texthindi{जन्}}
		&\texthindi{जाजन्य}\\
		&\texthindi{जाजाय}\\

		\multirow{2}{*}{\texthindi{जन्}}
		&\texthindi{जाजन्य}\\
		&\texthindi{जाजाय}\\
		\hline
	\end{tabular}
	\caption{Examples of Case XII of \texthindi{यङ्} function}
	\label{table:7.12}
\end{center}
\end{table}

\textbf{Case XIII:}\\
If $x$ $\epsilon$ [\texthindi{घ्रा ध्मा ग्रह् ज्या वेञ् व्यध् वश् व्यच् व्रश्च् प्रच्छ्}]\\
then\\
\begin{equation}
\text{\texthindi{यङ्}}(x) = c^{\prime}(1)[\text{\texthindi{क्/ख् ग्/ह् फ् ध् भ्}} \rightarrow\text{\texthindi{च् ज् प् द् ब्}}]+\text{\texthindi{ए}} +x[\text{\texthindi{आ}}\xrightarrow{1}\text{\texthindi{ई}}]+\text{\texthindi{य}}
\end{equation}

\begin{table}[h!]
\begin{center}
	\begin{tabular}{ |c|c|c| } 
		\hline
		$x$&
		$c^{\prime}(1)$[\text{\texthindi{क्/ख् ग्/ह् फ् ध् भ्}}$\rightarrow$ \text{\texthindi{च् ज् प् द् ब्}}]+\text{\texthindi{ए}}] 
		&\texthindi{यङ्}$(x)$\\\hline 
		\texthindi{घ्रा}&
		\texthindi{जे}&
		\texthindi{जेघ्रीय}\\
		\texthindi{ध्मा}&
		\texthindi{दे}&
		\texthindi{देध्मीय}\\
		\texthindi{ज्या}&
		\texthindi{जे}&
		\texthindi{जेजीय}\\
		\texthindi{व्यध्}&
		\texthindi{वे}&
		\texthindi{वेविध्य}\\
		\hline
	\end{tabular}
	\caption{Examples of Case XIII of \texthindi{यङ्} function}
	\label{table:7.13}
\end{center}
\end{table} 


\textbf{Case XIV:}\\
If $x$ $=$ \texthindi{हन्}\\
then\\
\begin{equation}
\text{\texthindi{यङ्}}(x) = \begin{cases}
	c^{\prime}(1)[\text{\texthindi{ह्}} \rightarrow\text{\texthindi{ज्}} ] + \text{\texthindi{ए}} + \text{\texthindi{घ्न्}}+x[\texthindi{आ}\xrightarrow{1}\text{\texthindi{ई}}]+\text{\texthindi{य}}\\
	c^{\prime}(1)[\text{\texthindi{ह्}} \rightarrow\text{\texthindi{ज्}}] +\text{\texthindi{अ}} +\text{\texthindi{घन्}}+\text{\texthindi{य}}\\
\end{cases}
\end{equation}

\begin{table}[h!]
\begin{center}
	\begin{tabular}{ |c|c|c| } 
		\hline
		$x$&
		$\begin{cases}
			c^{\prime}(1)[\text{\texthindi{ह्}} \rightarrow\text{\texthindi{ज्}}] + \text{\texthindi{ए}} +\text{\texthindi{घ्न्}}\\
			c^{\prime}(1) [\text{\texthindi{ह्}} \rightarrow \text{\texthindi{ज्}}] + \text{\texthindi{अ}} +\text{\texthindi{घन्}}\\
		\end{cases}$&
		\texthindi{यङ्}$(x)$\\
		\hline 
		\multirow{2}{*}{\texthindi{ हन्}}
		&\texthindi{जेघ्न्}
		&\texthindi{जेघ्नीय}\\
		&\texthindi{जघन्}
		&\texthindi{जघन्य}\\





	\hline
	\end{tabular}
	\caption{Examples of Case XIV of \texthindi{ यङ्} function}
	\label{table:7.14}
\end{center}
\end{table}

\textbf{Case XV:}\\
If $x$ has only two syllables\\
and $x^{\prime}(2)$ $=$ \texthindi{ऋ}\\
then\\
\begin{equation}
\text{\texthindi{यङ्}}(x) = c^{\prime}(1)[\text{\texthindi{क्/ख् ग्/ह् फ् ध् भ्}} \rightarrow\text{\texthindi{च् ज् प् द् ब्}}]+\text{\texthindi{ए}} +x[\text{\texthindi{ऋ}}\xrightarrow{1}\text{\texthindi{र्}}+\text{\texthindi{ई}}]+\text{\texthindi{य}}
\end{equation}

\begin{table}[h!]
\begin{center}
	\begin{tabular}{ |c|c|c| } 
		\hline
		$x$&
		$c^{\prime}(1)$[\text{\texthindi{क्/ख् ग्/ह् फ् ध् भ्}} $\rightarrow$ \text{\texthindi{च् ज् प् द् ब्}}]+\text{\texthindi{ए}} &\texthindi{यङ्}$(x)$ \\
		\hline 
		\texthindi{षृ}&
		\texthindi{से}&
		\texthindi{सेष्रीय}\\
		\texthindi{घृ}&
		\texthindi{जे}&
		\texthindi{जेघ्रीय}\\
		\texthindi{हृ}&
		\texthindi{जे}&
		\texthindi{जेह्रीय}\\
		\texthindi{गृ}&
		\texthindi{जे}&
		\texthindi{जेग्रीय}\\
		\texthindi{घृ}&
		\texthindi{जे}&
		\texthindi{जेघ्रीय}\\\hline
	\end{tabular}
	\caption{Examples of Case XV of \texthindi{ यङ्} function}
	\label{table:7.15}
\end{center}
\end{table}

\textbf{Case XVI:}\\
If $x$ $=$ \texthindi{ऊर्णु}
then\\ 
\begin{equation}
\text{\texthindi{यङ्}}(x) =x[\text{\texthindi{उ}} \xrightarrow{1}\text{\texthindi{ओ}}]+\text{\texthindi{नू}} + \text{\texthindi{य}}  
\end{equation}

\begin{table}[h!]
\begin{center}
	\begin{tabular}{ |c|c|c| } 
		\hline
		$x$&
		$x[\text{\texthindi{उ}} \xrightarrow{1}\text{\texthindi{ओ}}]$&
		\texthindi{यङ्}$(x)$\\\hline 
		\texthindi{ऊर्णु}&
		\texthindi{ऊर्णो}&
		\texthindi{ऊर्णोनूय}\\\hline
	\end{tabular}
	\caption{Examples of Case XVI of \texthindi{ यङ्} function}
	\label{table:7.16}
\end{center}
\end{table}
Now, lets move on to the \texthindi{सन्} function for the \texthindi{सन्} pratyayas.