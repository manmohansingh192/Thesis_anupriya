\renewcommand{\thepage}{\roman{page}} \setcounter{page}{1}
Pāṇini in his Aṣṭādhyāyī has written the grammar of Sanskrit in an extremely concise manner in the form of about 4000 sūtras. We have attempted to mathematically remodel the data produced by these sūtras. The mathematical modelling is a way to show that the Pāṇinian approach is a minimal method of capturing the grammatical data for Sanskrit which is a natural language. The sūtras written by Pāṇini can be written as functions, that is for a single input the function produces a single output of the form y=f(x), where x and y is the input and output respectively. However, we observe that for some input dhātus, we get multiple outputs. For such cases, we have written multivalued functions that is the functions which give two or more outputs for a single input. In other words, multivalued function is a way to represent optional output forms which are expressed in Pāṇinian grammar with the help of 3 terms i.e. vā, vibhaṣā, and anyatarasyam. Comparison between the techniques employed by Pāṇini and our notation of functions helps us understand how Pāṇinian techniques ensure brevity and terseness, hence illustrating that Pāṇinian grammar is minimal.