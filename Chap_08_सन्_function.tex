This function is used for the meaning of ‘again and again. The \texthindi{ङ्} in the ‘\texthindi{यङ्} goes away and only \texthindi{य} is left. It is used only for \texthindi{आत्मनेपद }dhātus. \\
to express a meaning such as ‘x wants to go', Sanskrit has at least two possible ways. The first way is to use a periphrastic expression such as \texthindi{गन्तुम् इच्छति}, an infinitive “to go” with a verb of desire. The other way is to use a morphological desiderative, e.g. \texthindi{जिगमिषति}. Such a morphological desiderative may be derived from a root of any of the ten conjugations, and theoretically it may be conjugated in all tenses and moods. Generally, the desideratives for \texthindi{परस्मैपद} and \texthindi{आत्मनेपद} verbs retain the same classification (Madhav M. Deshpande, 2007). The two noteworthy features of desideratives are\\
a)	reduplication of the verb, and\\
b)	the affix \texthindi{स} (variants, -\texthindi{ष}, -\texthindi{इष})\\

\subsection{Definition of \texthindi{सन्} function}
\texthindi{सन्}: set of dhātus → set of derived dhātus /  prātipadikas of \texthindi{सन्}

\subsection{Cases for \texthindi{सन्} function}

\textbf{Case I:}\\
If $x^{\prime}(1)=c$, $x^{\prime}(2)=v\neq$ \texthindi{इ उ ऋ } \\
and $x^{\prime}(4)=c$(if present) in $x$\\
then\\
\begin{equation}
	\text{\texthindi{सन्}} (x)=c^{\prime}(1)[\text{\texthindi{क्/ख् ग् भ् ष् ह्}}\rightarrow\text{\texthindi{च् ज् ब् स् ज्}}] + v^{\prime}(1)[\text{\texthindi{अ/आ/ए/ई ऊ उ}}\rightarrow\text{\texthindi{इ उ}}]+x+p(x)+\text{\texthindi{ष}}
\end{equation}

\begin{table}[h!]
	\begin{center}
		\begin{tabular}{|c|c|c|c|} 
			\hline
			$x$&
			$c^{\prime}(1)$[\text{\texthindi{क्/ख् ग् भ् ष् ह्}}$\rightarrow$\text{\texthindi{च् ज् ब् स् ज्}}] + $v^{\prime}(1)$[\text{\texthindi{अ/आ/ए/ई ऊ उ}}$\rightarrow$\text{\texthindi{इ उ}}]&
			x+p(x)+\text{\texthindi{ष}}
			&\texthindi{सन्}$(x)$\\\hline 
			\texthindi{भू}&
			\texthindi{बु}&
			\texthindi{भूष}&
			\texthindi{बुभूष}\\
			\texthindi{गाध्}&
			\texthindi{जि}&
			\texthindi{गाधिष}&
			\texthindi{जिगाधिष}\\
			\texthindi{बाध्}&
			\texthindi{बि}&
			\texthindi{बाधिष}&
			\texthindi{बिबाधिष}\\
			\texthindi{दध्}&
			\texthindi{दि}&
			\texthindi{दधिष}&
			\texthindi{दिदधिष}\\
			\texthindi{वन्द्}&
			\texthindi{वि}&
			\texthindi{वन्दिष}&
			\texthindi{विवन्दिष}\\
			\texthindi{कूर्द्}&
			\texthindi{चु}&
			\texthindi{कूर्दिष}&
			\texthindi{चुकूर्दिष}\\
			\texthindi{सूद्}&
			\texthindi{सु}&
			\texthindi{सूदिष}&
			\texthindi{सुसूदिष}\\
			\texthindi{वेथ्}&
			\texthindi{वि}&
			\texthindi{वेथिष}&
			\texthindi{विवेथिष}\\
			\texthindi{भी}&
			\texthindi{बि}&
			\texthindi{भीष}&
			\texthindi{बिभीष}\\
			\texthindi{गा}&
			\texthindi{जि}&
			\texthindi{गास}&
			\texthindi{जिगास}\\
			\texthindi{कथ्}&
			\texthindi{चि}&
			\texthindi{कथिष}&
			\texthindi{चिकथिष}\\\hline
		\end{tabular}
		\caption{Examples of Case I of \texthindi{सन्} function}
		\label{table:8.1}
	\end{center}
\end{table} 

\textbf{Case II:}\\
If $x^{\prime}(1)=c$, $x^{\prime}(2)=v=$\texthindi{ऋ} $x^{\prime}(3)=c$ in $x$(exactly 3 letters)\\
then\\
\begin{equation}
	\text{\texthindi{सन्}(x)} = c^{\prime}(1)[\text{\texthindi{क्/ख् ग् भ् ष् ह्}}\rightarrow\text{\texthindi{च् ज् ब् स् ज्}}]+v^{\prime}(1)[\text{\texthindi{ऋ}}\xrightarrow{2}\text{\texthindi{इ}}]+x[\text{\texthindi{ऋ}}\xrightarrow{2}\text{\texthindi{अर्}}] +p(x)+\text{\texthindi{ष}}
\end{equation}

\begin{table}[h!]
	\begin{center}
		\begin{tabular}{ |c|c|c|c| } 
			\hline
			$x$&
			$c^{\prime}(1)$[\text{\texthindi{क्/ख् ग् भ् ष् ह्}}$\rightarrow$\text{\texthindi{च् ज् ब् स् ज्}}]+$v^{\prime}(1)$[\text{\texthindi{ऋ}}$\xrightarrow{2}$\text{\texthindi{इ}}]&
			x[\text{\texthindi{ऋ}}$\xrightarrow{2}$\text{\texthindi{अर्}}] +p(x)+\text{\texthindi{ष}}
			&\texthindi{सन्}$(x)$\\\hline  
			\texthindi{मृड्}&
			\texthindi{मि}&
			\texthindi{मर्डिष}&
			\texthindi{मिमर्डिष}\\
			\texthindi{मृद्}&
			\texthindi{मि}&
			\texthindi{मर्दिष}&
			\texthindi{मिमर्दिष}\\
			\texthindi{गृह्}&
			\texthindi{जि}&
			\texthindi{गर्हिष}&
			\texthindi{जिगर्हिष}\\\hline
		\end{tabular}
		\caption{Examples of Case II of \texthindi{सन्} function}
		\label{table:8.2}
	\end{center}
\end{table} 

\textbf{Case III:}\\
If $x^{\prime}(1)=c$ and $x^{\prime}(2)=v=$\texthindi{ऋ} and $x^{\prime}(3)= \phi$ in $x$ (exactly 2 letters)\\
then\\

\begin{equation}
	\text{\texthindi{सन्}} (x)= c^{\prime}(1)[\text{\texthindi{क्/ख् ग् भ् ष् ह्}}\rightarrow\text{\texthindi{च् ज् ब् स् ज्}}]+v^{\prime}(1)[\text{\texthindi{ऋ}}\xrightarrow{2}\text{\texthindi{इ}}]+v^{\prime}(1)[\text{\texthindi{ऋ}}\xrightarrow{2}\text{\texthindi{ईर्}}] +p(x)+\text{\texthindi{ष}}
\end{equation}


\begin{table}[h!]
	\begin{center}
		\begin{tabular}{ |c|c|c|c| } 
			\hline
			$x$&
			$c^{\prime}(1)$[\text{\texthindi{क्/ख् ग् भ् ष् ह्}}$\rightarrow$\text{\texthindi{च् ज् ब् स् ज्}}]+$v^{\prime}(1)$[\text{\texthindi{ऋ}}$\xrightarrow{2}$\text{\texthindi{इ}}]&
			$v^{\prime}(1)$[\text{\texthindi{ऋ}}$\xrightarrow{2}$\text{\texthindi{ईर्}}] +p(x)+\text{\texthindi{ष}}
			&\texthindi{सन्}$(x)$
			\\
			\hline  
			\texthindi{कृ}&
			\texthindi{चि}&
			\texthindi{कीर्ष}&
			\texthindi{चिकीर्ष}\\
			\texthindi{जृ}&
			\texthindi{जि}&
			\texthindi{जीर्ष}&
			\texthindi{जिजीर्ष}\\
			\texthindi{हृ}&
			\texthindi{जि}&
			\texthindi{हीर्ष}&
			\texthindi{जिहीर्ष}\\\hline
		\end{tabular}
		\caption{Examples of Case III of \texthindi{सन्} function}
		\label{table:8.3}
	\end{center}
\end{table} 

\textbf{Case IV:}\\
If $x$ $\epsilon$ [\texthindi{मी मा धा पत् पद् दा}]\\
then\\ 
\begin{equation}
	\text{\texthindi{सन्}} (x)= x [v^{\prime}(1) \rightarrow \text{\texthindi{इत्}} ,x^{\prime}(3)\rightarrow\phi]+\text{\texthindi{स}}
\end{equation}

\begin{table}[h!]
\begin{center}
	\begin{tabular}{ |c|c|c| } 
		\hline
		$x$&
		x[$v^{\prime}(1)$ $\rightarrow$ \text{\texthindi{इत्}},$x^{\prime}(3)$ $\rightarrow$ $\phi$]&
		\texthindi{सन्}$(x)$
		\\
		\hline  
		\texthindi{मा}&
		\texthindi{मित्}&
		\texthindi{मित्स}\\
		\texthindi{मी}&
		\texthindi{मित्}&
		\texthindi{मित्स}\\
		\texthindi{पत् }&
		\texthindi{पित्}&
		\texthindi{पित्स}\\
		\texthindi{पद्}&
		\texthindi{पित्}&
		\texthindi{पित्स}\\
		\texthindi{दा}&
		\texthindi{दित्}&
		\texthindi{दित्स}\\
		\hline
	\end{tabular}
	\caption{Examples of Case IV of \texthindi{सन्} function}
	\label{table:8.4}
\end{center}
\end{table} 

\textbf{Case V:}\\
If $x$ $=$ [\texthindi{रभ् लभ्}]\\
then\\

\begin{equation}
\text{\texthindi{सन्}} (x)= x[v^{\prime}(1)\rightarrow\text{\texthindi{इप्}},x^{\prime}(3)\rightarrow\phi]+\text{\texthindi{स}}
\end{equation}

\begin{table}[h!]
\begin{center}
	\begin{tabular}{ |c|c|c| } 
		\hline
		$x$&
		x[$v^{\prime}(1)$ $\rightarrow$ \text{\texthindi{इप्}},$x^{\prime}(3)$ $\rightarrow$ $\phi$]&
		\texthindi{सन्}$(x)$
		\\
		\hline  
		\texthindi{रभ्}&
		\texthindi{रिप्}&
		\texthindi{रिप्स}\\
		\texthindi{लभ्}&
		\texthindi{लिप्}&
		\texthindi{लिप्स}\\
		\hline
	\end{tabular}
	\caption{Examples of Case V of \texthindi{सन्} function}
	\label{table:8.5}
\end{center}
\end{table} 

\textbf{Case VI:}\\
If $x$ $=$ \texthindi{शक्}\\
then\\
\begin{equation}
	\text{\texthindi{सन्}} (x)= x[\text{\texthindi{अ}}\xrightarrow {v^{\prime}(1)} \text{\texthindi{इ}} ]+\text{\texthindi{ष}}
\end{equation}

\begin{table}[h!]
	\begin{center}
		\begin{tabular}{|c|c|c|} 
			\hline
			x &
			x [\text{\texthindi{अ}}$\xrightarrow {v^{\prime}(1)}$ \text{\texthindi{इ}} ]&
			\text{\texthindi{सन्}}$(x)$\\
			\hline  
			\texthindi{शक्}&
			\texthindi{शिक्}&
			\texthindi{शिक्ष}\\
			\hline
		\end{tabular}
		\caption{Examples of Case VI of \texthindi{सन्} function}
		\label{table:8.6}
	\end{center}
\end{table} 

\textbf{Case VII:}\\
If $x^{\prime}(1)=v$ and $x^{\prime}(2)=c$\\
then\\
the intermediate form is\\
\begin{equation}
i(x)= x[\text{\texthindi{इ उ ऋ }}\xrightarrow{1^{\prime}} \text{\texthindi{ए ओ अर्}}; \text{\texthindi{ध्}}\xrightarrow{2^{\prime}}\text{\texthindi{द्}}]+\text{\texthindi{इ}}  
\end{equation}
and\\
\begin{equation}
\text{\texthindi{सन्}}(x)=  i[v^{\prime}(1)\rightarrow\phi;\text{\texthindi{स् द्}}\rightarrow\text{\texthindi{ष् ध्]}}+\text{\texthindi{ष}}
\end{equation}

\begin{table}[h!]
\begin{center}
	\begin{tabular}{ |c|c|c|c| } 
		\hline
		$x$&
		$i(x)$&
		i[$v^{\prime}(1)$ $\rightarrow$ $\phi$]
		&\texthindi{सन्}$(x)$\\
		\hline 
		\texthindi{अत्}&
		\texthindi{अति}&
		\texthindi{ति}&
		\texthindi{अतितिष}\\
		\texthindi{अश्}&
		\texthindi{अशि}&
		\texthindi{शि}&
		\texthindi{अशिशिष}\\
		\texthindi{ऋ}&
		\texthindi{अरि}&
		\texthindi{रि}&
		\texthindi{अरिरिष}\\
		\texthindi{अस्}&
		\texthindi{असि}&
		\texthindi{सि}&
		\texthindi{असिषिष}\\
		\texthindi{एध्}&
		\texthindi{एदि}&
		\texthindi{दि}&
		\texthindi{एदिधिष}\\
		\hline
	\end{tabular}
	\caption{Examples of Case VII of \texthindi{सन्} }
	\label{table:8.7}
\end{center}
\end{table} 

\textbf{Case VIII:}\\

If $x^{\prime}(1)=v$ and\\
$c^{\prime}(1)=$\texthindi{ङ् ञ् ण् न् र्}\\
then\\
\begin{equation}
\text{\texthindi{सन्}}(x)=  x +\text{\texthindi{इ}} + c(1) + \text{\texthindi{इ}} + \text{\texthindi{ष}}
\end{equation}

\begin{table}[h!]
\begin{center}
	\begin{tabular}{|c|c|c|c|} 
		\hline
		$x$&
		$ x + \texthindi{इ}$&
		c(1) + \texthindi{इ} + \texthindi{ष}
		&\texthindi{सन्}$(x)$\\
		\hline 
		\texthindi{ऊर्द्}&
		\texthindi{ऊर्दि}&
		\texthindi{दिष}&
		\texthindi{ऊर्दिदिष}\\
		\texthindi{अङ्ग्}&
		\texthindi{अञ्जि}&
		\texthindi{गिष}&
		\texthindi{अञ्जिगिष}\\
		\texthindi{अञ्ज्}&
		\texthindi{अञ्जि}&
		\texthindi{जिष}&
		\texthindi{अञ्जिजिष}\\
	\hline
	\end{tabular}
	\caption{Examples of Case VIII of \texthindi{सन्} function}
	\label{table:8.8}
\end{center}
\end{table} 

\textbf{Case IX:}\\

If $x^{\prime}(1)=c$, $x^{\prime}(2)=c$ and $x^{\prime}(3)=v\neq$\texthindi{इ उ ऋ}\\


\begin{equation}
\text{\texthindi{सन्}} (x)= a[\text{\texthindi{क्/ख् ग् भ् ष् ह्}}\rightarrow\text{\texthindi{च् ज् ब् स् ज्}}]+v^{\prime}(1)[\text{\texthindi{अ/आ}}\rightarrow\text{\texthindi{इ}}]+x+p(x)+\text{\texthindi{ष}}
\end{equation}

where $a$ is determined by the method shown in Figure~\ref{fig:a}.
\begin{figure}[!h]
	\centering
	\includegraphics[width=1.0\textwidth]{Figures/a.png}
	\hspace{1mm}
	\caption{Determining $a$} 
	\label{fig:8.1}
\end{figure}

\begin{table}[h!]
\begin{center}
	\begin{tabular}{ |c|c|c|c| } 
		\hline
		$x$&
		a[\texthindi{क्/ख् ग् भ् ष् ह्}$\rightarrow$ \texthindi{च् ज् ब् स् ज्}]+ $v^{\prime}(1)$ [\texthindi{अ/आ}$\rightarrow$ \texthindi{इ}]&
		x+p(x)+\text{\texthindi{ष}}
		&\texthindi{सन्}$(x)$
		\\
		\hline 
		\texthindi{स्पर्ध}&
		\texthindi{पि}&
		\texthindi{स्पर्धिष}&
		\texthindi{पिस्पर्धिष}\\
		\texthindi{स्कुन्द्}&
		\texthindi{चु}&
		\texthindi{स्कुन्दिष}&
		\texthindi{चुस्कुन्दिष}\\
		\texthindi{ख्या}&
		\texthindi{चि}&
		\texthindi{ख्यास}&
		\texthindi{चिख्यास}\\
		\texthindi{ह्राद }&
		\texthindi{जि}&
		\texthindi{ह्रादिष}&
		\texthindi{जिह्रादिष}\\
		\texthindi{श्रन्थ्}&
		\texthindi{शि}&
		\texthindi{श्रन्थिष}&
		\texthindi{शिश्रन्थिष}\\
		\texthindi{श्वस्}&
		\texthindi{शि}&
		\texthindi{श्वसिष}&
		\texthindi{शिश्वसिष}\\
		\texthindi{स्वद्}&
		\texthindi{सि}&
		\texthindi{स्वदिष}&
		\texthindi{सिस्वदिष}\\
		\texthindi{स्वर्द्}&
		\texthindi{सि}&
		\texthindi{स्वर्दिष}&
		\texthindi{सिस्वर्दिष}\\
		\hline
	\end{tabular}
	\caption{Examples of Case IX of \texthindi{सन्} function}
	\label{table:8.9}
\end{center}
\end{table}

\textbf{Multivalued functions:}\\
\textbf{Case I:}\\
If $x$ $=$ \texthindi{दरिद्रा}\\

\begin{equation}
\text{\texthindi{सन्}} (x)= 
\begin{cases}
	c^{\prime}(1)+v^{\prime}(1)[\text{\texthindi{अ }}\rightarrow\text{\texthindi{इ}}]+x+\text{\texthindi{स}}\\
	c^{\prime}(1)+v^{\prime}(1)[\text{\texthindi{अ }}\rightarrow\text{\texthindi{इ}}]+x(1)[\text{\texthindi{आ}}\rightarrow\phi]+\text{\texthindi{इष}}\\
\end{cases} 
\end{equation}


\begin{table}[h!]
\begin{center}
	\begin{tabular}{ |c|c|c| } 
		\hline
		$x$&
		$c^{\prime}(1)$ + $v^{\prime}(1)$ [\text{\texthindi{अ }}$\rightarrow$\text{\texthindi{इ}}]&
		\texthindi{सन्}$(x)$
		\\
		\hline 
		\multirow{2}{*}{\texthindi{दरिद्रा}}
		&\multirow{2}{*}{\texthindi{दि}}
		&\texthindi{दिदरिद्रास}\\
		&
		&\texthindi{दिदरिद्रिष}\\

	\hline
	\end{tabular}
	\caption{Examples of Case I for multivalued functions of \texthindi{सन्} }
	\label{table:8.10}
\end{center}
\end{table}

\textbf{Case II:}\\
If $x^{\prime}(1)=c$, \\
$x^{\prime}(2)=v= $ \texthindi{इ उ}\\
$x^{\prime}(3)=c$ in x(which has exactly 3 letters)\\
then\\
\begin{equation}
\text{\texthindi{सन्}} (x)= 
\begin{cases}
	c^{\prime}(1)[\text{\texthindi{क् ग् भ् ष् ह्}}\rightarrow\text{\texthindi{च् ज् ब् स् ज्}}]+v^{\prime}(1)+x[\text{\texthindi{इ उ }}\rightarrow\text{\texthindi{ए ओ}}]+\text{\texthindi{इष}}\\
	c^{\prime}(1)[\text{\texthindi{क् ग् भ् ष् ह्}}\rightarrow\text{\texthindi{च् ज् ब् स् ज्}}]+v^{\prime}(1)+x+\text{\texthindi{इष}}\\
\end{cases} 
\end{equation}

\begin{table}[h!]
\begin{center}
	\begin{tabular}{ |c|c|c|c| } 
		\hline
		$x$&
		$c^{\prime}(1)$ [\text{\texthindi{क् ग् भ् ष् ह्}}$\rightarrow$ \text{\texthindi{च् ज् ब् स् ज्}}&
		$\begin{cases}
			x[\text{\texthindi{इ उ }}\rightarrow\text{\texthindi{ए ओ}}+\text{\texthindi{इष}}\\
			x+\text{\texthindi{इष}}\\
		\end{cases}$ &
		\texthindi{सन्}$(x)$\\\hline 
		\multirow{2}{*}{\texthindi{गुद्}}
		&\multirow{2}{*}{\texthindi{जु}}
		&\texthindi{गोदिष}
		&\texthindi{जुगोदिष}\\
		&
		&\texthindi{गुदिष}
		&\texthindi{जुगुदिष}\\

		\multirow{2}{*}{\texthindi{युत्}}
		&\multirow{2}{*}{\texthindi{यु}}
		&\texthindi{योतिष}
		&\texthindi{युयोतिष}\\
		&
		&\texthindi{युतिष}
		&\texthindi{युयुतिष}\\

		\multirow{2}{*}{\texthindi{विथ्}}
		&\multirow{2}{*}{\texthindi{वि}}
		&\texthindi{वेथिष}
		&\texthindi{विवेथिष}\\
		&
		&\texthindi{विथिष}
		&\texthindi{विविथिष}\\

		\multirow{2}{*}{\texthindi{चित्}}
		&\multirow{2}{*}{\texthindi{चि}}
		&\texthindi{चेतिष}
		&\texthindi{चिचेतिष}\\
		&
		&\texthindi{चितिष}
		&\texthindi{चिचितिष}\\
	\hline
	\end{tabular}
	\caption{Examples of Case II for multivalued functions of \texthindi{सन्} }
	\label{table:8.11}
\end{center}
\end{table}

\textbf{Case III:}\\

If there is only one $v$ in $x$, such that $x(2)=v=$ \texthindi{इ उ }\\
and starts with atleast two consonants i.e $x^{\prime}(1)=c$,and\\ $x^{\prime}(2)=c$\\
then\\ 
\begin{equation}
\text{\texthindi{सन्}}(x)= 
\begin{cases}
	c^{\prime}+v^{\prime}(1)+x[\text{\texthindi{इ उ }}\xrightarrow{v(2)}\text{\texthindi{ए ओ}}+p(x)+\text{\texthindi{ष}}\\
	c^{\prime}+v^{\prime}(1)+x+p(x)+\text{\texthindi{ष}}\\
\end{cases} 
\end{equation}

\begin{table}[h!]
\begin{center}
	\begin{tabular}{|c|c|c|c|} 
		\hline
		$x$&
		$c^{\prime}$ + $v^{\prime}(1)$&
		$\begin{cases}
			x[\text{\texthindi{इ उ}}\xrightarrow{v(2)}\text{\texthindi{ए ओ}}+p(x)+\text{\texthindi{ष}}\\
			x+p(x)+\text{\texthindi{ष}}\\
		\end{cases}$ &
		\texthindi{सन्}$(x)$\\\hline 
		\multirow{2}{*}{\texthindi{च्युत्}}
		&\multirow{2}{*}{\texthindi{चु}}
		&\texthindi{च्योतिष}
		&\texthindi{चुच्योतिष}\\
		&
		&\texthindi{च्युतिष}
		&\texthindi{चुच्युतिष}\\

		\multirow{2}{*}{\texthindi{श्च्युत्}}
		&\multirow{2}{*}{\texthindi{चु}}
		&\texthindi{श्च्योतिष}
		&\texthindi{चुश्च्योतिष}\\
		&
		&\texthindi{श्च्युतिष}
		&\texthindi{चुश्च्युतिष}\\

		\multirow{2}{*}{\texthindi{क्लिश्}}
		&\multirow{2}{*}{\texthindi{चि}}
		&\texthindi{क्लेशिष}
		&\texthindi{चिक्लेशिष}\\
		&
		&\texthindi{क्लिशिष}
		&\texthindi{चिक्लिशिष}\\

		\multirow{2}{*}{\texthindi{क्ष्विद्}}
		&\multirow{2}{*}{\texthindi{चि}}
		&\texthindi{क्ष्वेदिष}
		&\texthindi{चिक्ष्वेदिष}\\
		&
		&\texthindi{क्ष्विदिष}
		&\texthindi{चिक्ष्विदिष}\\

	\hline
	\end{tabular}
	\caption{Examples of Case III for multivalued functions of \texthindi{सन्} }
	\label{table:8.12}
\end{center}
\end{table}