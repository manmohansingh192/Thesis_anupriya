\chapter*{Conclusions}
\label{sec:8}
\lettrine[findent=2pt]{\textbf{O}}{}ur aim was to mathematically model the Sanskrit grammar by using mathematical functions. This approach of constructiong functions using the input and output data is the progressive way of writing this grammar. It also gives us an insight into the mind of grammarian Pāṇini, while he constructed the sūtras in Aṣṭādhyāyī which are oftentimes compared to equations. When we construct functions for various Pratayas in the form of cases by using the sets of inputs and outputs in the form of dhātus and verbs/prātipadikas respectively, we discover that the cases become much simpler when some of those Pāṇinian techniques are used. The grouping of dhātus done in the dhātupatha is also an elegant way for the grouping of dhātus. \\
Mathematical modelling of Pāṇinian grammar in this way helps identify some general patterns like, each of which is grouped separately as a case in the functions. These patterns are mainly dependent upon the occurance of certain specific syllables at certain places. We wrote the functions for \texthindi{णिच्, तुमुन्, यङ् }and \texthindi{सन् प्रत्यय}, where we got various Cases. These cases look at the input and compare it to the conditions mentioned in these cases. The input is accepted only when the case matches with the input. However, we observed that there are some dhātus hich even after fulfilling the conditions given in the cases, give an output which is different from what is observed in the literature. All such cases needed a separate approach. Hence the for the treatment of such cases input sets for those particular cases have been defined. \\
We also observed that in the literature there were certain inputs which do not give out a unique output. There are some input dhātus which have two or three output forms. To account for more than two forms of a word, Pāṇini uses optional form rules to state that alternate forms are also possible. For example, sūtra (rule) 1.2.3 vibhaṣorṇoḥ states that ‘After the verb ūrṇa  
'to cover', the affix beginning with the augment iṭ is regarded optionally like ṅit (Sutravali, 2020)’. We have used multivalued functions to denote such optional forms in our system of representing the pratyayas as functions. Thus, we wrote multivalued functions for the optional forms of various dhātus. In other words, multivalued function is a way to represent optional output forms which are expressed in Pāṇinian grammar with the help of 3 terms i.e. vā, vibhaṣā, and anyatarasyām.\\
Since theoretically the process of feeding the output into the input of one function can continue infinitely, we have demonstrated this by using the example of the  \texthindi{णिच्} and \texthindi{ तुमुन्} functions where the output of  \texthindi{णिच्}$(x)$ function is input into the \texthindi{तुमुन्}$(x)$ function.\\ 
Comparison between the techniques employed by Pāṇini and our notation of functions offers a better understanding of how Pāṇinian techniques ensure brevity and terseness. Pāṇini has used the pratyāhāra and various technical terms for brevity like upadhā, ṭi, ekāc, antya etc. \\

\begin{table}[h!]
\begin{center}
\begin{tabular}{ |c|c| } 
 \hline
 Functions & Pāṇinian techniques\\
 \hline
 $x(2)$ & \texthindi{ उपधा} \\
 \hline
 $c^{\prime}1,c^{\prime}2,…,v^{\prime}1$ if $x^{\prime}1=$consonant;& \multirow{2}{*}{\texthindi{एकाच्}}\\
$c^{\prime}1,c^{\prime}2,…,v^{\prime}2$ if $x^{\prime}1=$vowel & \\

\hline 
Multivalued functions  & \texthindi{वा, विभाषा,} and \texthindi{अन्यतरस्याम्}\\
\hline
 $x(1)$ & \texthindi{अन्त्य}\\
\hline
 - & \texthindi{अनुवृत्ति }\\
\hline

\end{tabular}
\caption{Examples of \texthindi{णिच्} function Case IX}
\label{table:1}
\end{center}
\end{table}
\begin{itemize}
    \item 

What we are essentially denoting as x(2) in our functions i.e. the penultimate term is nothing but \texthindi{उपधा.} In the functions x(2) plays a vital role especially when the change has to occur in the \texthindi{इ}, \texthindi{उ}, and \texthindi{ऋ} vowels present at x(2) in the input x. In the Aṣṭādhyāyī rule ‘1.1.65 \texthindi{अलोऽन्त्यात् पूर्व उपधा}’ gives the definition of \texthindi{उपधा }and it says that ‘the letter immediately preceding the last letter of a word is called penultimate (\texthindi{उपधा})’.
\item 	The term \texthindi{अच्} is used to denote all vowels and using the śivasūtras we can write then as\\
 \texthindi{अच् = अ इ उ ऋ ऌ ए ऐ ओ औ}\\
The following are the 14 śivasūtras given in the beginning of Aṣṭādhyāyī and the highlighted letters represent the construction of the technical term \texthindi{अच्} (using the first and the last syllables of the string) meaning all vowels.\\
 
1. \texthindi{अ इ उ ण्।}\\
2. \texthindi{ऋ ऌ क्।}\\
3. \texthindi{ए ओ ङ्।}\\
4. \texthindi{ऐ औ च्।}\\
5. \texthindi{ह य व र ट्।}\\
6. \texthindi{ल ण्।}\\
7. \texthindi{ञ म ङ ण न म्।}\\
8. \texthindi{झ भ ञ्।}\\
9. \texthindi{घ ढ ध ष्।}\\
10. \texthindi{ज ब ग ड द श्।}\\
11. \texthindi{ख फ छ ठ थ च ट त व्।}\\
12. \texthindi{क प य्।}\\
13. \texthindi{श ष स र्।}\\
14. \texthindi{ह ल्।}\\
 
The term \texthindi{एकाच }(ekāc) is composed of two parts \texthindi{एक} and \texthindi{अच्}, and thus represents a string containing only one vowel. This is represented in the functions as c’1,c’2,…,v’1 if x'1=consonant; or c’1,c’2,…,v’2 if x'1=vowel depending upon the case.
\item 	We come across various multivalued functions as Pāṇini has given the term ‘vā’ in the sūtras.
\item 	Pāṇini by convention treats x(1) as the end and calls it antya. This is clear from the definition of upadhā given by Pāṇini in Aṣṭādhyāyī sūtra ‘1.1.65 alontyāt pūrva upadhā’, which means ‘The letter immediately preceding the last letter of a word is called penultimate (upadhā) (\cite{creative})’. 
\item 	Anuvṛtti: Another important feature is anuvṛtti, which is a technique of carrying some parts of the previous sūtras to the next sūtras. Due to anuvṛtti, the order in which various elements appear in the sūtra itself are very important, but this technique need not be captured in these functions as all the functions are complete individually and the knowledge of one function is not a pre-requiite to make sense of the other.
\end{itemize}
Writing such functions for all other pratyaya functions may lead us towards a global function for pratyayas and for other grammatical tools as well. This technique of mathematical modelling is extremely helpful to understand Sanskrit grammar for people who are non-linguists or do not understand the technicalities of Sanskrit grammar. This mathematical model can also form a base for further processing of the grammatical rules for natural language processing of the language with the help of well-defined input and output sets.

\section{Limitations and Future Scope}
However, this approach of representing the word formation in the form of functions has some limitations of its own. For instance, it does not encompass the concept of meaning which is a pre-condition to the process of word formation. Meaning is the first step towards any word formation process and when we directly sart consutructing function from the input and output sets, the function is merely technically performing those mathematical functions on the syllables and it is not related to the meaning in any way. Moreover, accent is also an important aspect of the Sanskrit language which is not possible to capture in the current definition of functions. 
To encompass the concept of meaning and accent, some modifications to the current definition of functions has to be made by a researcher trying to work on these functions further. \\
Another drawback of this approach is that some of the inputs gave multiple outputs and hence could not fit into the definition of function which says that each input has a unique output of the form $y = f (x)$. Thus, for multivalued functions we are essentially going beyond the conventional definition of functions which are supposed to have a unique output for every input. Here we can see that we get two to three outputs for a single dhātu.\\
Some input-output pairs did not fit into the definitaion of certain cases even when they fulfilled the input conditions. Hence sets had to be defined for such dhātus. This led to deviation from the general cases and to accommodate for such cases we had to define the input sets. Further improvement in the construction of such cases can be done to include these special cases within the general cases which will in turn lead to the reduction in the number of cases.\\

An end goal of such mathematical modelling is the construction of a global function for example a global function for all pratyayas which will encapsulate all the sub-functions defining individual pratyayas.



