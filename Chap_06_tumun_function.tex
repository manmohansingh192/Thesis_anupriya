\section{Tumun Function}
\subsection{Definition}
\subsection{Cases for {\texthindi{तुमुन्}} function}
Type 1: x does not belong to the 10th gaṇa.\\
\textbf{Case I:}\\
If x $\not\in$  {{ \texthindi{दीधी वेवी दी (दीण् दिवादिर्गण) चिरि जिरि}}}\\ 
x(1) = \texthindi{इ/ई उ/ऊ ऋ/ॠ}, x(2) = c,\\ 
then,\\
\begin{equation}
	\text{\texthindi{ तुमुन् }}(x) = x[\text{\texthindi{इ/ई उ/ऊ ऋ/ॠ }}\xrightarrow{1} \text{\texthindi{ ए ओ अर् }}] + p(x) +  \text{\texthindi{ तुम् }}  
\end{equation} 

\begin{table}[h!]
	\begin{center}
		\begin{tabular}{ |c|c|c| } 
			\hline
			x&	  x[\texthindi{इ/ई उ/ऊ ऋ/ॠ } $\xrightarrow{1} $ \texthindi{ ए ओ अर् }] + p(x) &	\text{\texthindi{ तुमुन्}}(x) \\
			\hline
			\texthindi{ श्रि}&	\texthindi{ श्रयि}&	\texthindi{ श्रयितुम्}\\
			\hline
		\end{tabular}
		\caption{Examples of \texthindi{तुमुन्} function Case I}
		\label{table:6.12}
	\end{center}
\end{table}

Here we have used the ‘+’ rule \texthindi{ए}+\texthindi{इ}=\texthindi{अयि}
\\
\textbf{Case II:}\\
If x $\epsilon$ [\texthindi{दीधी वेवी चिरि जिरि}] 
then\\
\begin{equation}
	\text{\texthindi{तुमुन्}}(x) = x[\text{\texthindi{इ/ई}}  \xrightarrow{1} \phi] + p(x) + \text{\texthindi{तुम्}}\\
\end{equation}

\begin{table}[h!]
	\begin{center}
		\begin{tabular}{ |c|c|c| } 
			\hline
			x&	F(x) = x[\text{\texthindi{इ/ई}}  $\xrightarrow{1} \phi$] + p(x)&	\texthindi{तुमुन्}(x) \\
			\hline
			\texthindi{दीधी}&	\texthindi{दीधि}&	\texthindi{दीधितुम्}\\
			\texthindi{वेवी}&	\texthindi{वेवि}&	\texthindi{वेवितुम्}\\
			\hline
		\end{tabular}
		\caption{Examples of \texthindi{तुमुन्} function Case II}
		\label{table:6.13}
	\end{center}
\end{table}


\textbf{Case III:}\\ 
If x = \texthindi{दी} and belongs to 4th gaṇa (\texthindi{दिवादिर्गण})\\
then\\
\begin{equation}
	\text{\texthindi{तुमुन्}}(x) = x[\text{\texthindi{ई}}  \xrightarrow{1} \text{\texthindi{आ}}] + p(x) + \text{\texthindi{तुम्}}
\end{equation}

\begin{table}[h!]
	\begin{center}
		\begin{tabular}{ |c|c|c| } 
			\hline
			x&	x[\text{\texthindi{ई}} $ \xrightarrow{1}$ \text{\texthindi{ आ}}] + p(x)&	\texthindi{तुमुन्} (x)  \\
			\hline
			\texthindi{ दी }&	\texthindi{ दा }&	\texthindi{ दातुम् }\\
			\hline
		\end{tabular}
		\caption{Examples of \texthindi{तुमुन्} function Case III}
		\label{table:6.14}
	\end{center}
\end{table}

\textbf{Case IV:}\\
If x $\not\in$ {\texthindi{कुट् पुट् गुज् गुढ् ढिप् चुर् स्फुट् मुट् त्रुट् तुट् चुट् लुट् कृढ् कुढ् पुढ् घुट् तुढ् थुढ् स्फुढ् स्फुर् स्फुल् स्फुढ् स्फुर् स्फुल् कुढ् बुढ् क्रुढ् भुढ् गुर् व्रीढ् दृश् सृज् तिज् गुप् विज् तिज्} (1 gaṇa \texthindi{भ्वादिर्गण}) \texthindi{गुप्} (10 gaṇa) \texthindi{गुप्} (1 gaṇa \texthindi{भ्वादिर्गण गुपू रक्षणे}) \texthindi{गुप्} (1 gaṇa \texthindi{भ्वादिर्गण गोप गोपने}) \texthindi{लिह् मिह्}}\\
x(1) = c, x(2) = \texthindi{इ उ ऋ},\\ 
then\\

\begin{equation}
	\text{\texthindi{ तुमुन् }}(x) = x[\text{\texthindi{ \texthindi{इ} उ ऋ }}\xrightarrow{2} \text{\texthindi{ ए ओ अर् }}]+ p(x) + \text{\texthindi{ तुम् }}  
\end{equation}


\begin{table}[h!]
	\begin{center}
		\begin{tabular}{ |c|c|c| } 
			\hline
			x&	 x[\text{\texthindi{ \texthindi{इ} उ ऋ }}$\xrightarrow{2}$ \text{\texthindi{ ए ओ अर् }}]+ p(x)&	\texthindi{तुमुन्}(x) \\
			\hline
			\texthindi{ विथ्}&	\texthindi{ वेथि}&	\texthindi{ वेथितुम्}\\
			\texthindi{ मुद्}&	\texthindi{ मोदि}&	\texthindi{ मोदितुम्}\\
			\texthindi{ वृत्}&	\texthindi{ वर्ति}&	\texthindi{ वर्तितुम्} \\ 
			\hline
		\end{tabular}
		\caption{Examples of \texthindi{तुमुन्} function Case IV}
		\label{table:6.15}
	\end{center}
\end{table}

\textbf{Case V:}\\
If x $\epsilon$ [\texthindi{कुट् पुट् गुज् गुढ् ढिप् चुर् स्फुट् मुट् त्रुट् तुट् चुट् लुट् कृढ् कुढ् पुढ् घुट् तुढ् थुढ् स्फुढ् स्फुर् स्फुल् स्फुढ् स्फुर् स्फुल् कुढ् बुढ् क्रुढ् भुढ् गुर भृढ् गुर् व्रीढ् कु गु ध्रु}]\\
or if x(1)= \texthindi{अ/आ}/c; x(2)= c/v $\neq$ \texthindi{इ/उ/ऋ}; \\
then\\
\begin{equation}
	\text{\texthindi{ तुमुन् }}(x) = x+ p(x) + \text{\texthindi{ तुम् }}  
\end{equation}

\begin{table}[h!]
	\begin{center}
		\begin{tabular}{ |c|c|c| } 
			\hline
			x&	x+ p(x)&	\texthindi{तुमुन्}(x) \\
			\hline
			\texthindi{ कुट्}&	\texthindi{कुटि}&	\texthindi{ कुटितुम्}\\
			\hline
		\end{tabular}
		\caption{Examples of \texthindi{तुमुन्} function Case V}
		\label{table:6.16}
	\end{center}
\end{table}

\textbf{Case VI:}\\
If x $\epsilon$ {\texthindi{लिह् मिह्}}\\
then \\
\begin{equation}
	\text{\texthindi{ तुमुन् }}(x) = x[\text{\texthindi{ इ }}\xrightarrow{2} \text{\texthindi{ ए }}; [\text{\texthindi{ ह् }}\xrightarrow{1} \text{\texthindi{ ढ् }}]+ \text{\texthindi{ उम् }}  
\end{equation}


\begin{table}[h!]
	\begin{center}
		\begin{tabular}{ |c|c|c| } 
			\hline
			x&	 x[\text{\texthindi{ इ }} $\xrightarrow{2}$ \text{\texthindi{ ए }}; \text{\texthindi{ ह् }} $\xrightarrow{1}$ \text{\texthindi{ ढ् }}]&	\texthindi{तुमुन्}(x) \\
			\hline 
			\texthindi{लिह्}&	\texthindi{लेढ्}&	\texthindi{लेढुम्} \\ 
			\hline
		\end{tabular}
		\caption{Examples of \texthindi{तुमुन्} function Case VI}
		\label{table:6.17}
	\end{center}
\end{table}

\textbf{Case VII:}\\
If x $\epsilon$ [\texthindi{दृश् सृज्}]\\
then\\ 
\begin{equation}
	\text{\texthindi{ तुमुन् }}(x) = x[\text{\texthindi{ ऋ }}\xrightarrow{2} \text{\texthindi{ र् }}]+ \text{\texthindi{ तुम् }}  
\end{equation}

\begin{table}[h!]
	\begin{center}
		\begin{tabular}{ |c|c|c| } 
			\hline
			x&	x[\text{\texthindi{ ऋ }}$\xrightarrow{2}$ \text{\texthindi{ र् }}]  &	\texthindi{ तुमुन्}(x)  \\
			\hline
			\texthindi{ दृश्}&	\texthindi{ द्रश्}&	\texthindi{ द्रष्टुम्}\\
			\texthindi{ सृज्}&	\texthindi{ स्रज्}&	\texthindi{ स्रष्टुम्} \\ 
			\hline
		\end{tabular}
		\caption{Examples of \texthindi{तुमुन्} function Case VII}
		\label{table:6.18}
	\end{center}
\end{table}

\textbf{Case VIII:}\\
If x $\not\in$ [\texthindi{रुह् वह् बध् दरिद्रा}]\\
x(1) $\neq$ \texthindi{इ/ई उ/ऊ ऋ/ॠ};  x(2) $\neq$ \texthindi{इ उ ऋ},\\
then\\ 
\begin{equation}
	\text{\texthindi{ तुमुन् }}(x) = x + p(x) + \text{\texthindi{ तुम्} }
\end{equation}

\begin{table}[h!]
	\begin{center}
		\begin{tabular}{ |c|c|c| } 
			\hline
			x&	x+p(x)&	\texthindi{ तुमुन्(x) }\\
			\hline
			\texthindi{ तन्क् }&	\texthindi{ तन्कि }&	\texthindi{ तन्कितुम् } \\ 
			\hline
		\end{tabular}
		\caption{Examples of \texthindi{तुमुन्} function Case VIII}
		\label{table:6.19}
	\end{center}
\end{table}

\textbf{Case IX:}\\
If x $\epsilon$ [\texthindi{रुह् वह्}]\\
then\\
\begin{equation}
	\text{\texthindi{ तुमुन् }}(x) = x[\text{\texthindi{ उ अ }}\xrightarrow{2} \text{\texthindi{ ओ }}]+p(x)+ \text{\texthindi{ ह् }}\xrightarrow{1} \phi + \text{\texthindi{ तुम् }}  
\end{equation}

\begin{table}[h!]
	\begin{center}
		\begin{tabular}{ |c|c|c| } 
			\hline
			x&	x[\texthindi{ उ अ } $\xrightarrow{2}$ \texthindi{ ओ }]+p(x)+ \texthindi{ ह् }$\xrightarrow{1}$ $\phi$&	\texthindi{ तुमुन्}(x) \\
			\hline
			\texthindi{ रुह् }&	\texthindi{रो}&	\texthindi{ रोढुम्}\\
			\texthindi{ वह् }&	\texthindi{वो }&	\texthindi{ वोढुम् }\\
			\hline
		\end{tabular}
		\caption{Examples of \texthindi{तुमुन्} function Case IX}
		\label{table:6.20}
	\end{center}
\end{table}

\textbf{Case X:}\\
If x = \texthindi{दरिद्रा}\\
then\\
\begin{equation}
	\text{\texthindi{ तुमुन् }}(x) = x[\text{\texthindi{ आ }}\xrightarrow{1} \phi]+p(x)+  \text{\texthindi{ तुम् }}  
\end{equation} 

\begin{table}[h!]
	\begin{center}
		\begin{tabular}{ |c|c|c| } 
			\hline
			x&	x[\text{\texthindi{आ}}  $\xrightarrow{1}$ $\phi$] + p(x)&	\texthindi{तुमुन्}(x)\\
			\hline
			\texthindi{ दरिद्रा }&	\texthindi{ दरिद्रि }&	\texthindi{ दरिद्रितुम् }\\
			\hline
		\end{tabular}
		\caption{Examples of \texthindi{तुमुन्} function Case X}
		\label{table:6.21}
	\end{center}
\end{table}

\textbf{Case XI:}\\
x(2)= \texthindi{र् व्}, x(3)= \texthindi{उ}\\
then\\
\begin{equation}
	\text{\texthindi{ तुमुन् }}(x) = x[\text{\texthindi{ उ }}\xrightarrow{3} \text{\texthindi{ ऊ }}]+p(x)+  \text{\texthindi{ तुम् }}  
\end{equation}

\begin{table}[h!]
	\begin{center}
		\begin{tabular}{ |c|c|c| } 
			\hline
			x &	x[\text{\texthindi{ उ }}$\xrightarrow{3}$ \text{\texthindi{ ऊ }}]+p(x)&	\texthindi{तुमुन्}(x)\\
			\hline
			\texthindi{उर्द}&	\texthindi{ऊर्दि}&	\texthindi{ऊर्दितुम्}\\
			\texthindi{उर्व}&	\texthindi{ऊर्वि}&	\texthindi{ऊर्वितुम्} \\ 
			\hline
		\end{tabular}
		\caption{Examples of \texthindi{तुमुन्} function Case XI}
		\label{table:6.22}
	\end{center}
\end{table}




\textbf{Type 2: x belongs to the 10th gaṇa.}\\
\texthindi{तुम्} = \texthindi{तुम्(णिच्}(x))\\
y = \texthindi{णिच्}(x)\\
y(1) = \texthindi{इ}\\
\begin{equation}
	\text{\texthindi{ तुमुन्(णिच् }}(x)) = \text{\texthindi{ तुम्}} y  
\end{equation}
then\\
\begin{equation}
	\text{\texthindi{ तुमुन्}}(y) = y[\texthindi{इ}\xrightarrow{1}\texthindi{ए}] + p(x) + \text{\texthindi{ तुम्}}  
\end{equation}
Note: - The output of function \texthindi{णिच्}(x) is treated as input for \texthindi{तुमुन्}(x) and the same Cases as Type 1 can be used.\\
Examples:\\

\begin{table}[h!]
	\begin{center}
		\begin{tabular}{ |c|c|c|c|c|}
			\hline
			Case &	x&	y = \texthindi{णिच्}(x)&	y[\texthindi{इ}  $\xrightarrow{1}$ \texthindi{ए}] + p(x) &	\texthindi{तुम्(णिच्}(x))\\
			\hline
			I &	\texthindi{स्पर्ध्}&	\texthindi{स्पर्धि}&	\texthindi{स्पर्धयि}&	\texthindi{स्पर्धयितुम्}\\
			I &	\texthindi{दध्}&	\texthindi{दधि}&	\texthindi{दधयि}&	\texthindi{दधयितुम्}\\
			I &	\texthindi{नाथ्}&	\texthindi{नाथि}&	\texthindi{नाथयि}&	\texthindi{नाथयितुम्}\\
			II &	\texthindi{दद्}&	\texthindi{दादि}&	\texthindi{दादयि}&	\texthindi{दादयितुम्}\\
			III &	\texthindi{पिट्}&	\texthindi{पेटि}&	\texthindi{पेटयि}&	\texthindi{पेटयितुम्}\\
			III &	\texthindi{मुद्}&	\texthindi{मोदि}&	\texthindi{मोदयि}&	\texthindi{मोदयितुम्}\\
			IV &	\texthindi{नी}&	\texthindi{नायि}&	\texthindi{नाययि}&	\texthindi{नाययितुम्}\\
			\hline
		\end{tabular}
		\caption{Examples of \texthindi{तुमुन्} function if x belongs to the 10th gaṇa}
		\label{table:6.23}
	\end{center}
\end{table}

\textbf{Multivalued functions:}\\ 
In some Cases, there are multiple words from a single dhātu. To account for such Cases, the concept of multivaled functions is used where a single input to the \texthindi{तुमुन्} (x) function generates multiple outputs.\\
\textbf{Case I:}\\
If x $\epsilon$ \texthindi{सिध्(षिधू)} 
then\\
\begin{equation}
	\text{\texthindi{तुमुन् }} (x) = 
	\begin{cases}
		x [\text{\texthindi{ इ/ई उ/ऊ ऋ/ॠ }}\xrightarrow{2}\text{\texthindi{ ए ओ अर् }}]+ \text{\texthindi{इ}} + \text{\texthindi{तुम्}}\\
		x [\text{\texthindi{ इ/ई उ/ऊ ऋ/ॠ }}\xrightarrow{2} \text{\texthindi{ ए ओ अर् }}]+ \phi + \text{\texthindi{तुम्}}\\
	\end{cases}
\end{equation}

\begin{table}[h!]
	\begin{center}
		\begin{tabular}{ |c|c|c| } 
			\hline
			x &
			$
				\begin{cases}
							x [\text{\texthindi{ इ/ई उ/ऊ ऋ/ॠ }}\xrightarrow{2}\text{\texthindi{ ए ओ अर् }}]+ \text{\texthindi{इ}} + \text{\texthindi{तुम्}}\\
							x [\text{\texthindi{ इ/ई उ/ऊ ऋ/ॠ }}\xrightarrow{2} \text{\texthindi{ ए ओ अर् }}]+ \phi + \text{\texthindi{तुम्}}\\
				\end{cases}
			$ &
			\texthindi{तुमुन्}(x)\\
			\hline
			\multirow{2}{*}{\texthindi{सिध्}}&	
			\text{\texthindi{सेधि }}&
			\text{\texthindi{सेधितुम्}}\\ 
			&
			\text{\texthindi{सेध्}}&
			\text{\texthindi{सेद्धुम्}}\\
			\hline
		\end{tabular}
		\caption{Examples of Case I for multivalued functions of \texthindi{तुमुन्} }
		\label{table:6.24}
	\end{center}
\end{table}

\textbf{Case II:}\\
If x $\epsilon$ [\texthindi{स्वृ सू (षूङ् अदादिर्गणः) धू}]\\
then\\
\begin{equation}
	\text{\texthindi{तुमुन्}} (x) = 
	\begin{cases}
		x[\text{\texthindi{ इ/ई उ/ऊ ऋ/ॠ }}\xrightarrow{1}\text{\texthindi{ ए ओ अर् }}]+ \text{\texthindi{इ}} + \text{\texthindi{तुम्}}\\
		x[\text{\texthindi{ इ/ई उ/ऊ ऋ/ॠ }}\xrightarrow{1} \text{\texthindi{ ए ओ अर् }}]+ \phi + \text{\texthindi{तुम्}}\\
	\end{cases}
\end{equation}

\begin{table}[h!]
	\begin{center}
		\begin{tabular}{ |c|c|c| } 
			\hline
			x & 
			$
				\begin{cases}
					x [\text{\texthindi{ इ/ई उ/ऊ ऋ/ॠ }} \xrightarrow{1} \text{\texthindi{ ए ओ अर् }}]+ \text{\texthindi{इ}} \\
					x [\text{\texthindi{ इ/ई उ/ऊ ऋ/ॠ }} \xrightarrow{1} \text{\texthindi{ ए ओ अर् }}]+ \phi \\
				\end{cases}
			$
			& \texthindi{तुमुन्}(x)\\
			\hline
			\multirow{2}{*}{\texthindi {सू }}&	
			\texthindi{ सवि}&
			\texthindi{सवितुम्}\\ 
			& \texthindi{सो}&
			\texthindi{सोतुम्}\\
			\hline
			\multirow{2}{*}{\texthindi { स्वृ }}&
			\texthindi{ स्वरि }&
			\texthindi {स्वरितुम् }\\ 
			&
			\texthindi{ स्वर् }&
			\texthindi {स्वर्तुम् }\\
			\hline
			
		\end{tabular}
		\caption{Examples of Case II for multivalued functions of \texthindi{तुमुन्} }
		\label{table:6.25}
	\end{center}
\end{table}

\textbf{Case III:}\\
If x $\epsilon$ \texthindi{पॄ(3 जुहोत्यादिर्गणः) जॄ झॄ वृ कॄ गॄ स्तॄ कॄ वॄ शॄ पॄ भॄ मॄ दृ जॄ नृ कृ ऋ ॠ (तुदादिगण) गृ})\\ 
then\\

\begin{equation}
	\text{\texthindi{तुमुन्}} (x) = 
	\begin{cases}
		x [\text{\texthindi{ इ/ई उ/ऊ ऋ/ॠ }} \xrightarrow{1} \text{\texthindi{ ए ओ अर् }} ]+  \text{\texthindi{इ}}  +  \texthindi{तुम्}\\
		x [\texthindi{ इ/ई उ/ऊ ऋ/ॠ } \xrightarrow{1} \texthindi{ ए ओ अर् } ]+  \texthindi{ई}  +  \texthindi{तुम्}\\
	\end{cases}
\end{equation}

\begin{table}[h!]
	\begin{center}
		\begin{tabular}{ |c|c|c| } 
			\hline
			x & 
			$ \begin{cases}
				x [\text{\texthindi{ इ/ई उ/ऊ ऋ/ॠ }}\xrightarrow{1} \text{\texthindi{ ए ओ अर् }}+ \text{\texthindi{इ}}\\
				x [\text{\texthindi{ इ/ई उ/ऊ ऋ/ॠ}}\xrightarrow{1} \text{\texthindi{ ए ओ अर् }}]+ \text{\texthindi{ई}}\\
			\end{cases} $
			& \texthindi{तुमुन्}(x)\\
			\hline
			\multirow{2}{*}{\texthindi{वृ}}
			&\texthindi{वरि}
			&\texthindi{वरितुम्}\\ 
			&\texthindi{वरी}
			&\texthindi{वरीतुम्}\\
			\hline
			\multirow{2}{*}{\texthindi{कॄ}}
			&\texthindi{करि }
			&\texthindi{करितुम् }\\
			&\texthindi{करी}
			&\texthindi{करीतुम्}\\
			\hline
		\end{tabular}
		\caption{Examples of Case III for multivalued functions of \texthindi{तुमुन्} }
		\label{table:6.26}
	\end{center}
\end{table}\

\textbf{Case IV:}\\
If x $\epsilon$ [\texthindi{सृप् कृष्}]\\
then\\

\begin{equation}
	\text{\texthindi{तुमुन्}} (x) = 
	\begin{cases}
		x[\text{\texthindi{ ऋ/ॠ }}\xrightarrow{2}\text{\texthindi{ अर् }}] + \text{\texthindi{इ}} + \text{\texthindi{तुम्}}\\
		x[\text{\texthindi{ ऋ/ॠ }}\xrightarrow{2}\text{\texthindi{ र् }}] + \phi + \text{\texthindi{तुम्}}\\
	\end{cases}
\end{equation}

\begin{table}[h!]
	\begin{center}
		\begin{tabular}{ |c|c|c| } 
			\hline
			x &
			$ \begin{cases}
				x[\text{\texthindi{ ऋ/ॠ }}\xrightarrow{2}\text{\texthindi{ अर् }}] + \text{\texthindi{इ्}}\\
				x[\text{\texthindi{ ऋ/ॠ }}\xrightarrow{2}\text{\texthindi{ र् }}] + \phi \\
			\end{cases} $
			& \texthindi{तुमुन्}(x)\\
			\hline
			\multirow{2}{*}{\texthindi{सृप्}}
			&\texthindi{सर्पि}
			&\texthindi{सर्पितुम् }\\
			&\texthindi{स्रप्}
			&\texthindi{स्रप्तुम्}\\
			\multirow{2}{*}{\texthindi{कृष्}}
			&\texthindi{कर्षि }
			&\texthindi{कर्षितुम् }\\
			&\texthindi{क्रष्}
			&\texthindi{क्रष्टुम्}\\
			\hline
		\end{tabular}
		\caption{Examples of Case IV for multivalued functions of \texthindi{तुमुन्} }
		\label{table:6.27}
	\end{center}
\end{table}

\textbf{Case V:}\\
If x =\texthindi{ऊर्णु}\\
then\\

\begin{equation}
	\texthindi{तुमुन्} (x)  = 	
	\begin{cases}
		x[\text{\texthindi{उ}}\xrightarrow{1}\text{\texthindi{ओ}}\xrightarrow{1}\text{\texthindi{अव्}}] + \text{\texthindi{इ}} + \text{\texthindi{तुम्}}\\
		x[\text{\texthindi{उ}}\xrightarrow{1}\text{\texthindi{ अव्}}] + \text{\texthindi{इ}}+ \text{\texthindi{तुम्}}\\
	\end{cases}
\end{equation}

\begin{table}[h!]
	\begin{center}
		\begin{tabular}{ |c|c|c| } 
			\hline
			x & 
			$ \begin{cases}
				x[\text{\texthindi{उ}}\xrightarrow{1}\text{\texthindi{ओ}}\xrightarrow{1}\text{\texthindi{अव्}}] + \text{\texthindi{इ}}\\
				x[\text{\texthindi{उ}}\xrightarrow{1}\text{\texthindi{ अव्}}] + \text{\texthindi{इ}}\\
			\end{cases} $ & \texthindi{तुमुन्}(x)\\
			\hline
			\multirow{2}{*}{\texthindi{ऊर्णु}}
			&\texthindi{ऊर्णवि}
			&\texthindi{ऊर्णवितुम् }\\
			&\texthindi{ऊर्णुवि}
			&\texthindi{ऊर्णुवितुम्}\\

		\hline
		\end{tabular}
		\caption{Examples of Case V for multivalued functions of \texthindi{तुमुन्} }
		\label{table:6.28}
	\end{center}
\end{table}

\textbf{Case VI:}\\
If x $\epsilon$ \texthindi{ली}\\
then\\

\begin{equation}
	\texthindi{तुमुन्} (x)  = 
	\begin{cases}
		x[\text{\texthindi{ई}}\xrightarrow{1}\text{\texthindi{ए}}]+ \text{\texthindi{तुम्}}\\
		x[\text{\texthindi{ई}}\xrightarrow{1}\text{\texthindi{आ}}]+ \text{\texthindi{तुम्}}\\
	\end{cases}
\end{equation}

\begin{table}[h!]
	\begin{center}
		\begin{tabular}{ |c|c|c| } 
			\hline
			x & $\begin{cases}
				x[\text{\texthindi{ई}}\xrightarrow{1}\text{\texthindi{ए}}]\\
				x[\text{\texthindi{ई}}\xrightarrow{1}\text{\texthindi{आ}}]\\
			\end{cases}$ & \texthindi{तुमुन्}(x)\\
			\hline
			\multirow{2}{*}{\texthindi{ली}}
			&\texthindi{ले}
			&\texthindi{लेतुम्}\\ 
			&\texthindi{ला}
			&\texthindi{लातुम्}\\
			\hline
		\end{tabular}
		\caption{Examples of Case VI for multivalued functions of \texthindi{तुमुन्} }
		\label{table:6.29}
	\end{center}
\end{table}

\textbf{Case VII:}\\
If x $\epsilon$ \texthindi{गुप्} (1 gaṇa \texthindi{भ्वादिर्गणः (गुपू रक्षणे)}\\
then\\
\begin{equation}
	\texthindi{तुमुन्} (x) = 	
	\begin{cases}
		x[\text{\texthindi{ इ/ई उ/ऊ ऋ/ॠ}}\xrightarrow{2}\text{\texthindi{ ए ओ अर्}}]+ \text{\texthindi{इ}} + \text{\texthindi{तुम्}}\\
		x[\text{\texthindi{ इ/ई उ/ऊ ऋ/ॠ}}\xrightarrow{2}\text{\texthindi{ ए ओ अर्}}]+ \phi + \text{\texthindi{तुम्}}\\
		x[\text{\texthindi{ इ/ई उ/ऊ ऋ/ॠ}}\xrightarrow{2}\text{\texthindi{ ए ओ अर्}}]+ \text{\texthindi{इ}} + \text{\texthindi{आय्}}+ \text{\texthindi{तुम्}}\\
	\end{cases}
\end{equation}

\begin{table}[h!]
	\begin{center}
		\begin{tabular}{ |c|c|c| } 
			\hline
			x & 
			$\begin{cases}
				x[\texthindi{ इ/ई उ/ऊ ऋ/ॠ}\xrightarrow{2}\texthindi{ ए ओ अर्}]+ \texthindi{इ}\\
				x[\texthindi{ इ/ई उ/ऊ ऋ/ॠ}\xrightarrow{2}\texthindi{ ए ओ अर्}]+ \phi \\
				x[\texthindi{ इ/ई उ/ऊ ऋ/ॠ}\xrightarrow{2}\texthindi{ ए ओ अर्}]+ \texthindi{इ} + \texthindi{आय्}\\
			\end{cases}$ 
			&\multirow{4}{*}{\texthindi{तुमुन्}(x)}\\
			\hline
			\multirow{3}{*}{\texthindi{गुप्}}
			&\texthindi{गोपि }
			&\texthindi{गोपितुम्}\\ 
			&\texthindi{गोप्}
			&\texthindi{गोप्तुम् }\\
			&\texthindi{गोपायि}
			&\texthindi{गोपायितुम्}\\
			\hline
		\end{tabular}
		\caption{Examples of Case VII for multivalued functions of \texthindi{तुमुन्} }
		\label{table:6.30}
	\end{center}
\end{table}

\textbf{Case VIII:}\\
If x $\epsilon$ \texthindi{गुप्} (1 gaṇa \texthindi{भ्वादिर्गणः गु॒प गोपने})\\

then\\
\begin{equation}
	\texthindi{तुमुन्}(x) = 	\begin{cases}
		x[\text{\texthindi{उ}}\xrightarrow{2}\text{\texthindi{ओ}}]+\text{\texthindi{अय्}}+ \text{\texthindi{इ}} + \text{\texthindi{तुम्}}\\
		\text{\texthindi{जु}} + x + \text{\texthindi{स्}} + \text{\texthindi{इ}} + \text{\texthindi{तुम्}}\\
	\end{cases}
\end{equation}


\begin{table}[h!]
	\begin{center}
		\begin{tabular}{ |c|c|c| } 
			\hline
			x &
			$\begin{cases}
				x[\text{\texthindi{उ}}\xrightarrow{2}\text{\texthindi{ओ}}]+\text{\texthindi{अय्}}\\
				\text{\texthindi{जु}} + x + \text{\texthindi{स्}}\\
			\end{cases}$ & \texthindi{तुमुन्}(x)\\
			\hline
			\multirow{2}{*}{\texthindi{गुप्}}
			&\texthindi{गोपय्}
			&\texthindi{गोपयितुम्}\\ 
			&\texthindi{जुगुप्स्}
			&\texthindi{जुगुप्सितुम्}\\
			\hline
		\end{tabular}
		\caption{Examples of Case VIII for multivalued functions of \texthindi{तुमुन्} }
		\label{table:6.31}
	\end{center}
\end{table}

\textbf{Case IX:}\\
If x $\epsilon$ [\texthindi{रुष् रिष् लुभ् क्लिद् लुभ् इष्}]\\
then\\
\begin{equation}
	\texthindi{तुमुन्}(x) = 	\begin{cases}
		x[\text{\texthindi{ इ\/ई उ\/ऊ ऋ\/ॠ}}\xrightarrow{2}\text{\texthindi{ ए ओ अर्}}]+ \text{\texthindi{इ}} + \text{\texthindi{तुम्}}\\
		x[\text{\texthindi{ इ\/ई उ\/ऊ ऋ\/ॠ}}\xrightarrow{2}\text{\texthindi{ ए ओ अर्}}]+ \phi + \text{\texthindi{तुम्}}\\
	\end{cases}
\end{equation}

\begin{table}[h!]
	\begin{center}
		\begin{tabular}{ |c|c|c| } 
			\hline
			x & 
			\texthindi{तुमुन्}(x) =
			$\begin{cases}
				x[\texthindi{ इ\/ई उ\/ऊ ऋ\/ॠ}\xrightarrow{2}\texthindi{ ए ओ अर्}]+ \texthindi{इ} \\
				x[\texthindi{ इ\/ई उ\/ऊ ऋ\/ॠ}\xrightarrow{2}\texthindi{ ए ओ अर्}]+ \phi \\
			\end{cases}$ & 
			\texthindi{तुमुन्}(x) \\ 
			\hline
			\multirow{2}{*}{\texthindi{रिष्}}
			&\texthindi{रेषि}
			&\texthindi{रेषितुम्}\\
			&\texthindi{रेष्}
			&\texthindi{रेष्टुम्}\\
			\multirow{2}{*}{\texthindi{लुभ्}}
			&\texthindi{लोभि}
			&\texthindi{लोभितुम्}\\
			&\texthindi{लोभ्}
			&\texthindi{लोब्धुम्}\\
			\hline
		\end{tabular}
		\caption{Examples of Case IX for multivalued functions of \texthindi{तुमुन्} }
		\label{table:6.32}
	\end{center}
\end{table}

\textbf{Case X:}\\
If x $\epsilon$ \texthindi{गुह्} (1 gaṇa \texthindi{भ्वादिर्गणः) गृह्} (1 gaṇa \texthindi{भ्वादिर्गणः)}\\
then\\
\begin{equation}
	\texthindi{तुमुन्}(x) = 	\begin{cases}
		x[\text{\texthindi{ उ ऋ}}\xrightarrow{2}\text{\texthindi{ ऊ अर्}}]+ \text{\texthindi{इ}} + \text{\texthindi{तुम्}}\\
		x[\text{\texthindi{ उ ऋ}}\xrightarrow{2}\text{\texthindi{ ओ अर्}}]]+ \text{\texthindi{ह्}}\rightarrow\phi + \text{\texthindi{ढुम्}}\\
	\end{cases}
\end{equation}

\begin{table}[h!]
	\begin{center}
		\begin{tabular}{ |c|c|c| } 
			\hline
			x & 
			\texthindi{तुमुन्}(x) = 	
			$\begin{cases}
				x[\text{\texthindi{ उ ऋ}}\xrightarrow{2}\text{\texthindi{ ऊ अर्}}]+ \text{\texthindi{इ}}\\
				x[\text{\texthindi{ उ ऋ}}\xrightarrow{2}\text{\texthindi{ ओ अर्}}]]+ \text{\texthindi{ह्}}\rightarrow\phi \\
			\end{cases}$ & 
			\texthindi{तुमुन्}(x) \\ 
			\hline
			\multirow{2}{*}{\texthindi{ गुह्}}
			&\texthindi{ गूहि}
			&\texthindi{ गूहितुम्}\\
			&\texthindi{ गो}
			&\texthindi{ गोढुम्}\\
			\multirow{2}{*}{\texthindi{ गृह्}}
			&\texthindi{ गर्हि}
			&\texthindi{ गर्हितुम्}\\ 
			&\texthindi{ गर्}
			&\texthindi{ गर्ढुम्}\\
			\hline
		\end{tabular}
		\caption{Examples of Case X for multivalued functions of \texthindi{तुमुन्} }
		\label{table:6.33}
	\end{center}
\end{table}


\textbf{Case XI:}\\
If x $\epsilon$ \texthindi{तिज्} (1 gaṇa \texthindi{भ्वादिर्गणः})
then\\
\begin{equation}
	\texthindi{तुमुन्}(x) = 	\begin{cases}
		x[\text{\texthindi{इ}}\xrightarrow{2}\text{\texthindi{ए}}]+\text{\texthindi{अय्}}+ \text{\texthindi{इ}} + \text{\texthindi{तुम्}}\\
		\text{\texthindi{ति}} + x[\text{\texthindi{ज्}}\xrightarrow{1}\text{\texthindi{क्ष्}}] +  \text{\texthindi{इ}} + \text{\texthindi{तुम्}}\\
	\end{cases}
\end{equation}


\begin{table}[h!]
	\begin{center}
		\begin{tabular}{ |c|c|c| } 
			\hline
			x & 
			\texthindi{तुमुन्}(x) = 	
			$\begin{cases}
				x[\text{\texthindi{इ}}\xrightarrow{2}\text{\texthindi{ए}}]+\text{\texthindi{अय्}}\\
				\text{\texthindi{ति}} + x[\text{\texthindi{ज्}}\xrightarrow{1}\text{\texthindi{क्ष्}}]\\
			\end{cases}$ & 
			\texthindi{तुमुन्}(x) \\ 
			\hline
			\multirow{2}{*}{\texthindi{तिज्}}
			&\texthindi{तेजय्}
			&\texthindi{तेजयितुम्}\\
			&\texthindi{तितिक्ष्}
			&\texthindi{तितिक्षितुम्}\\
			\hline
		\end{tabular}
		\caption{Examples of Case XI for multivalued functions of \texthindi{तुमुन्} }
		\label{table:6.34}
	\end{center}
\end{table}

\textbf{Case XII:}\\
If x $\epsilon$ \texthindi{कित्} (1 gaṇa \texthindi{भ्वादिर्गणः})\\
then\\
\begin{equation}
	\texthindi{तुमुन्}(x) = 	
	\begin{cases}
		x[\text{\texthindi{इ}}\xrightarrow{2}\text{\texthindi{ए}}]+\text{\texthindi{अय्}}+ \text{\texthindi{इ}} + \text{\texthindi{तुम्}}\\
		\text{\texthindi{चि}} + x + \text{\texthindi{स्}} +  \text{\texthindi{इ}} + \text{\texthindi{तुम्}}\\
	\end{cases}
\end{equation}

\begin{table}[h!]
	\begin{center}
		\begin{tabular}{ |c|c|c| } 
			\hline
			x & 
			\texthindi{तुमुन्}(x) = 	
			$\begin{cases}
				x[\text{\texthindi{इ}}\xrightarrow{2}\text{\texthindi{ए}}]+\text{\texthindi{अय्}}\\
				\text{\texthindi{चि}} + x + \text{\texthindi{स्}}\\
			\end{cases}$ & \texthindi{तुमुन्}(x) \\ 
			\hline
			\multirow{2}{*}{\texthindi{कित्}}
			&\texthindi{केतय्}
			&\texthindi{केतयितुम्}\\ 
			&\texthindi{चिकित्स्}
			&\texthindi{चिकित्सितुम्}\\
			\hline
		\end{tabular}
		\caption{Examples of Case XII for multivalued functions of \texthindi{तुमुन्} }
		\label{table:6.35}
	\end{center}
\end{table}

\textbf{Case XIII:}\\
If x $\epsilon$ \texthindi{तृप् दृप् (दृप हर्षमोहनयोः (दिवादिर्गणः))}\\
then\\

\begin{equation}
	\texthindi{तुमुन्}(x) = 
	\begin{cases}
		x[\text{\texthindi{ ॠ}}\xrightarrow{2}\text{\texthindi{ अर्}}]+ \phi + \text{\texthindi{तुम्}}\\
		x[\text{\texthindi{ ॠ}}\xrightarrow{2}\text{\texthindi{ र्}}]+ \phi + \text{\texthindi{तुम्}}\\
		x[\text{\texthindi{ ॠ}}\xrightarrow{2}\text{\texthindi{ अर्}}]+ \text{\texthindi{इ}} + \text{\texthindi{तुम्}}\\
	\end{cases}
\end{equation}


\begin{table}[h!]
	\begin{center}
		\begin{tabular}{ |c|c|c| } 
			\hline
			x & 
			\texthindi{तुमुन्}(x) = 	
			$\begin{cases}
				x[\text{\texthindi{ ॠ}}\xrightarrow{2}\text{\texthindi{ अर्}}]+ \phi\\
				x[\text{\texthindi{ ॠ}}\xrightarrow{2}\text{\texthindi{ र्}}]+ \phi\\
				x[\text{\texthindi{ ॠ}}\xrightarrow{2}\text{\texthindi{ अर्}}]+ \text{\texthindi{इ}}\\
			\end{cases}$ & \texthindi{तुमुन्}(x) \\ 
			\hline
			\multirow{3}{*}{\texthindi{दृप्}}
			&\texthindi{दर्प्}
			&\texthindi{दर्पतुम्}\\ 
			&\texthindi{द्रप्}
			&\texthindi{द्रप्तुम्}\\
			&\texthindi{दर्पि}
			&\texthindi{दर्पितुम्}\\
			\hline
		\end{tabular}
		\caption{Examples of Case XIII for multivalued functions of \texthindi{तुमुन्} }
		\label{table:6.36}
	\end{center}
\end{table}

\textbf{Case XIV:}\\
If x $\epsilon$ [\texthindi{द्रुह् मुह् स्नुह् स्निह्}]\\
then\\
\begin{equation}
	\texthindi{तुमुन्}(x) = 	
	\begin{cases}
		x[\texthindi{ इ उ}\xrightarrow{2}\texthindi{ ए ओ}]+ \phi + \texthindi{तुम्}\\
		x[\texthindi{ इ उ }\xrightarrow{2}\texthindi{ ए ओ }]+ \phi + \texthindi{तुम्}\\
		x[\texthindi{ इ उ}\xrightarrow{2}\texthindi{ ए ओ }]+ \texthindi{इ} + \texthindi{तुम्}\\
	\end{cases}
\end{equation}

\begin{table}[h!]
	\begin{center}
		\begin{tabular}{|c|c|} 
			\hline
			x & \texthindi{तुमुन्}(x)\\
			\hline
			\multirow{3}{*}{\texthindi{मुह्}}
			&\texthindi{मोग्धुम्}\\
			&\texthindi{मोढुम्}\\
			&\texthindi{मोहितुम्}\\
			\multirow{3}{*}{\texthindi{स्निह्}}
			&\texthindi{स्नेग्धुम् }\\
			&\texthindi{स्नेढुम् }\\
			&\texthindi{स्नेहितुम्}\\
			\hline
		\end{tabular}
		\caption{Examples of Case XIV for multivalued functions of \texthindi{तुमुन्} }
		\label{table:6.37}
	\end{center}
\end{table}

\textbf{Case XV:}\\
If x $\epsilon$ [\texthindi{कृष् स्पृश् मृश्}]\\
then\\

\begin{equation}
	\texthindi{तुमुन्}(x) = 	
	\begin{cases}
		x[\text{\texthindi{ॠ}}\xrightarrow{2}\text{\texthindi{अर्}}+ \phi + \text{\texthindi{टुम्}}\\
		x[\text{\texthindi{ॠ}}\xrightarrow{2}\text{\texthindi{र्}}+\phi + \text{\texthindi{टुम्}}\\
	\end{cases}
\end{equation}

\begin{table}[h!]
	\begin{center}
		\begin{tabular}{ |c|c|c| } 
			\hline
			x & 
			\texthindi{तुमुन्}(x) = 	
			$\begin{cases}
				x[\text{\texthindi{ॠ}}\xrightarrow{2}\text{\texthindi{अर्}}+ \phi\\
				x[\text{\texthindi{ॠ}}\xrightarrow{2}\text{\texthindi{र्}}+\phi\\
			\end{cases}$ & \texthindi{तुमुन्}(x) \\ 
			\hline
			\multirow{2}{*}{\texthindi{कृष्}}
			&\texthindi{कर्ष्}
			&\texthindi{कर्ष्टुम् }\\
			&\texthindi{क्रष्}
			&\texthindi{क्रष्टुम्}\\	
			\multirow{2}{*}{\texthindi{मृश्}}
			&\texthindi{मर्ष् }
			&\texthindi{मर्ष्टुम् }\\
			&\texthindi{म्रष्}
			&\texthindi{म्रष्टुम्}\\
			\hline
		\end{tabular}
		\caption{Examples of Case XV for multivalued functions of \texthindi{तुमुन्} }
		\label{table:6.38}
	\end{center}
\end{table}

\textbf{Case XVI:}\\
If x $\epsilon$ [\texthindi{वृह् तृह् स्तृह्}]\\
then\\ 
\begin{equation}
	\texthindi{तुमुन्}(x) = 	
	\begin{cases}
		x[\text{\texthindi{ॠ}}\xrightarrow{2}\text{\texthindi{अर्}}+ \text{\texthindi{इ}}+ + \text{\texthindi{तुम्}}\\
		x[\text{\texthindi{ॠ}}\xrightarrow{2}\text{\texthindi{र्}}+ \text{\texthindi{ह्}} \rightarrow \phi + \text{\texthindi{ढुम्}}\\
	\end{cases}
\end{equation}

\begin{table}[h!]
	\begin{center}
		\begin{tabular}{ |c|c|c| } 
			\hline
			x & 
			\texthindi{तुमुन्}(x) = 	
			$\begin{cases}
				x[\text{\texthindi{ॠ}}\xrightarrow{2}\text{\texthindi{अर्}}+ \phi\\
				x[\text{\texthindi{ॠ}}\xrightarrow{2}\text{\texthindi{र्}}+\phi\\
			\end{cases}$ & \texthindi{तुमुन्}(x)\\
			\hline
			\multirow{2}{*}{\texthindi{वृह्}}
			&\texthindi{वर्हि}
			&\texthindi{वर्हितुम्}\\
			&\texthindi{वर्}
			&\texthindi{वर्ढुम्}\\
			\multirow{2}{*}{\texthindi{तृह्}}
			&\texthindi{तर्हि}
			&\texthindi{तर्हितुम्}\\
			&\texthindi{तर्}
			&\texthindi{तर्ढुम्}\\
			\hline
		\end{tabular}
		\caption{Examples of Case XVI for multivalued functions of \texthindi{तुमुन्} }
		\label{table:6.39}
	\end{center}
\end{table}

\textbf{Case XVII:}\\
If x $\epsilon$ \texthindi{कुष्}\\
then\\

\begin{equation}
	\texthindi{तुमुन्}(x) = 	
	\begin{cases}
		x[\text{\texthindi{ उ}}\xrightarrow{2}\text{\texthindi{ ओ}}+ \text{\texthindi{इ}}+ \text{\texthindi{तुम्}}\\
		\text{\texthindi{निष्}} + x[\text{\texthindi{ उ}}\xrightarrow{2}\text{\texthindi{ ओ}}+ \phi + \text{\texthindi{तुम्}}\\
		\text{\texthindi{निष्}} + x[\text{\texthindi{ उ}}\xrightarrow{2}\text{\texthindi{ ओ}}+ \text{\texthindi{इ}}+ \text{\texthindi{तुम्}}\\
	\end{cases}
\end{equation}


\begin{table}[h!]
	\begin{center}
		\begin{tabular}{ |c|c|c| } 
			\hline
			x & 	
			$\begin{cases}
				x[\text{\texthindi{ उ}}\xrightarrow{2}\text{\texthindi{ ओ}}+ \text{\texthindi{इ}}\\
				\text{\texthindi{निष्}} + x[\text{\texthindi{ उ}}\xrightarrow{2}\text{\texthindi{ ओ}}+ \phi \\
				\text{\texthindi{निष्}} + x[\text{\texthindi{ उ}}\xrightarrow{2}\text{\texthindi{ ओ}}+ \text{\texthindi{इ}}\\
			\end{cases}$ 
			&\texthindi{तुमुन्}(x)\\
			\hline
			\multirow{3}{*}{\texthindi{कुष्}}
			&\texthindi{कोषि}
			&\texthindi{कोषितुम्}\\  
			&\texthindi{निष्कोष् }
			&\texthindi{निष्कोष्टुम् }\\
			&\texthindi{निष्कोषि}
			&\texthindi{निष्कोषितुम्}\\
			\hline
		\end{tabular}
		\caption{Examples of Case XVII for multivalued functions of \texthindi{तुमुन्} }
		\label{table:6.40}
	\end{center}
\end{table}


\begin{equation}
	\texthindi{तुमुन्}(x) = 	
	\begin{cases}
		x[\text{\texthindi{ ॠ}}\xrightarrow{2}\text{\texthindi{आर्}}+\text{\texthindi{इ}}+\text{\texthindi{तुम्}}\\
		x[\text{\texthindi{ ॠ}}\xrightarrow{2}\text{\texthindi{आर्}}+\phi+\text{\texthindi{तुम्}}\\
	\end{cases}
\end{equation}

\begin{table}[h!]
	\begin{center}
		\begin{tabular}{ |c|c|c| } 
			\hline
			x & 	
			$\begin{cases}
				x[\text{\texthindi{ ॠ}}\xrightarrow{2}\text{\texthindi{आर्}}+\text{\texthindi{इ}}\\
				x[\text{\texthindi{ ॠ}}\xrightarrow{2}\text{\texthindi{आर्}}+\phi\\
			\end{cases} $& \texthindi{तुमुन्}(x)  \\ 
			\hline
			\multirow{2}{*}{\texthindi{मृज्}}
			&\texthindi{मार्जयि}
			&\texthindi{मार्जयितुम्}\\
			&\texthindi{मार्जि}
			&\texthindi{मार्जितुम्}\\
			\hline
		\end{tabular}
		\caption{Examples of Case XVIII for multivalued functions of \texthindi{तुमुन्} }
		\label{table:6.41}
	\end{center}
\end{table}

\textbf{Case XIX:}\\
If x $\epsilon$ \texthindi{अज्}\\
then\\

\begin{equation}
	\texthindi{तुमुन्}(x) = 	
	\begin{cases}
		x+\text{\texthindi{इ}}+\text{\texthindi{तुम्}}\\
		\text{\texthindi{वे}}+\text{\texthindi{तुम्}}\\
	\end{cases}
\end{equation}


\begin{table}[h!]
	\begin{center}
		\begin{tabular}{|c|c|} 
			\hline
			x & \texthindi{तुमुन्}(x) \\ 
			\hline
			\multirow{2}{*}{\texthindi{अज्}}
			&\texthindi{अजितुम् }\\
			&\texthindi{वेतुम्}\\
			\hline
		\end{tabular}
		\caption{Examples of Case XIX for multivalued functions of \texthindi{तुमुन्} }
		\label{table:6.42}
	\end{center}
\end{table}

\textbf{Case XX:}\\
If x $\epsilon$ [\texthindi{त्रप् क्षम् कल्प् अश् अञ्ज् तन्च्}]\\
then\\
\begin{equation}
	\texthindi{तुमुन्}(x) = 	
	\begin{cases}
		x+\text{\texthindi{इ}}+\text{\texthindi{तुम्}}\\
		x+\phi+\text{\texthindi{तुम्}}\\
	\end{cases}
\end{equation}


\begin{table}[h!]
	\begin{center}
		\begin{tabular}{|c|c|c|} 
			\hline
			x & 
			$\begin{cases}
				x+\texthindi{इ}\\
				x+\phi\\
			\end{cases}$
			&\texthindi{तुमुन्}(x) \\ 
			\hline
			\multirow{2}{*}{\texthindi{त्रप्}}
			&\texthindi{त्रपि}
			&\texthindi{त्रपितुम्}\\
			&\texthindi{त्रप्}
			&\texthindi{त्रप्तुम्}\\
			\multirow{2}{*}{\texthindi{कल्प्}}
			&\texthindi{कल्पि}
			&\texthindi{कल्पितुम्}\\
			&\texthindi{कल्प्}
			&\texthindi{कल्प्तुम्}\\
			\hline
		\end{tabular}
		\caption{Examples of Case XX for multivalued functions of \texthindi{तुमुन्} }
		\label{table:6.43}
	\end{center}
\end{table}

\textbf{Case XXI:}\\
If x $\epsilon$ \texthindi{धूप्}\\
then\\
\begin{equation}
	\texthindi{तुमुन्}(x) = 	
	\begin{cases}
		x+\text{\texthindi{आय्}}+\text{\texthindi{इ}}+\text{\texthindi{तुम्}}\\
		x+\text{\texthindi{अय्}}+\text{\texthindi{इ}}+\text{\texthindi{तुम्}} \\ 
	\end{cases}
\end{equation}

\begin{table}[h!]
	\begin{center}
		\begin{tabular}{|c|c|c|} 
			\hline
			x & 
			$\begin{cases}
				x+\text{\texthindi{आय्}}\\
				x+\text{\texthindi{अय्}}\\
			\end{cases}$
			&\texthindi{तुमुन्}(x) \\ 
			\hline
			\multirow{2}{*}{\texthindi{धूप्}}
			&\texthindi{धूपाय्}
			&\texthindi{धूपायितुम्}\\
			&\texthindi{धूपय्}
			&\texthindi{धूपयितुम्}\\
			\hline
		\end{tabular}
		\caption{Examples of Case XXI for multivalued functions of \texthindi{तुमुन्} }
		\label{table:6.44}
	\end{center}
\end{table}

\textbf{Case XXII:}\\
If x $\epsilon$ \texthindi{पण् विच्छ्}\\
then\\
\begin{equation}
	\text{\texthindi{तुमुन्}}(x) = 	
	\begin{cases}
		x+\text{\texthindi{आय्}}+\text{\texthindi{इ}}+\text{\texthindi{तुम्}}\\
		x+\text{\texthindi{इ}}+\text{\texthindi{तुम्}}\\
	\end{cases}
\end{equation}

\begin{table}[h!]
	\begin{center}
		\begin{tabular}{|c|c|c|} 
			\hline
			x & 
			$\begin{cases}
				x+\text{\texthindi{आय्}}\\
				x+\text{\texthindi{अय्}}\\
			\end{cases}$
			&\texthindi{तुमुन्}(x) \\ 
			\hline
			\multirow{2}{*}{\texthindi{पण्}}
			&\texthindi{पणाय्}
			&\texthindi{पणायितुम्}\\
			&\texthindi{पण्}
			&\texthindi{पणितुम्}\\
			\hline
		\end{tabular}
		\caption{Examples of Case XXII for multivalued functions of \texthindi{तुमुन्} }
		\label{table:6.45}
	\end{center}
\end{table}

\textbf{Case XXIII:}\\
If x $\epsilon$ \texthindi{सह्}\\
then\\
\begin{equation}
	\texthindi{तुमुन्}(x) =	
	\begin{cases}
		x\text{\texthindi{अ}}\xrightarrow{2}\text{\texthindi{ओ}}+\text{\texthindi{ह्}}\xrightarrow{1}\phi+\text{\texthindi{ढुम्}}\\
		x+\text{\texthindi{इ}}+\text{\texthindi{तुम्}}\\
	\end{cases}
\end{equation}

\begin{table}[h!]
	\begin{center}
		\begin{tabular}{|c|c|c|} 
			\hline
			x & 
			$\begin{cases}
				x\text{\texthindi{अ}}\xrightarrow{2}\text{\texthindi{ओ}}+\text{\texthindi{ह्}}\xrightarrow{1}\phi\\
				x+\text{\texthindi{इ}}\\
			\end{cases}$
			&\texthindi{तुमुन्}(x) \\ 
			\hline
			\multirow{2}{*}{\texthindi{सह्}}
			&\texthindi{सो}
			&\texthindi{सोढुम्}\\
			&\texthindi{सहि}
			&\texthindi{सहितुम्}\\
			\hline
		\end{tabular}
		\caption{Examples of Case XXIII for multivalued functions of \texthindi{तुमुन्} }
		\label{table:6.46}
	\end{center}
\end{table}

\textbf{Case XXIV:}\\
If x $\epsilon$ \texthindi{बध्}\\
then\\
\begin{equation}
	\texthindi{तुमुन्}(x) =	
	\begin{cases}
		x\text{\texthindi{अ}}\xrightarrow{2}\text{\texthindi{आ}}+\text{\texthindi{अय्}}+\text{\texthindi{इ}}+\text{\texthindi{तुम्}}\\
		\text{\texthindi{बी}} + \text{\texthindi{ब्}}\xrightarrow{2}\text{\texthindi{भ्}}, [\text{\texthindi{ध्}}\xrightarrow{2}\text{\texthindi{त्}} ]+\text{\texthindi{स्}}+\text{\texthindi{इ}}+\text{\texthindi{तुम्}} \\ \end{cases}
\end{equation}

\begin{table}[h!]
	\begin{center}
		\begin{tabular}{|c|c|c|} 
			\hline
			x & 
			$\begin{cases}
				x\text{\texthindi{अ}}\xrightarrow{2}\text{\texthindi{आ}}+\text{\texthindi{अय्}}\\
				\text{\texthindi{बी}} + \text{\texthindi{ब्}}\xrightarrow{2}\text{\texthindi{भ्}}, [\text{\texthindi{ध्}}\xrightarrow{2}\text{\texthindi{त्}} ]+\text{\texthindi{स्}}\\
			\end{cases}$
			&\texthindi{तुमुन्}(x) \\ 
			\hline
		\end{tabular}
		\caption{Examples of Case XXIV for multivalued functions of \texthindi{तुमुन्} }
		\label{table:6.47}
	\end{center}
\end{table}

\textbf{Case XXV:}\\
If x $\epsilon$ [\texthindi{दान् शान्}]\\
then\\
\begin{equation}
	\texthindi{तुमुन्}(x) =	
	\begin{cases}
		x+\text{\texthindi{अय्}}+\text{\texthindi{इ}}+\text{\texthindi{तुम्}}\\
		\text{\texthindi{दी/शी}} + x + \text{\texthindi{स्}}+\text{\texthindi{इ}}+\text{\texthindi{तुम्}} \\ 
	\end{cases}
\end{equation}

\begin{table}[h!]
	\begin{center}
		\begin{tabular}{|c|c|c|} 
			\hline
			x & 
			$\begin{cases}
				x+\text{\texthindi{अय्}}+\text{\texthindi{इ}}\\
				\text{\texthindi{दी/शी}} + x + \text{\texthindi{स्}}+\text{\texthindi{इ}} \\
			\end{cases}$
			&\texthindi{तुमुन्}(x) \\
			\hline
			\multirow{2}{*}{\texthindi{शान्}}
			&\texthindi{शानयि}
			&\texthindi{शानयितुम्}\\
			&\texthindi{शीशांसि}
			&\texthindi{शीशांसितुम्}\\
			\multirow{2}{*}{\texthindi{दान्}}
			&\texthindi{दानयि}
			&\texthindi{दानयितुम्}\\
			&\texthindi{दीदांसि}
			&\texthindi{दीदांसितुम्}\\
			\hline
		\end{tabular}
		\caption{Examples of Case XXV for multivalued functions of \texthindi{तुमुन्} }
		\label{table:6.48}
	\end{center}
\end{table}

\textbf{Case XXVI:}\\
If x $\epsilon$ \texthindi{नश्}\\
then\\
\begin{equation}
	\texthindi{तुमुन्}(x) =	
	\begin{cases}
		x[\text{\texthindi{अ}}\xrightarrow{2}\text{\texthindi{आ}}]+\text{\texthindi{इ}}+\text{\texthindi{तुम्}}\\
		x + \phi +\text{\texthindi{तुम्}} \\
	\end{cases}
\end{equation}


\begin{table}[h!]
	\begin{center}
		\begin{tabular}{|c|c|c|} 
			\hline
			x & 
			$\begin{cases}
				x[\text{\texthindi{अ}}\xrightarrow{2}\text{\texthindi{आ}}]+\text{\texthindi{इ}}\\
				x + \phi \\
			\end{cases}$
			&\texthindi{तुमुन्}(x) \\ 
			\hline
			\multirow{2}{*}{\texthindi{नश्}}
			&\texthindi{नाशि}
			&\texthindi{नाशितुम्}\\
			&\texthindi{नश्}
			&\texthindi{नष्टुम् }\\
			\hline
		\end{tabular}
		\caption{Examples of Case XXVI for multivalued functions of \texthindi{तुमुन्} }
		\label{table:6.49}
	\end{center}
\end{table}

\textbf{Type 2: x belongs to the 10th gaṇa.}\\
\begin{equation}
	\text{\texthindi{तुमुन्}}(x) = \text{\texthindi{तुमुन्}}( \text{\texthindi{णिच्}} (x))\\
\end{equation}

\textbf{Case I:}\\
If x $\epsilon$ [\texthindi{चिन्त् स्फुन्ड् मिन्द् लन्ड् तुन्ज् पिन्ज् पन्थ् छन्द् खन्ड् कुन्ड् कुन्ड् गुन्ड् खुन्ड् वन्ट् मन्ड् भन्ड् पन्ड् पंस् चन्प् क्षन्प् छन्ज् चुन्ब् टन्क् शुन्ठ् पन्च् कुन्ब् लुन्ब् तुन्ब् चुन्ट् जन्स् पिन्ड्  दन्श् दंस् वन्च् जन्भ् जस् तन्स् अञ्च् लिन्ग् ग्रस् दल् पट् तुन्ज् मिन्ज् पिन्ज् लुन्ज् भन्ज् लन्घ् त्रन्स् पिन्स् कुन्स् दन्श् कुन्श् घट् घण्ट् वृन्ह् बर्ह् वल्ह् चीव् नद् तर्क् वृत् वृध् लन्ज् अन्ज् दन्स् भृन्श् रुन्श् शीक् रुन्स् नट् पुण्ट् रन्घ् लन्घ् अंह् रंह् मंह् लन्ड् तड् नल् पूर् अर्च् सह् ईर् श्रथ् ग्रन्थ् शीक् चीक् अर्द् हिंस् अर्ह् शुन्ध् छद् श्रन्थ् ग्रन्थ् आप् तन् वद् वच् मान् गर्ह् मार्ग् कण्ठ् गर्व मूत्र वल्क चित्र मिश्र छिद्र दण्ड दुःख वर्ण पर्ण विष्क तुत्थ जस् लोक् लोच् ज्रि}]\\

\begin{equation}
	\text{\texthindi{तुमुन्}}(x) =	
	\begin{cases}
		\text{\texthindi{तुमुन्}}( \text{\texthindi{णिच्}} (x))\\
		x + \text{\texthindi{इ}} +\text{\texthindi{तुम्}} \\ 
	\end{cases}
\end{equation}

\begin{table}[h!]
	\begin{center}
		\begin{tabular}{|c|c|} 
			\hline
			x & \texthindi{तुमुन्}(x) \\ 
			\hline
			\multirow{2}{*}{\texthindi{चिन्त्}}
			&\texthindi{चिन्तयितुम्}\\ 
			&\texthindi{चिन्तितुम्}\\
			\multirow{2}{*}{\texthindi{स्फुन्ड्}}
			&\texthindi{स्फुण्डयितुम्}\\ 
			&\texthindi{स्फुण्डितुम्}\\
			\multirow{2}{*}{\texthindi{मान्}}
			&\texthindi{मानयितुम्}\\ 
			&\texthindi{मानितुम्}\\
			\hline
		\end{tabular}
		\caption{Examples of Case I for multivalued functions of \texthindi{तुमुन्}if x is Type 2 }
		\label{table:6.50}
	\end{center}
\end{table}

\textbf{Case II:}\\
If x $\epsilon$ \texthindi{पॄ (चुरादिगण)}\\
then\\
\begin{equation}
	\texthindi{तुमुन्}(x) =	
	\begin{cases}
		\text{\texthindi{तुमुन्}}( \text{\texthindi{णिच्}} (x))\\
		x[\text{\texthindi{ऋ}}\xrightarrow{1}\text{\texthindi{अर्}}] + \text{\texthindi{इ}} +\text{\texthindi{तुम्}} \\
	\end{cases}
\end{equation}

\begin{table}[h!]
	\begin{center}
		\begin{tabular}{|c|c|} 
			\hline
			x & \texthindi{तुमुन्}(x) \\ 
			\hline
			\multirow{2}{*}{\texthindi{पॄ}}
			&\texthindi{पारयितुम्}\\
			&\texthindi{परितुम्}\\
			\hline
		\end{tabular}
		\caption{Examples of Case II for multivalued functions of \texthindi{तुमुन्}if x is Type 2 }
		\label{table:6.51}
	\end{center}
\end{table}

\textbf{Case III:}\\
If x $\epsilon$ \texthindi{सद्}\\
then\\
\begin{equation}
	\text{\texthindi{तुमुन्}}(x) =	
	\begin{cases}
		\text{\texthindi{आ}}+\text{\texthindi{तुमुन्}}( \text{\texthindi{णिच्}} (x))\\
		\text{\texthindi{आ}}+x[\text{\texthindi{ इ\/ई उ\/ऊ ऋ\/ॠ }}\xrightarrow{1}\text{\texthindi{ ए ओ अर्}}] + \phi +\text{\texthindi{तुम्}} \\ 
	\end{cases}
\end{equation}

\begin{table}[h!]
	\begin{center}
		\begin{tabular}{|c|c|} 
			\hline
			x & \texthindi{तुमुन्}(x) \\ 
			\hline
			\multirow{2}{*}{\texthindi{सद्}}
			&\texthindi{आसादयितुम्}\\ 
			&\texthindi{आसत्तुम्}\\
			\hline
		\end{tabular}
		\caption{Examples of Case III for multivalued functions of \texthindi{तुमुन्}if x is Type 2 }
		\label{table:6.52}
	\end{center}
\end{table}


\textbf{Case IV:}\\
If x $\epsilon$ [\texthindi{दिव् घुष् शृध् पुष् पुट् लुट् गुप् पुथ् कुप् रुट् रुज् युज् पृच् वृज् रिच् शिष् तृप् छृद् दृभ् दृभ् जुष् प्री भू (भू} 10th gaṇa) \texthindi{मृष् धृष्}]\\
then\\
\begin{equation}
	\text{\texthindi{तुमुन्}}(x) =	
	\begin{cases}
		\text{\texthindi{आ}}+\text{\texthindi{तुमुन्}}( \text{\texthindi{णिच्}} (x))\\
		x[\text{\texthindi{ इ\/ई उ\/ऊ ऋ}}\xrightarrow{2}\text{\texthindi{ ए ओ अर्}}] + \text{\texthindi{इ}} +\text{\texthindi{तुम्}} \\
	\end{cases}
\end{equation}

\begin{table}[h!]
	\begin{center}
		\begin{tabular}{|c|c|} 
			\hline
			x & \texthindi{तुमुन्}(x) \\ 
			\hline
			\multirow{2}{*}{\texthindi{ दिव्}}
			&\texthindi{ देवयितुम्}\\ 
			&\texthindi{ देवितुम्}\\

			\multirow{2}{*}{\texthindi{ पृच्}}
			&\texthindi{ पर्चयितुम् }\\
			&\texthindi{ पर्चितुम्}\\
			\hline
		\end{tabular}
		\caption{Examples of Case IV for multivalued functions of \texthindi{तुमुन्}if x is Type 2 }
		\label{table:6.53}
	\end{center}
\end{table}

\textbf{Case V:}\\
If x $\epsilon$ \texthindi{चि (चिञ् चयने)}\\
then\\
\begin{equation}
	\text{\texthindi{तुमुन्}}(x) =	
	\begin{cases}
		\text{\texthindi{आ}}+\text{\texthindi{तुमुन्}}( \text{\texthindi{णिच्}} (x))\\
		x[\text{\texthindi{इ}}\xrightarrow{1}\text{\texthindi{ए}}] + \text{\texthindi{तुम्}}\\
		x[\text{\texthindi{इ}}\xrightarrow{1}\text{\texthindi{आ}}] +\text{\texthindi{प्}}+\text{\texthindi{अय्}}+\text{\texthindi{इ}}+ \text{\texthindi{तुम्}}\\
	\end{cases}
\end{equation}

\begin{table}[h!]
	\begin{center}
		\begin{tabular}{|c|c|} 
			\hline
			x & \texthindi{तुमुन्}(x) \\ 
			\hline
			\multirow{3}{*}{\texthindi{चि}}
			&\texthindi{चाययितुम्}\\ 
			&\texthindi{चापयितुम् }\\
			&\texthindi{चेतुम्}\\
			\hline
		\end{tabular}
		\caption{Examples of Case V for multivalued functions of \texthindi{तुमुन्}if x is Type 2 }
		\label{table:6.54}
	\end{center}
\end{table}

\textbf{Case VI:}\\
If x $\epsilon$ \texthindi{उध्रस्}\\
then\\
\begin{equation}
	\text{\texthindi{तुमुन्}}(x) =	
	\begin{cases}
		\text{\texthindi{आ}}+\text{\texthindi{तुमुन्}}( \text{\texthindi{णिच्}} (x))\\
		x+\text{\texthindi{इ}} + \text{\texthindi{तुम्}}\\
		x[\text{\texthindi{उ}}\xrightarrow{4}\phi] +\text{\texthindi{इ}}+ \text{\texthindi{तुम्}}\\
	\end{cases}
\end{equation}
\begin{table}[h!]
	\begin{center}
		\begin{tabular}{|c|c|} 
			\hline
			x & \texthindi{तुमुन्}(x) \\ 
			\hline
			\multirow{3}{*}{\texthindi{उध्रस्}}
			&\texthindi{उध्रासयितुम् }\\ 
			&\texthindi{उध्रसितुम् }\\
			&\texthindi{ध्रसितुम्}\\
			\hline
		\end{tabular}
		\caption{Examples of Case VI for multivalued functions of \texthindi{तुमुन्}if x is Type 2 }
		\label{table:6.55}
	\end{center}
\end{table}

\textbf{Case VII:}\\
If x $\epsilon$ \texthindi{धूप् विच्छ्}\\
then\\
\begin{equation}
	\text{\texthindi{तुमुन्}}(x) =	
	\begin{cases}
		\text{\texthindi{आ}}+\text{\texthindi{तुमुन्}}( \text{\texthindi{णिच्}} (x))\\
		x+\text{\texthindi{इ}} + \text{\texthindi{तुम्}}\\
		x+\text{\texthindi{आय्}}+\text{\texthindi{इ}}+ \text{\texthindi{तुम्}}\\
	\end{cases}
\end{equation}


\begin{table}[h!]
	\begin{center}
		\begin{tabular}{|c|c|} 
			\hline
			x & \texthindi{तुमुन्}(x) \\
			\hline
			\multirow{3}{*}{\texthindi{धूप्}}
			&\texthindi{धूपयितुम् }\\ 
			&\texthindi{धूपितुम् }\\
			&\texthindi{धूपायितुम्}\\
			\multirow{3}{*}{\texthindi{विच्छ्}}
			&\texthindi{विच्छयितुम्}\\ 
			&\texthindi{विच्छितुम्}\\
			&\texthindi{विच्छाययितुम्}\\
			\hline
		\end{tabular}
		\caption{Examples of Case VII for multivalued functions of \texthindi{तुमुन्}if x is Type 2 }
		\label{table:6.56}
	\end{center}
\end{table}

\textbf{Case VIII:}\\
If x $\epsilon$ \texthindi{जि}\\
then\\
\begin{equation}
	\text{\texthindi{तुमुन्}}(x) =	
	\begin{cases}
		\text{\texthindi{आ}}+\text{\texthindi{तुमुन्}}( \text{\texthindi{णिच्}} (x))\\
		x+[\text{\texthindi{इ}}\xrightarrow{1}\text{\texthindi{ए}}] + \text{\texthindi{तुम्}}\\
	\end{cases}
\end{equation}

\begin{table}[h!]
	\begin{center}
		\begin{tabular}{|c|c|} 
			\hline
			x & \texthindi{तुमुन्}(x) \\ 
			\hline
			\multirow{2}{*}{\texthindi{जि}}
			&\texthindi{जाययितुम्}\\ 
			&\texthindi{जेतुम्}\\
			\hline
		\end{tabular}
		\caption{Examples of Case VIII for multivalued functions of \texthindi{तुमुन्}if x is Type 2 }
		\label{table:6.66}
	\end{center}
\end{table}

\textbf{Case IX:}\\
If x $\epsilon$ \texthindi{चि}\\
then\\

\begin{equation}
	\text{\texthindi{तुमुन्}} (x) = 
	\begin{cases}
		\text{\texthindi{तुमुन्(णिच्}}(x))\\
		x[\text{\texthindi{इ}} \xrightarrow{1} \text{\texthindi{आ}} + \text{\texthindi{प्}} + \text{\texthindi{अय्}} + \text{\texthindi{इ}} \text{\texthindi{तुम्}}\\
		x[\text{\texthindi{इ}}] \xrightarrow{1} \text{\texthindi{ए}}+\text{\texthindi{तुम}}
	\end{cases}
\end{equation}

\begin{table}[h!]
	\begin{center}
		\begin{tabular}{|c|c|} 
			\hline
			x & \texthindi{तुमुन्}(x) \\ 
			\hline
			\multirow{3}{*}{\texthindi{चि}}
			&\texthindi{चाययितुम्}\\ 
			&\texthindi{चापयितुम्}\\
			&\texthindi{चेतुम्}\\
			\hline
		\end{tabular}
		\caption{Examples of Case IX for multivalued functions of \texthindi{तुमुन्}if x is Type 2 }
		\label{table:6.67}
	\end{center}
\end{table}

\textbf{Case X:}\\
If x $\epsilon$ \texthindi{स्वद्}\\
then\\
\begin{equation}
	\text{\texthindi{तुमुन्}}(x)=\begin{cases}
		\text{\texthindi{तुमुन्(णिच्}}(x))\\
		x[\text{\texthindi{अ}} \xrightarrow{2} \text{\texthindi{आ}} + \text{\texthindi{इ}} \text{\texthindi{तुम्}}\\
	\end{cases}
\end{equation}

\begin{table}[h!]
	\begin{center}
		\begin{tabular}{|c|c|} 
			\hline
			x & \texthindi{तुमुन्}(x) \\ 
			\hline
			\multirow{2}{*}{\texthindi{स्वद्}}
			&\texthindi{स्वादयितुम्}\\ 
			&\texthindi{स्वादितुम्}\\
			\hline
		\end{tabular}
		\caption{Examples of Case X for multivalued functions of \texthindi{तुमुन्}if x is Type 2 }
		\label{table:6.68}
	\end{center}
\end{table}

\textbf{Case XI:}\\
If x $\epsilon$ \texthindi{ली}\\
then\\
\begin{equation}
	\text{\texthindi{तुमुन्}}(x)=\begin{cases}
		\text{\texthindi{तुमुन्(णिच्}}(x))\\
		x[\text{\texthindi{इ}} \xrightarrow{1} \text{\texthindi{ए}} + \text{\texthindi{तुम्}}\\
	\end{cases}
\end{equation}

\begin{table}[h!]
	\begin{center}
		\begin{tabular}{|c|c|} 
			\hline
			x & \texthindi{तुमुन्}(x) \\ 
			\hline
			\multirow{2}{*}{\texthindi{ली}}
			&\texthindi{लाययितुम्}\\ 
			&\texthindi{लेतुम्}\\

			\hline
		\end{tabular}
		\caption{Examples of Case XI for multivalued functions of \texthindi{तुमुन्}if x is Type 2 }
		\label{table:6.69}
	\end{center}
\end{table}

\textbf{Case XII:}\\
If x $\epsilon$ \texthindi{वृ जृ}\\
then\\
\begin{equation}
	\text{\texthindi{तुमुन्}} (x)=\begin{cases}
		\text{\texthindi{तुमुन्(णिच्}}(x))\\
		x[\text{\texthindi{ऋ}} \xrightarrow{1} \text{\texthindi{अर्}} +\text{\texthindi{इ}} +\text{\texthindi{तुम्}}\\
		x[\text{\texthindi{ऋ}} \xrightarrow{1}\text{\texthindi{अर्}} + \text{\texthindi{ई}} + \text{\texthindi{तुम्}}\\
	\end{cases}
\end{equation}

\begin{table}[h!]
	\begin{center}
		\begin{tabular}{|c|c|} 
			\hline
			x & \texthindi{तुमुन्}(x)\\
			\hline
			\multirow{3}{*}{\texthindi{वृ}}
			&\texthindi{वारयितुम्}\\ 
			&\texthindi{वरितुम्}\\
			&\texthindi{वरीतुम्}\\

			\multirow{3}{*}{\texthindi{जृ}}
			&\texthindi{जारयितुम्}\\ 
			&\texthindi{जरितुम्}\\
			&\texthindi{जरीतुम्}\\
			\hline
		\end{tabular}
		\caption{Examples of Case XII for multivalued functions of \texthindi{तुमुन्}if x is Type 2 }
		\label{table:6.70}
	\end{center}
\end{table}

\textbf{Case XIII:}\\
If x $\epsilon$ \texthindi{तप्}\\
then\\
\begin{equation}
	\text{\texthindi{तुमुन्}}(x)=\begin{cases}
		\text{\texthindi{तुमुन्(णिच्}}(x))\\
		x+\text{\texthindi{तुम्}}\\
	\end{cases}
\end{equation}

\begin{table}[h!]
	\begin{center}
		\begin{tabular}{|c|c|} 
			\hline
			x & \texthindi{तुमुन्}(x) \\ 
			\hline
			\multirow{2}{*}{\texthindi{तप्}}
			&\texthindi{तापयितुम्}\\ 
			&\texthindi{तप्तुम्}\\
			\hline
		\end{tabular}
		\caption{Examples of Case XIII for multivalued functions of \texthindi{तुमुन्}if x is Type 2 }
		\label{table:6.71}
	\end{center}
\end{table}

\textbf{Case XIV:}\\
If x $\epsilon$ \texthindi{धू}\\
then\\
\begin{equation}
	\text{\texthindi{तुमुन्}}(x)=
	\begin{cases}
		x + \text{\texthindi{न्}}+\text{\texthindi{अय्}}+\text{\texthindi{इ}}+\text{\texthindi{तुम्}}\\
		x[\text{\texthindi{ऊ}}\xrightarrow{1}\text{\texthindi{ओ}}+\text{\texthindi{इ}}+\text{\texthindi{तुम्}}\\
		x[\text{\texthindi{ऊ}}\xrightarrow{1}\text{\texthindi{ओ}}+\text{\texthindi{तुम्}}\\
	\end{cases}
\end{equation}

\begin{table}[h!]
	\begin{center}
		\begin{tabular}{|c|c|} 
			\hline
			x & \texthindi{तुमुन्}(x)\\
			\hline
			\multirow{3}{*}{\texthindi{धू}}
			&\texthindi{धूनयितुम्}\\ 
			&\texthindi{धवितुम्}\\
			&\texthindi{धोतुम्}\\
			\hline
		\end{tabular}
		\caption{Examples of Case XIV for multivalued functions of \texthindi{तुमुन्}if x is Type 2 }
		\label{table:6.72}
	\end{center}
\end{table}

\textbf{Case XV:}\\
If x $\epsilon$ \texthindi{मृज्}\\
then\\
\begin{equation}
	\text{\texthindi{तुमुन्}}(x)=\begin{cases}
		\text{\texthindi{तुमुन्(णिच्}}(x))\\
		x[\text{\texthindi{ऋ}} \xrightarrow{2} \text{\texthindi{आर्}} + \text{\texthindi{इ}} + \text{\texthindi{तुम्}}\\
	\end{cases}
\end{equation}

\begin{table}[h!]
	\begin{center}
		\begin{tabular}{|c|c|} 
			\hline
			x & \texthindi{तुमुन्}(x)\\
			\hline
			\multirow{2}{*}{\texthindi{मृज्}}
			&\texthindi{मार्जयितुम्}\\ 
			&\texthindi{मार्जितुम्}\\
			\hline
		\end{tabular}
		\caption{Examples of Case XV for multivalued functions of \texthindi{तुमुन्}if x is Type 2 }
		\label{table:6.73}
	\end{center}
\end{table}

\textbf{Case XVI:}\\
If x $\epsilon$ \texthindi{पत्}\\
then\\
\begin{equation}
	\text{\texthindi{तुमुन्}}(x)=
	\begin{cases}
		\text{\texthindi{तुमुन्}}(\text{\texthindi{णिच्}}(x))\\
		x+\text{\texthindi{अय्}}+\text{\texthindi{इ}}+\text{\texthindi{तुम्}}\\
		x+\text{\texthindi{इ}}+\text{\texthindi{तुम्}}\
	\end{cases}
\end{equation}

\begin{table}[h!]
	\begin{center}
		\begin{tabular}{|c|c|} 
			\hline
			x & \texthindi{तुमुन्}(x)\\
			\hline
			\multirow{3}{*}{\texthindi{पत्}}
			&\texthindi{पातयितुम्}\\ 
			&\texthindi{पतयितुम्}\\
			&\texthindi{पतितुम्}\\
			\hline
		\end{tabular}
		\caption{Examples of Case XVI for multivalued functions of \texthindi{तुमुन्}if x is Type 2 }
		\label{table:6.74}
	\end{center}
\end{table}

\textbf{Case XVII:}\\
If x $\epsilon$ \texthindi{कत्र्}\\
then\\
\begin{equation}
	\text{\texthindi{तुमुन्}}(x)=\begin{cases}
		\text{\texthindi{तुमुन्(णिच्}}(x))\\
		x+\text{\texthindi{इ}}+\text{\texthindi{तुम्}}\\
		x+\text{\texthindi{इ}}+\text{\texthindi{तुम्}}\
	\end{cases}
\end{equation}

\begin{table}[h!]
	\begin{center}
		\begin{tabular}{|c|c|} 
			\hline
			x & \texthindi{तुमुन्}(x)\\
			\hline
			\multirow{2}{*}{\texthindi{कत्र}}
			&\texthindi{कत्रयितुम्}\\ 
			&\texthindi{कत्रितुम्}\\
			\hline
		\end{tabular}
		\caption{Examples of Case XVII for multivalued functions of \texthindi{तुमुन्}if x is Type 2 }
		\label{table:6.75}
	\end{center}
\end{table}

\textbf{Case XVIII:}\\
If x $\epsilon$ \texthindi{गुप्} (10 gaṇa)\\
then\\
\begin{equation}
	\text{\texthindi{तुमुन्}}(x)=
	\begin{cases}
		\text{\texthindi{णिच्}}(x)+\text{\texthindi{इ}}+\text{\texthindi{तुम्}}\\
		\text{\texthindi{णिच्}}(x) + \text{\texthindi{तुम्}}\
	\end{cases}
\end{equation}

\begin{table}[h!]
	\begin{center}
		\begin{tabular}{|c|c|} 
			\hline
			x & \texthindi{तुमुन्}(x)\\
			\hline
			\multirow{2}{*}{\texthindi{गुप्}}
			&\texthindi{गोपयितुम्}\\ 
			&\texthindi{गोपितुम्}\\
			\hline
		\end{tabular}
		\caption{Examples of Case XVIII for multivalued functions of \texthindi{तुमुन्}if x is Type 2 }
		\label{table:6.76}
	\end{center}
\end{table}

\textbf{Note:} For \texthindi{तव्यत्} Pratyaya the same Cases are used only by replacing \texthindi{तुम्} by \texthindi{तव्य}.\\

Now we will look into two more sanādi pratyayas, the yaṅ and the san pratyayas.